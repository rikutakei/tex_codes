\documentclass[a4paper,12pt]{article}

\usepackage[utf8]{inputenc}
\usepackage[T1]{fontenc}
\usepackage{natbib}
\usepackage{geometry}
\usepackage{fixltx2e}
\usepackage{enumitem}
\usepackage{amsmath}

\geometry{
    a4paper,
    left=2.5cm,
    right=2.5cm,
    top=2.5cm,
    bottom=2.5cm
}

\begin{document}

\begin{center}
\LARGE{Rhodopsin Notes}
\end{center}

\section*{Rhodopsin}

\subsection*{\normalsize{Kim et al. 2004}}

Retinal is covalently bound to the receptor via the protonated Schiff linkage to the lysine 296 (K296), which is on the transmembrane helix 7.
The buried Schiff base is stabilised by an electrostatic interaction called the salt bridge with the carboxylate group of the glutamic acid 113 (E113).
Isomerization of the rhodopsin molecule from cis to all trans causes this salt bridge to be disrupted and leads to the conformational changes that activate the rhodopsin receptor.

Direct (E113Q) or indirect (G90D or A292E) mutations causes the disruption of this salt bridge and leads to a constitutive activation of the retinal-unbound form of rhodopsin, opsin.
The E113Q mutation will obviously lead to the disruption of the salt bridge, as this amino acid is essential for the stabilisation of the Schiff base that the retinal is bound to.
G90D and A292E mutations cause these amino acid side chains to competitively interact with the Schiff base, and thereby rupturing the salt bridge.

There are other mutations that causes a constitutive activation of opsin that does not seem to affect the salt bridge.
These mutations are M257Y and E134Q.
This study was investigating what the structural changes associated with each of these mutations.

\noindent\rule{\textwidth}{0.4pt}

The results showed that the even though the salt bridge mutations caused a change in conformation of rhodopsin and activated the receptor to a degree, this was not enough to activate the receptor in dark state.
Likewise, the non-salt bridge mutations caused a change in conformation, but not enough to fully activate the receptor.
Both salt bridge mutations and non-salt bridge mutations were not additive, meaning that more than two salt bridge mutations did not significantly increased the activation, likewise with non-salt bridge mutations.
These results showed that one type of mutation was not enough to induce the conformation changes required for full activation of the receptor.

However, when the salt bridge mutation was combined with the M257Y (non-salt bridge mutation), it showed a significant increase in the activity of the rhodopsin (with retinal bound), almost to a level of light induced state, likely caused by the TM6 movement.

\subsection*{\normalsize{Ernst et al 2005}}

This study was to determine what component of retinal was important for the conformatioal changes caused by light induction and how the information was effectively transduced.
It was known that in the meta II state of rhodopsin, the polyene chain of all trans retinal was firmly locked in this state and prevented the re-isomerisation of the all trans retinal into the cis retinal.
It was thought that the cyclic part of the retinal was important for the movement of the TM6 out of the helix bundle and activate the receptor.
They used acyclic (ac) retinal to test the role of this cyclic part of the retinal.

\noindent\rule{\textwidth}{0.4pt}

They found that the ac retinal was able to form an active rhodopsin, but not as efficiently as normal retinal.
This was due to the change in pKa observed between the normal rhodopsin (pKa = 7.5) and the ac retinal bound rhodopsin (pKa = 5.2).
In order to achieve an active state, the rhodopsin must be able to be protonated in order to hydrolyse and release the trans retinal.
This means that it is much harder for the ac rhodopsin to protonate, as it has lower pKa.

They also found that even when ac rhodopsin becomes active, it was only short-lived compared to the normal rhodopsin.
After its activation, the polyene chain is tightly fitted in the receptor and is only able to be translocated/dissociated from the complex.
This translatory movement is made easier for the ac retinal due to the loss of the cyclic ring structure, and also the acyclic structure is not able to interact as much as the TM6, therefore not able to maintain the active conformation as much as the all trans retinal.

Taken together, the acyclic retinal causes less formation of the active meta II as well as the increased degradation of the meta II.

\section*{Transducin/G-protein}

\subsection*{\normalsize{Yeagle and Albert 2003}}

This study was looking at how the interaction of the G-protein (transducin) with rhodopsin activated it and produce signals.
Transducin is a heterotrimeric protein that consist of an $\alpha$, $\beta$ and $\gamma$ subunits.
Beta and gamma subunits are tightly bound to one another and cannot be dissociated (unless under extreme denaturing condition).
The alpha subunit has GTPase domain at the interface of the complex with beta/gamma subunit, and have positive N-term helix and C-terminal end which binds to the negatively charged groove that is opened up upon light activation of rhodopsin (i.e. in meta II state).

The C-terminus of alpha subunit binds into the groove formed upon activation.
This binding places the lysine (K341) and aspartic acid (D341) near the conserved D(E)RY sequence of the rhodopsin, and stabilises the active conformation of rhodopsin in meta II state.

N-terminus of alpha subunit is ordered when it is bound to the beta/gamma subunit of the G-protein and when it is bound to GDP instead of GTP.
This flexible N-term helix undergoes a significant change in conformation upon binding to the receptor, and allows the GDP/GTP exchange to occur, and also favours the release of the beta/gamma subunit dissociation.

\subsection*{Bohm et al 1997}

Review of the structures of transducin.
The alpha subunit has a GTPase domain which is at the interface of the alpha-beta/gamma subunit.
This GTPase domain contains three switches (switches I to III) which are conformationally sensitive to the gamma phosphate of GDP/GTP.
These switches are bound/interacts with the GTP, but these are flexible when GDP is bound.
The switch region, especially switch II, interacts with the beta/gamma subunit when there is no phosphate (i.e. in the GDP bound form).
However, when GTP is bound, the switches are more tightly bound to the GDP than the beta/gamma subunits and decreases the binding site for the beta/gamma subunits, and therefore the complex dissociates.

\begin{description}
\item[Phosducin] is a regulator of beta/gamma subunits and it prevents the re-association with the alpha subunit.
Since the beta/gamma subunits are required for the GDP/GTP exchange, phosducin prevents the activation of the alpha subunit.
\item[Regulator of G-protein Signalling (RGS)] binds to the switch regions and increases the alpha subunit's GTPase acitivity (therefore decreases the duration).
\end{description}

\subsection*{Slep et al 2001}

This paper looked at how the alpha subunit of the transducin was able to activate the peophodiesterase (PDE) enzyme, which converts cGMP into GMP and closes the calcium channel.

PDE has an inhibitory gamma subunit which inhibits the PDE activity.
The C-terminal peptide of this domain inhibits PDE activity.
This C terminal inhibitory peptide can also bind to/interact with the switch II of the alpha subunit, which allows the alpha/beta subunit of PDE to catalyse the cGMP into GMP reaction.

RGS binding to the alpha subunit of transducin is enhanced by the binding of the PDE gamma subunit binding.
This allows the increase in the GTP hydrolysis by the alpha subunit by positioning and stabilising the switch regions and catalytic groups for efficient hydrolysis.

\section*{Rhodopsin kinase and recoverin}

\subsection*{Arshavsky 2002}

This paper was to address the mechanisms proposed by different researchers about the inactivation process of  rhodopsin.
Rieke and Baylor (1998) proposed that the inactivation of rhodopsin is a multi-step process rather than a single stochastic event so the inactivation process is uniform and therefore able to produce reliable electric response.

Different studies show different amount of phosphorylation on the C-terminal end of rhodopsin to inactivate rhodopsin.
One group proposed that the inactivation process occurs in a 1 to 7 phosphorylation step mechanism by rhodopsin kinase.
This lead to the hypothesis where phosphorylation event itself accounted for the reproducibility of the response, which was mediated by the calcium feedback due to the activation of the signal (calcium feedback hypothesis).
The phosphorylation is done by rhodopsin kinase, which is mediated by recoverin (constantly regulated by calcium concentration).

Multiple phosphorylation hypothesis, on the other hand, proposes that rather than the individual phosphorylation event, the number of phosphorylation on rhodopsin provides the reproducibility of the signal.
This is based on the observation where arrestin binds to rhodopsin and inactivate it, but binds to rhodopsin with higher affinity when it is phosphorylated more.
Thus the arrestin binding provides the reproducible signal response.

Kennedy et al (2001) showed that the phosphorylation of  rhodopsin was sequential and in a highly ordered pattern, and likewise with dephosphorylation.
This supports the multiple phosphorylation hypothesis.
However, this does not mean that the calcium feedback hypothesis is incorrect -- these two mechanisms may work together to produce a reproducible signal.

Another thing mentioned was that there were two phases of dark adaptation in mouse.
The first phase occurs within minutes (2min), whereas the other occurs after 40 minutes.
These corresponded to the time frames of the phosphorylation of rhodopsin and the slow decay of the photoactivated rhodopsin and the subsequent regeneration of the cis retinal bound form of rhodopsin, as all of the decay product can activate the signal.

\subsection*{Koch et al 2003}

This study was looking at the relevance of the N-terminal myristoylation of recoverin.
At high calcium concentration, recoverin is able to bind to rhodopsin kinase and inhibit its activity.
When the rods are illuminated, the signal via rhodopsin leads to a decrease in the calcium concentration in the cell, which causes the activation of rhodopsin kinase.

\begin{description}
\item[T state]is where the recoverin is in a calcium free state and the myristoyl is buried within the hydrophobic pocket.
This state activates rhodopsin kinase.
\item[R state]is where the recoverin have calcium bound to it, which causes the myristoyl to protrude out.
\end{description}

There are two calcium binding domain in recoverin -- EF2 which binds to calcium with a low affinity, and EF3 which binds to calcium with high affinity.
This means that calcium binds to the EF3 first, and then onto EF2.
Furthermore, the myristoyl is suggedted to be an allosteric regulator that affects the sequential binding of calcium onto recoverin (EF3 first, then EF2).

Koch et al investigated what the myristoyl modification was important for.

\noindent\rule{\textwidth}{0.4pt}

With a non-myristoylated recoverin, they showed that the myristoylation modification was not necessary for the inhibition of the rhodopsin kinase, but the calcium binding to the EF2 was (E85Q mutant that could not bind to calcium was not able to inhibit rhodopsin kinase).

They showed that the calcium binding to EF3 was sufficient to cause the conformational changes from T to R state by destabilising the T state of the N-terminal domain of recoverin.
This N-terminal destabilisation is counteracted by the myristoyl group, where the myristoyl group stabilises the N term in the T state, and thus prevent inhibition of rhodopsin kinase.
When the calcium binds to the EF2 domain, this myristoyl stabilisation of the N term is relieved, and the full conformational change from T state to the R state.
The E85Q mutant which could not bind calcium in EF2 therefore was not able to adopt the full R state conformation, and therefore was not able to inhibit rhodopsin kinase.

They also showed that upon calcium binding, a hydrophobic patch appears.
When calcium binds to the EF3 domain, most of the patch appears, but there is a hydroxyl group (from tyrosine) in the middle of the patch that prevents the interaction with rhodopsin kinase.
This hydroxyl group is completely buried when calcium binds to the EF2 domain, again showing that calcium binding to the EF2 domain is essential for the inhibition of rhodopsin kinase.

\subsection*{Higgins et al 2006}

This paper was looking at how recoverin inhibited rhodopsin kinase.
Rhodopsin kinase is known to be able to phosphorylate rhodopsin as well as itself (autophosphorylation).
It has been known that the phosphorylation site of rhodopsin by itself has low affinity by rhodopsin kinase, suggesting that there must be additional interaction site to recognise the activated rhodopsin as a substrate.

\noindent\rule{\textwidth}{0.4pt}

Higgins et al showed that the first 15 N-terminal residues of rhodopsin kinase were essential for the phosphorylation of rhodopsin, and also for recoverin interaction.
However, they showed that even though these residues were essential for phosphorylation, it was not essential for rhodopsin kinase's catalytic activity (i.e. they were able to autophosphorylate itself without the residues).
This suggests that these residues are important for the recognition of the rhodopsin kinase by other proteins but not the kinase activity.

They predicted that these 15 residues formed an amphipathic helix that was able to be recognised by various proteins.
Mutations that altered the hydrophobic residues of this amphipathic helix abolished the ability of rhodopsin kinase recognition by other proteins, signifying the importance of these residues.

Higgins et al concluded that the N terminal residues are essential for the recognition of rhodopsin by other proteins, and the binding of antibody to these residues blocked the phosphorylation of rhodopsin by preventing the rhodopsin kinase interacting with rhodopsin.

\section*{Arrestin}

\subsection*{Burns et al 2006}

This study was done to establish the roles of the arrestin splice variant.
\begin{description}
\item[p48 arrestin] is the full length arrestin that is 10 times as more abundant that p44 splice variant, and is excluded from the rod outer segment that is dark adapted.
Under constant light illumination, arrestin moves to the outer segment from the inner segment.
\item[p44 arrestin] is the C terminal truncated arrestin that can turn off rhodopsin more efficiently than p48.
\end{description}
This lead to the hypothesis where the p44 arrestin is more important than p48 in deactivating rhodopsin under dark adapted conditions.

\noindent\rule{\textwidth}{0.4pt}

First they showed that under light condition, the binding of p48 to rhodopsin does not cause significant increase in the p44 concentration -- shows that the p44 variant is not due to the proteolysis of p48 variant.

Constant activation of rhodopsin can cause degeneration of the photoreceptor.
They showed that both p48 and p44 arrestin was able to prevent the photoreceptor degeneration under normal condition (where the exposure to light was alternated with darkness).
However, p48 was able to prevent the degeneration of the photoreceptor under constant light condition, whereas the p44 splice variant could not.
This suggested that p48 was able to prevent light induced retinal damage.

Previous reports on the abrupt turn off of rhodopsin in rhodopsin kinase knock out cells was thought to be due to the p44 arrestin binding to the non-phosphorylated rhodopsin.
They showed that the arrestin and rhodopsin kinase double knockout cells still had the abrupt turn off (but after a longer time).
They also showed that the presence of p48 made the response look similar to the WT cells, but the presence of p44 did not, suggesting that p48 is able to deactivate the non-phosphorylated receptor, but not p44.
They concluded that p48 was important for decreasing the time required for dark adaptation by stabilising the decay product by binding to rhodopsin, but not p44.

Arrestin mechanism:

\begin{enumerate}
\item Arrestin has two domains, activation recognition domain (low affinity to activated rhodopsin), and phosphorylation recognition domain (low affinity to phosphorylated rhodopsin).
\item Encounter with the activated and phosphorylated rhodopsin disrupts the interaction that stabilise the basal state of arrestin (arrestin activation).
\item This leads to a conformation change and exposes the secondary binding site of arrestin that has a high affinity for the activated and phophorylated rhodopsin.
\end{enumerate}

\end{document}
