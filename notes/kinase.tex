\documentclass[a4paper,12pt]{article}

\usepackage[utf8]{inputenc}
\usepackage[T1]{fontenc}
\usepackage{natbib}
\usepackage{geometry}
\usepackage{fixltx2e}
\usepackage{enumitem}
\usepackage{amsmath}

\geometry{
    a4paper,
    left=2.5cm,
    right=2.5cm,
    top=2.5cm,
    bottom=2.5cm
}

\begin{document}

\begin{center}
\LARGE{Kinase Notes}
\end{center}

\section* {RAF Transactivation}

\subsection*{\normalsize{Hatzivassiliou et al 2010}}

One of the first evidence that V600E mutant inhibitors activated the WT BRAF/KRAS paradoxically.

\noindent\rule{\textwidth}{0.4pt}

CRAF knockout was able to prevent the pathway activation, but BRAF knockout could not.
This suggested that CRAF was essential for the pathway activation, but not BRAF.
They mutated the CRAF so that the inhibitor was unable to bind to it (T421N mutant), and showed that this was also able to prevent the activation of the pathway, signifying the fact that the inhibitor binding to the active site was essential for effective downstream signalling.

They further investigated the inhibitor AZ-628 and compared it with the other inhibitors such as PLX 4720 and GDC 6879.
\begin{description}
\item[AZ-628] binds to the kinase in an inactive conformation of the active site, and has a very slow/irreversible off rate.
\item[GDC6879 and PLX4720] both have a very rapid off rate, and therefore unable to occupy the ATP binding site for long.
This means that these inhibitors cannot inhibit CRAF kinase activity effectively.
\end{description}

They showed that AZ628 was able to inhibit the pathway acitvation effectively, but the others could not.
They hypothesised that this was due to the ability of AZ628 to occupy the active site of CRAF for a long period of time, and thus able to inhibit the kinase activity of CRAF.

They also showed that the inhibitor binding to the active site affected the dimerisation of BRAF/CRAF, and that the dimerisation was essential for the pathway activation via the recruitment to the membrane in a RAS-dependent manner.
They also showed that the mutants that could not be dimerised were not able to activate the pathway, showing further evidence of dimer dependent pathway activation.

\subsection*{\normalsize{Poulikakos et al 2010}}

Another evidence suggesting that the BRAF inhibitor activates WT BRAF/CRAF.
The inhibitor is effective against the BRAF V600E mutants, but not the WT BRAF.

\noindent\rule{\textwidth}{0.4pt}

First, they showed that the BRAF KO, but not CRAF KO activated the pathway in a dose dependent manner when inhibitor was added (i.e. activation occurred even without BRAF in the cell).
This result shows how CRAF is essential for the pathway activation, but not BRAF.

Secondly, they created a mutant (T421M) that was not able to bind to the PLX inhibitors, but was able to bind to sorafenib.
In this mutant, they showed that the PLXs were not able to activate the pathway, but sorafenib was able to acitvate it.
This meant that direct binding of the inhibitor to the ATP binding site of the kinase was essential for the activation of the pathway.

Next, they tested whether the kinase activity was required for the pathway acitvation.
They generated a JAB sensitive mutant which is also a kinase dead mutant, and co-transfected this mutant with various cells (e.g. WT, kinase dead, etc.).
To these cells, they added JAB inhibitor which is going to inhibit the JAB sensitive but not the WT CRAF.
The results showed that when it was co transfected with the WT kinase, the pathway was activated, but was not activated when together with either a dimer mutant (R401H) or the kinase dead (D486N) mutant.
This suggested that the pathway was activated by transactivating the kinase partner, rather than the kinase that gets inhibited by the inhibitor, and that dimerisation is essential for the transactivation.

Lastly, since the dimerisation of the RAF kinases are RAS dependent, they tested whether over activation of RAS would cause the re-activation of the pathway in V600E cells.
As predicted, the pathway was activated even with the inhibitor added to the cells, and showed that the drug resistance can be accomplished by overactivating the RAS.

\subsection*{\normalsize{Hu et al 2013}}

Previous studies have shown that it was the transactivation that caused the pathway activation, and this activation required dimerisation, kinase activity in the transactivating partner, and a competitive inhibitor binding.
RAF is known to have two phosphorylation sites that must be phosphorylated in order to activate itself.
One of these sites is in/at the activation loop, and the other phosphorylation site is at the N terminal acidic (NtA) motif.
This paper showed that the N term acidic residues are essential for this transactivation.

\noindent\rule{\textwidth}{0.4pt}

They first showed that the phosphorylation of the NtA motif was required for the dimer dependent transactivation.
The serine residue in the NtA (SSDD residues) of the BRAF kinase are constitutively phosphorylated, and mutation in these residues lead to decrease pathway activation.
In contrast, the corresponding residues in CRAF (SSYY) are not constitutively phosphorylated, and this phosphorylation is required for the activation.
Mutation that mimicked the phosphorylation state (changing it to acidic residues) stimulated the pathway, demonstrating that the phosphorylation of these residues are essential for pathway activation.
Furthermore, when the RAFs were mutated so that they were unable to form dimers, this transactivation process was abolished, even with the constitutively transactivating mutants.

Secondly, they tested whether this NtA phosphorylation was required by both the activator and the receiver (where activator is kinase dead, but receiver is not).
The results showed that the receiver with mutated NtA was still able to be transactivated by an activator, suggesting that the phosphorylation of NtA is necessary on the activator, but not on the receiver.

Next, they tested the relevance of the conserved tryptophan after the NtA motif.
Substitution mutation of this tryptophan impaired the ability for the BRAF/CRAF to be an activator or a receiver.
They showed that this impairment was not due to the lack of dimerisation, and concluded that this tryptophan was important to place the NtA correctly at the dimer interface.

They also showed that the dimerisation of the activator and the receiver resulted in the phosphorylation of the activation loop.
This result suggested that, since the activator is kinase dead, the activation loop phosphorylation is from the auto phosphorylation by the receiver.
This meant that, in order to autophosphorylate itself, the kinase must be in the active conformation before this phosphorylation, and that the transactivation process changes the conformation into the active conformation.

They hypothesised that the NtA motif placement caused the hydrophobic regulatory spine (R spine) to change into the active DFG ``in'' positionm rather than the DFG ``out'' inactive conformation.
As predicted, the mutation that constitutively changed the R loop into the active DFG in conformation activated itself without dimer formation and/or NtA phosphorylation.

Finally, they showed that the NtA phosphorylation was MEK dependent.
This meant that BRAF activates B/CRAF via transactivation, which phosphorylates MEK.
Phosphorylated MEK then in turn phosphorylates CRAF at the NtA motif, which makes it an activator.

\section*{Tumour Resistance}

\subsection*{\normalsize{Thakur et al 2013}}

They used V600E primary tumour from humans and transplanted/xenografted it into mice, and the applied vemurafenib until resistance developed.
This resistant tumour was transplanted into a different mouse and used to test the effect of drugs and compared it with the WT tumour that is sensitive to the drug.

\noindent\rule{\textwidth}{0.4pt}

First, they showed the tumour responded to the drug (vemurafenib) in a dose dependent manner, but drug resistant tumour did not.
They also showed that the resistant tumour had constantly higher pERK level compared to the primary tumour, and still responded to the drug, but less effective.
This suggested that the V600E tumour still required this mutation (i.e. BRAF V600E) for its growth, and that the resistance was not due to mutations in RAS or MEK (since no secondary mutations were found).
They showed that this resistance was not due to the splice variant of BRAF, but due to the overexpression of BRAF by elevated mRNA and/or copy number variation of the BRAF gene.

Next, to test whether elevated level of BRAF was the cause, they made a cell culture with the primary tumour cells.
In doing so, they showed that either too much or too little vemurafenib caused a decrease in the tumour growth, showing that the tumour was reliant on the drug.
After they were able to grow the tumour cells, they tested whether the BRAF knockdown had an effect on growth.
They found that complete knockdown of BRAF resulted in the suppression of proliferation but partial suppression of BRAF (to the level of wild type) caused re-sensitization to the drug.
This result showed that the tumour resistance was due to increased levels of BRAF and is also dependent on the signal made from V600E mutant, rather than another mechanism (e.g. RAS/MEK mutations).

They showed that the drug withdrawal, although it increased the pERK level, decreased the growth of the tumour.
This decreased growth was reverted/overcome after a while as it got used to the drug concentration.
This showed how the tumour showed fitness deficit in the absence of the drug (i.e. reliant on the drug).
The withdrawal symptoms were applicable to other BRAF mutant cell lines, proving that these tumours were reliant on the drug.

Finally, they showed that intermittent drug application increased the period until drug resistance emerged.
This signified how drug treatment plans could be altered to prevent rapid drug resistance.

\noindent\rule{\textwidth}{0.4pt}

\begin{enumerate}
\item Vemurafenib resistance of V600E tumours rely on the re-activation of the ERK pathway via BRAF, even though that is the component that is being inhibited.
This was achieved by increasing the amount of BRAF in the cell and/or the up/downstream components of the pathway.
\item Fitness deficit from drug withdrawal was due to the elevated ERK activation.
This lead to the decrease of cell division/cycle and decreased apoptosis.
\item Emergence of drug resistance can be controlled by intermittent drug application/administration.
\end{enumerate}

\subsection*{\normalsize{Lito et al 2012}}

This paper focussed on the V600E mutant cells to figure out the mechanism in which the resistance to drugs arises.

\noindent\rule{\textwidth}{0.4pt}

Firstly, they found that in V600E cells the RAS level was very low compared to the normal cells.
This meant that V600E acts independently of RAS, due to the overactive negative feedback mechanism by ERK (via Spry2 and DUSP6 inhibits the signal pathway from RTKs to RAS).

Secondly, they showed that the activation of RAS was followed by ERK activation, even though vemurafenib was added.
This was due to the relief of the ERK mediated feedback inhibition of RAS (by constitutively active BRAF), which leads to the increased activity of RAS.
The activation of RAS causes the BRAF to dimerise with CRAF and transactivate it, which can stimulate the pathway.

From this result, they hypothesised that if the re-activation was caused by the transactivated CRAF, then a MEK inhibitor should be able to inhibit this re-activated pathway.
In fact, MEK inhibitor was able to inhibit the pathway.
They also showed that the combination of the RAF and MEK inhibitor was able to reduce the pathway activation significantly, compared to either drug alone.

ERK over activation by V600E caused the de-sensitization of the receptor's ability to recognise the external signal, and so hypothesised that the relief of ERK mediated feedback inhibition would re-sensitize the receptor to signals.
They proved this by showing that ERK was phosphorylated after 2 hours of vemurafenib treatment with EGF and NRG.
This showed that the rebound of the pathway was partly due to the presence of the ligands that stimulated the pathway and therefore re-activation of the pathway via RAS.

\section*{Unravelling Complex Pathway}

\subsection*{\normalsize{Toettcher et al 2013}}

This study used optogentically activatible SOS that specifically activated RAS to determine what specific downstream component is activated by RAS.
There are 2 models of signalling:
\begin{description}
\item[Combinatorial] is where different stimuli from different signal(s) in combination leads to a downstream effect, which is specific to the combination of the stimuli.
\item[Dynamic] is where the period of time in which the component is stimulated by a particular signal may be detected and cause a downstream effect.
Since different signals will have different different timescale, it will be detected slightly differently.
\end{description}
The output was measured as the amount of BFP-ERK in the nucleus after stimulus, as well as the recruitment of YFP-SOS to the membrane.

\noindent\rule{\textwidth}{0.4pt}

They showed that optogentics approach was able to mimic (and comparable to) the signal stimuli such as PDGF and NGF.
There were high variability within the population, but the signal/response from optogentics was consistent in a single cell (likely due to the variable expression between the cells).
With this approach, they showed that the RAS specifically activated MAPK/ERK pathway, but not the PI3K pathway, suggesting that PI3K pathway activation requires more than RAS by itself.

They also showed that the RAS signalling was a high bandwidth and low pass filter pathway, meaning that RAS/ERK pathway is able to respond to the stimulus over a broad range of time period (from 4min to 2hrs) and filters the transient minute scale inputs (hence low pass filter).
This means that only long lasting stimuli are transmitted efficiently through RAS/ERK pathway and short stochastic stimuli are ignored.

From the high-throughput protein array screen, they managed to identify STAT3 being activated as a result of prolonged (more than 2hrs stimulus) RAS, and therefore ERK activation.
Furthermore, they showed that STAT3 activation resulted from the release of a cytokine from the cell that has active ERK pathway and affects the nearby cells in paracrine fashion.
However, they showed that the cytokine released did not activate STAT3 in the cells that produced it (i.e. did not act in an autocrine fashion).
This was possible in a population of cells, as some cells had lower expression levels than other cells, allowing the STAT3 sensitivity even if it was activating ERK pathway (by a small amount).

\subsection*{\normalsize{Dar et al 2012}}

First of all, they noted that the most successful drugs for cancers are the ones that target more than one part of the pathway.
However, these discoveries were due to luck more than the design of the drug.
So, they proposed a whole-body/phenotypic screening method using Drosophila that allowed them to identify these poly pharmacological drugs that target multiple pathways.

\noindent\rule{\textwidth}{0.4pt}

They made mutant flies such that the flies were expressing the oncogenic receptor tyrosine kinase RET.
They calibrated these flies so that 50\% of the flies were able to develop into pupae, but none were able to grow into adult flies.
Thus, to get adult flies, this RET receptor had to be inhibited by a drug.
They discovered AD57 which could rescue the flies, and also found AD58 to have a toxic effect (i.e. made it worst).

They focussed on these drugs to identify why one was such a good drug, whereas the other drug was not.
They showed that different drugs inhibited various pathway by a different degree.
This provided some targets/anti-targets to test why the inhibition of some pathway led to toxicity (AD58), but not in the other (AD57).

They found that AD58 toxicity was due to the over activation of the ERK pathway by relieving the inhibition of RAF by dTor.
Since AD58 inhibits dTor, RAF is not inhibited by it, and therefore leads to the activation of ERK.
Since AD57 also inhibited dTor, they tested whether making a drug that did not inhibit dTor would make AD57 a better inhibitor.
AD80 was designed as a result, and it worked the best out of all the drugs identified, and it inhibited all of the pathways without significantly activating other pathways (e.g. by dTor).

\section*{Insight into the complex mechanism of the ERK pathway}

\subsection*{\normalsize{Haling et al 2014}}

This paper showed the additional mechanism of how the RAF/MEK/ERK pathway is activated/regulated.

\noindent\rule{\textwidth}{0.4pt}

They showed that in WT  or RAS mutant cells, the RAF/MEK complexes are enriched, suggesting that RAF is mainly in a complex with MEK in the cytosol and is quiescent and awaiting activation.
They also showed that the BRAF dimerisation was required for the interaction with RAS-GTP, but not with MEK, meaning that the complex is quiescent until activation by RAF dimerisation through RAS-GTP.

They noted that in the RAF/MEK complex, MEK adopts an inactive conformation but RAF is in an active conformation where the MEK activation loop is able to access the active site of the RAF, ready for phosphorylation.
This phosphorylation event may lead to the dissociation of the complex due to the steric and electrostatic interactions.

Mutations that prevented the RAF/MEK complex interaction caused a decrease in the ERK phosphorylation.
This was due to the inability of MEK to interact with RAF, and therefore no phosphorylation event, eventually leading to no activation of ERK.

The P-loop mutation in BRAF resulted in the weakening of the RAF/MEK complex formation, and therefore increased the BRAF/CRAF dimerisation (instead with MEK), and as a consequence, lead to the activation of the pathway via CRAF transactivation.

Since RAF is in an active conformation when bound to MEK, they tested whether the conformation affected the complex formation.
Inhibitors such as vemurafenib that keeps RAF in an inactive conformation increased the ERK activation (by promoting the dissociation from MEK and transactivation of CRAF).
However, mutations that kept the active conformation was able to inhibit the activation (as the complex was still held in place).

\noindent\rule{\textwidth}{0.4pt}
\begin{enumerate}
\item Normally, BRAF is in complex with MEK in the cytosol and is primed but sequestered.
\item Extracellular signal causes dimerisation of BRAF with BRAF/CRAF and allows catalytic activity and phosphorylates MEK (which is already attached to RAF).
\item Phosphorylated MEK then dissociates from the RAF and phosphorylates ERK, which activates the pathway.
\end{enumerate}


\end{document}
