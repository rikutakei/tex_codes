\chapter{\Gls{gwas}}
\label{cha:gwas}

This chapter will be written using the information from the \citet{Hirschhorn2005} and \citet{McCarthy2008} reviews as a starting point/guide.

\section{Mendelian diseases}
\label{sec:mendelian_diseases}

Mendelian diseases, or monogenic diseases, are diseases caused by (often) rare gene variants that directly affects the phenotype of the patient.
These variants are usually penetrant, meaning that the carrier of the disease-causing allele will always show the disease phenotype.
Therefore, by tracking a ``marker'' at or near the disease-causing allele you are able to determine the disease status of an individual for a Mendelian disease (since these markers will co-segregate with the causal allele).

Another thing to note about Mendelian diseases is that the causal variants are usually rare.
This is because negative selection would have acted on the resulting phenotype (i.e.\ disease) which causes early-onset mortality and morbidity.
As a result, the frequencies of these variants observed in the population are dramatically reduced.
This also means that the variants that cause Mendelian diseases emerged reltively recently, in terms of evolutionary time scale, otherwise the variants would not have survived the negative selection over many generations.

Linkage analysis has been used to detect and identify the causal genes of Mendelian diseases.
In linkage studies, many markers are spread out along the genome and the segregation of these markers are tracked through generations.
Markers that are close to the disease-causing gene will segregate with the disease in the family since Mendelian diseases are highly penetrant (meaning that the individuals who have the disease allele will have the disease).
Therefore, markers that are close (within 10$\sim$20 cM) to  the disease-causing alleles will co-segregate with disease status, and able to be tracked and studied.

\section{Common diseases and quantitative traits}
\label{sec:common_diseases_and_quantitative_traits}

Common diseases and quantitative traits are not as simple as Mendelian diseases.
For common diseases and quantitative traits, it is hypothesised that the traits will be determined by variants that occur frequently in the population, but each only having a small impact on the resulting phenotype.
This is known as the ``common disease, common  variant'' hypothesis.
Since each individual variants contribute small amount to the phenotype (whether it is related to the disease or a trait), these variants would have gone through less strict negative selection process, unlike the rare Mendelian variants.
Therefore, most variants are likely to be ``ancient'' and have been present in the population for a long time.

Linkage analysis has been very successful in identifying genes that cause rare Mendelian diseases, but it was not so successful in identifying genes for common disease or quantitative trait.
There have been a few genes, gene regions and gene variants that have been identified that related to some diseases, but these variants usually explain only a small fraction of the \gls{heritability} of the disease.
For example, the identified variant may explain an excess risk of two-fold, where in reality there is a total excess risk of 30-fold, which shows that there are many other causal genes that are yet to be identified.

There are various reasons for this:
\begin{itemize}[noitemsep]
	\item low \gls{heritability} of most complex traits
	\item the inability of the standard set of microsatellite markers to extract complete information about inheritance
	\item the imprecise definition of phenotypes
	\item inadequately powered study designs
\end{itemize}
Linkage analysis has a poor power of detecting common alleles that have low penetrance and have modest effect on disease.
Therefore, an alternative strategy is required to study common diseases and traits.

\section{Candidate gene resequencing studies}
\label{sec:candidate_gene_resequencing_studies}

Candidate gene resequencing studies are where you select genes or gene regions of interest that may be important to the disease.
These genes are chosen based on their location in a region of linkage, or some other evidence that may affect the disease risk.
Once these genes or gene regions are resequenced, the variants or set of variants that are different between the cases and controls are searched.

This approach is expensive and labour-intensive, and therefore it is largely limited to the coding regions of one or a few candidate genes.

\section{Association studies}
\label{sec:association_studies}

In brief, association studies compare the frequency of common alleles or genotypes of  a particular variant between disease cases and controls.

\subsection{Candidate gene association studies}
\label{sub:candidate_gene_association_studies}

In this approach, a set of genes will be chosen based on the biological hypotheses, or the location of the genes that are within previously determined region of linkage.
These studies are made more efficient by the use of \gls{ld}, as less markers will be required to assess multiple variants or genomic regions.
Candidate gene association studies have identified many of the genes that are known to contribute to susceptability to common disease.
With that said, these studies can only identify  a fraction of genetic risk factors, even for diseases with well-known pathophysiology.
Furthermore, it is clear that if the physiological mechanism of a disease is unknown, this approach would not be able to fully explain the genetic basis of the disease.

\subsection{\Acrfull{gwas}}
\label{sub:gwas}

In contrast to the candidate gene association studies, \gls{gwas} looks at most of the genome for the causal variants.
Unlike candidate gene association studies, \gls{gwas} does not make any assumptions about the genomic location of the causal variants.
This means that you are able to carry out the association studies without having to know the causal genes of the disease.
``The genome-wide association approach therefore represents an unbiased yet fairly comprehensive option that can be attempted even in the absence of convincing evidence regarding the function or location of the causal genes'' \citep{Hirschhorn2005}.

\section{\glsdesc{ld}}
\label{sec:ld}

Consider two alleles or gene loci, A and B, that are on the same chromosome.
If these alleles are ``dependent'', then A and B will always segregate together.
Mathematically, the probability of both A and B segregating together is $P(AB)$.
Now, if A and B are ``independent'', then A being passed on to the offspring has no influence on B being passed on, and vice versa.
Mathematically, the probability of both A and B segregating in this case is given by $P(A)P(B)$.

\Gls{ld}, by definition, is the correlation between nearby variants such that the alleles at neighbouring markers on the same chromosome are associated within a population more often than they were unlinked.
In other words, {\bfseries \gls{ld} is a measure of how likely two alleles or loci are passed on to the offspring together}.
Mathematically, this is represented as:
\begin{equation}
	\label{eq:ld}
D = P(AB) - P(A)P(B)
\end{equation}
where $D$ is the coefficient of \gls{ld}.
When $D = 0$, A and B are said to be in equilibrium and will not segregate together.
On the other hand, when $D = 1$, A and B are said to be in perfect \gls{ld}, where A and B will always segregate together.

By measuring the \gls{ld} of two alleles, or in our case two markers, you are able to determine whether those alleles will be passed on together.
This means that only one marker is required to detect or track both alleles, if they are in strong \gls{ld}.
Therefore, you can choose several markers that represent several alleles that are in high \gls{ld} with the marker.
Furthermore, if that marker is associated with a disease, it is likely that at least one of the markers that are in high \gls{ld} with that marker is also associated with the disease.
This allows us to use less markers without losing any power to detect variants that are associated with the disease.








































