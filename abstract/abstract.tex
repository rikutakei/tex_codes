\documentclass[a4paper,12pt]{report}

\usepackage[
    style=authoryear,
    firstinits=true,
    backend=biber,
    uniquename=init,
    isbn=false,
    doi=false,
    url=false,
    mincitenames=1,
    maxcitenames=2,
    minbibnames=7,
    maxbibnames=7,
    natbib=true]{biblatex}
\renewcommand*{\bibfont}{\small}
\renewbibmacro{in:}{}
\renewbibmacro{volume+number+eid}{\printfield{volume}\addcolon}
\renewbibmacro*{name:andothers}{% Based on name:andothers from biblatex.def
    \ifboolexpr{
        test {\ifnumequal{\value{listcount}}{\value{liststop}}}
        and
        test \ifmorenames
    }
        {\ifnumgreater{\value{liststop}}{1}
           {\finalandcomma}
          {}%
            \andothersdelim\bibstring[\emph]{andothers}}
            {}}
\DeclareNameAlias{sortname}{last-first}
\DeclareFieldFormat[article]{title}{#1}
\DeclareFieldFormat[article]{pages}{#1}
\DeclareFieldFormat*{volume}{\mkbibbold{#1}}

\bibliography{../../../References/BibTeX/MSc}

\begin{document}

\begin{center}
\large{\textbf{Body Mass Index and Pathway Dysregulation in Cancer}}\\

\vspace{0.3cm}

Riku Takei (MSc)\\
Supervisor: Assoc. Prof. Mik Black
\end{center}

\section*{Introduction}

It is commonly understood now that cancers are caused by dysregulation of various pathways (known as the ``Hallmarks of Cancer'') that allow the cells to proliferate, survive and migrate \citep{Hanahan2011}.
There have been many reports on the association between obesity and worse cancer prognosis, suggesting that there may be mechanisms that underlie the tumorigenesis and/or cancer progression specifically in obese patients.
\citet{Creighton2012} identified 799 genes that were able to distinguish the obese patients from the non-obese patients in a sample of breast cancer patients.
Furthermore, they showed that small molecules were able to induce or repress these genes, signifying the fact that these obesity specific genes could be used as an indicator for alternative or combinatorial chemotherapy and assist in better clinical decisions.

This study has shown that there were obesity specific genetic signatures in breast cancer, but can these genetic signatures be applied to other cancer types?
A couple of studies have shown that there are genetic signatures common in multiple cancer types, suggesting a possibility of such genetic signatures that can be applied to multiple cancer types, which are also specific to obesity \citep{Alexandrov2013,Lawrence2014}.
However, these studies have not shown any evidence of obesity specific genetic signatures that are common in multiple cancer types.

This research aims to identify genetic signatures that are specific to obesity across multiple cancer types, and to investigate whether there are any pathways being dysregulated in cancers based on these genetic signatures.
Better understanding of the pathways being dysregulated in cancers in obese patients may lead to better clinical decisions and prognosis, and contribute towards personalised medication in the future.
%Furthermore, they were able to use small molecules to induce or repress these obesity specific genes, signifying the fact that these genes may be targeted for treatment in obese patients.

\section*{Methods}


\section*{Results and Discussion}


\printbibliography


\end{document}
