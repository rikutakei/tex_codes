\documentclass[a4paper, 12pt]{article}

\usepackage[utf8]{inputenc}
\usepackage{geometry}

\geometry{
    a4paper,
    left=3cm,
    right=3cm,
    top=3cm,
    bottom=3cm
}

\linespread{1.3}

\begin{document}

\begin{center}
\large{Olfactory Receptors, Copy Number Variations and Evolution}
\end{center}

\section*{\normalsize{Abstract}}

Human olfactory receptors (ORs) make up the largest family of genes in the genome, yet there has been little work done on characterising these receptors.
Recent advances in sequencing technology have allowed many researchers to investigate various aspects of the genome from single nucleotide polymorphisms (SNPs) to large chromosomal rearrangements.
Copy number variations (CNVs) is one such genomic rearrangement whereby some genes get duplicated, inserted, deleted and/or inverted.
Studies have shown that OR genes are highly copy number variable within and between species, but no study has associated this with phenotypic characteristics.

\section*{Introduction}

\section*{Research Proposal}

\section*{Conclusion}

\end{document}
