\documentclass[a4paper, 12pt]{article}

\usepackage[utf8]{inputenc}
\usepackage{geometry}

\geometry{
    a4paper,
    left=3cm,
    right=3cm,
    top=3cm,
    bottom=3cm
}

\linespread{1.3}

\begin{document}

\begin{center}
\large{Olfactory Receptors, Copy Number Variations and Evolution}
\end{center}

\section*{\normalsize{Abstract}}

Human olfactory receptors (ORs) make up the largest family of genes in the genome, yet there has been little work done on characterising these receptors.
Studies have shown that OR genes are highly copy number variable within and between species, but no study has associated this with phenotypic characteristics.
Sensory properties have been linked to satiety and eating behaviours, but has not been described in terms of the individual genotype.
Next generation sequencing will be used to investigate the effect of OR copy number variation (CNV) on the individual phenotype, and to see whether this affects the sensory-specific satiety and/or food consumption.

\section*{Introduction}

\subsection*{Intro to olfactory receptors}

Olfaction, or the sense of smell, is mediated by a large group of sensory receptors known as the olfactory receptors (ORs).
ORs all belong to the G-protein coupled receptor (GPCR) superfamily like other sensory receptors, such as for vision and taste.
ORs are known to be the largest gene family in the human genome and form clusters on many chromosomes, and consists of about 900 genes of which at least half of these genes are pseudogenes and have no functional activity (ref Glusman et al 2001; Niimura and Nei 2005).

Due to the fact that there are so many OR genes in the human genome, and that these ORs work in a combinatorial manner, it has been difficult for the researchers to decipher the complete code of olfaction (ref Hasin-Brumshtein et al, 2009; Nei et al 2008).
In addition to this, ORs are highly copy number variable within and between species, adding greater complexity to the olfactory process and remains to be a challenge for the researchers to make a connection between OR CNV with the phenotype (ref Nei et al, 2007).

OR genes also have an important implications in the evolution of different species.
There are two major classes of OR genes: class I and class II genes, or ``fish-like'' and ``mammalian-like'' genes respectively, as the different classes of genes belong to their respective species (ref original paper or Nei and Niimura 2005).
The study by Niimura and Nei (2005) showed that the diversity of the OR gene family in fishes was greater than that of the mammalian OR gene family, even though the number of OR genes in a single family were less in fishes.
Furthermore, they showed that amphibians have retained the diversity of the gene family as the fishes, while also having the large repertoire of OR genes in one family as in the mammals, signifying the fact that the CNV was likely due to the adaptation process from being aquatic to terrestial (Nei and niimura, 2005).
As both pseudogenization and CNV are evidence of genomic drift (ref Nei et al, 2007).



\subsection*{Intro to copy number variations}

Recent advances in next generation sequencing technology have allowed many researchers to investigate various aspects of the genome, ranging from something as small as single nucleotide polymorphisms (SNPs) to large chromosomal rearrangements (ref NGS papers).
Copy number variations (CNVs) is one such genomic rearrangement whereby some genes get duplicated, inserted, deleted and/or inverted.

\subsection*{Intro to olfaction and sensory specific satiety}



\subsection*{Significance of the work}

\section*{Research Proposal}

\subsection*{What am I going to do???}

\section*{Conclusion}



\end{document}
