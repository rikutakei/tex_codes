\documentclass[a4paper, 12pt]{article}

\usepackage[utf8]{inputenc}
\usepackage{natbib}
\usepackage{geometry}
\usepackage{fixltx2e}

\geometry{
    a4paper,
    left=3cm,
    right=3cm,
    top=3cm,
    bottom=3cm
}

\linespread{1.3}

\begin{document}

\begin{center}
    \Large{\textbf{Copy Number Variation of Olfactory Receptors and its Significance on Sensory Specific Satiety and Evolution}}
\end{center}

\section*{Abstract}

Human olfactory receptors (ORs) make up the largest family of genes in the genome, yet there has been little work done on characterising these receptors.
Studies have shown that OR genes are highly copy number variable within and between species, but no study has associated this with phenotypic characteristics.
Sensory properties have been linked to satiety and eating behaviours, but has not been described in terms of the individual genotype.
Next generation sequencing will be used to investigate the effect of OR copy number variation (CNV) on the individual phenotype, and to see whether this affects the sensory-specific satiety and/or food consumption.

\section*{Background}

\subsection*{Olfactory Receptors}

Olfaction, or the sense of smell, is mediated by a large group of sensory receptors known as the olfactory receptors (ORs).
ORs all belong to the G-protein coupled receptor (GPCR) superfamily like other sensory receptors, such as for vision and taste.
ORs are known to be the largest gene family in the human genome and form clusters on many chromosomes, consisting of about 900 genes of which at least half of these genes are pseudogenes and have no functional activity \citep{Glusman2001,Niimura2005}

Due to the fact that there are so many OR genes in the human genome, and that these ORs work in a combinatorial manner, it has been difficult for the researchers to decipher the complete code of olfaction \citep{Hasin-Brumshtein2009, Nei2008}.
In addition to this, ORs are highly copy number variable within and between species, adding greater complexity to the olfactory process which remains to be a challenge for the researchers to make a connection between OR CNV with the phenotype \citep{Nozawa2007}.

OR genes also have an important implications in the evolution of different species.
Two major classes of OR genes have been identified: class I and class II genes, or ``fish-like'' and ``mammalian-like'' genes respectively, as the different classes of genes were found mainly in their respective species \citep{Glusman2000}.
The study by \citet{Niimura2005} showed that the diversity of the OR gene families in fishes was greater than that of the mammalian OR gene families, even though the number of OR genes in a single family were less in fishes.
Furthermore, they showed that amphibians have retained the diversity of the gene families as the fishes, while also having the large repertoire of OR genes in one family as in the mammals, signifying the fact that the CNV was likely due to the adaptation process from being aquatic to terrestial \citep{Niimura2005}.

Another study by \citet{Liman2003} showed that pseudogenization and CNVs of the ORs coincided with the evolution of trichromatic vision in primates.
Although this was an evidence of how trichromacy in primates became more important than the ORs for their survival, better understanding of the ligands of the ORs is crucial to support this theory.

\subsection*{Sensory Specific Satiety and Eating Behaviour}

It has been noted for a long time how eating behaviour changed depending on the sensory properties of food, such as the smell and the taste of the food \citep{Rolls1981, Rolls1982}.
The effect of the sensory properties of food on the ability to change the pleasantness of eating that food was termed ``sensory specific satiety'' \citep{Rolls1981}.
In humans, the flavour of the food is perceived with both the taste and the smell of the food, and this contributes to the sensory specific satiety \citep{Rolls2005}.
This means that, although the taste of the food is important, the smell of the food is as important as the taste for the perception of flavour, yet the ligands and mechanism of olfactory process is poorly understood.

Obesity is a significant health problem in many Western countries.
Although the prevalence of obesity in United States has not changed significantly since 2003-2004 to 2011-2012, more than a third of the population are obese and remains a significant health problem \citep{Ogden2014}.
Better understanding of the olfactory process and their ligands may help understand the eating behaviour of obese people, and promote healthier eating and improve their diet.

\subsection*{Next Generation Sequencing and Copy Number Variations}

Recent advances in next generation sequencing (NGS) technology have allowed many researchers to investigate various aspects of the genome, ranging from something as small as single nucleotide polymorphisms (SNPs) to large chromosomal rearrangements \citep{Metzker2010}.
Copy number variation (CNV) is one such genomic rearrangement whereby some genes get duplicated, inserted, deleted and/or inverted.
For an overview of copy number variations, please read the review article by \citet{Girirajan2011}.

Traditionally, microarrays have been used to identify variations in the human genome before NGS was made available for lower cost.
This involved hybridisation of fluorescently labelled DNA sequence from a sample of interest to the complimentary DNA sequence probe on the microarray, and the fluorescence was detected.
Identification of the variant was inferred from the difference in the intensities of the fluorescence between the control and the test samples.
Although microarray technology was good for identifying SNPs, this method became problematic when studying highly copy number variable regions, as even the highest density microarray was not able to accurately capture the copy number variations in the genome \citep{Girirajan2011}.

Due to advances in NGS technology and its lower cost, it is now easier and more feasible to investigate complex genomic variations such as CNVs using these technologies.
Since the whole genome of the samples would be sequenced and aligned, even the highly variable regions of the genome may be annotated and analysed using NGS technology.
For example, paired-end sequences can be used to align the genome to a reference and if the sequences aligned abnormally to a region of the genome, this can inform us about the structural rearrangements that may have occurred in that region \citep{Girirajan2011}.
Even though NGS technology provides greater accuracy and identification of many variants, alignment of CNVs can still be problematic, especially if the region is highly copy number variable.
With that said though, single molecule sequencing technology may help define these complex genomic areas, as it would provide a definitive reference sequence for the alignment of CNVs \citep{Rank2009}.

In addition, there have been many studies conducted over the last few years that focussed on genomic variants using NGS.
The 1000 Genomes Project was one such study, where the whole genome sequences of more than 1000 people around the world have been sequenced to understand the variation across different population groups \citep{Durbin2010}.
Furthermore, these sequence data have been made publicly available for open use, which would be of great use to this project when comparing the extent of the OR CNVs between different population groups.

\subsection*{Significance of the Work}

Clearly, there is still a significant amount of work to be done on the ORs.
Due to the high number of OR genes (and therefore correspondingly large number of ORs), it has been difficult to identify the exact ligand-receptor pairs.
Furthermore, these ORs also work combinatorially and makes it even more difficult to identify these ligand-receptor pairs.

Better understanding of the olfactory process and the ligands of ORs will allow us to further understand the mechanism of sensory specific satiety, and promote healthier diet especially in Western country where obesity is highly prevalent.
Identification of these ligands and its connection to the CNV of the ORs will also allow us to clarify the significance of these receptors in evolution of mammalian species.

\section*{Research Proposal}

The main aim of this project is to identify ORs that are consistently copy number variable between normal samples and the samples with olfactory disorders, and to determine the physiological significance of the identified ORs in terms of sensory specific satiety.
To do this, samples must be recruited for this study.
Since the evaluation of olfaction can be difficult to assess, assistance by either a medical practitioner or a psychologist would be essential.
The samples should be matched by gender, age and ethnicity so that the CNVs are due to the perception of smell of the samples.

Once the samples are recruited, one of the control samples should be sequenced with single molecule real time (SMRT) sequencing and this sequence  would be used as the reference genome for the whole study.
Since ORs are highly copy number variable, it is important to have a good reference genome from a single sample, rather than the reference genome made from multiple different samples.
This is because we need a definitive CNV reference to identify whether the test samples have more or less copy numbers.
SMRT sequencing was chosen to make up the reference genome, as other sequencing techniques may still have some problems in aligning the CNVs and/or getting an exact value for the CNVs.

All of the other samples should be sequenced using either the paired-end sequencing method or SMRT sequencing, using the reference genome to align the sequences.
Although SMRT sequencing would give greater accuracy about the copy number, paired-end sequencing may be sufficient for CNV identification and may be cost efficient with less technical expertise requirement.

Identified ORs can then  be tested for multiple olfactory compounds, as well as their effects on sensory specific satiety.
As soon as the function of the ORs are uncovered, the data from The 1000 Genomes Project can be used to investigate whether these ORs are significant in other population groups, and use this to provide some evidence for evolutionary process.

\section*{Conclusion}

Although there have been many studies that investigated the effect of sensory properties of food on the pleasantness of food, there has been no link to the genotypic characteristics.
The use of NGS technology to sequence the whole genome of many samples will allow us to study the genomic variants in details.
The results from this study can be used to identify the ORs that are essential for the sense of smell, and also provide insight into how the olfactory process works.
Furthermore, these results can be used to investigate the significance of OR CNVs in terms of  both the sensory specific satiety and evolution of primates.


\newpage

%Referencing:
\bibliographystyle{BiocRefStyle}
\bibliography{../../../References/BibTeX/BIOC463}

\end{document}
