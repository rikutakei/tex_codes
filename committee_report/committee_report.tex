\documentclass[12pt, a4paper]{article}

\usepackage[utf8]{inputenc}
\usepackage[T1]{fontenc}
\usepackage{natbib}
\usepackage{geometry}
\usepackage{fixltx2e}
\usepackage{enumitem}
\usepackage{amsmath}
\usepackage{mathptmx}

\geometry{
    a4paper,
    left=2.5cm,
    right=2.5cm,
    top=2.5cm,
    bottom=2.5cm
}

\linespread{1.3}

\begin{document}

\begin{center}
\Large{\textbf{Body Mass Index and Pathway Dysregulation in Cancer}}\\
\end{center}

\begin{center}
Riku Takei (MSc)\\
Supervisor: Assoc. Prof. Mik Black
\end{center}

\subsection*{50 Words Summary}

Recent studies in breast cancer have shown that there are obesity-specific genetic signatures, and these signatures were further investigated in other cancer types (Creighton and fuentes mattei paper).
The obesity genes identified by Creighton \textit{et al.} did not correlate with any of the cancer types investigated.

\subsection*{Introduction}



\newpage

\subsection*{Methods}

\begin{description}[leftmargin=0pt]

\item[Validation of the methodology used.]

The original CEL files of all the samples used by Creighton \textit{et al.} were downloaded from Gene Expression Omnibus (GEO) database (website ref).
The microarray expression data for all 103 samples were loaded into R/RStudio and were RMA normalised using the package '????'.
The 799 obesity gene probes were extracted from the data and singular value decomposition was applied to this matrix to generate a metagene of the obesity genes.
The metagene was compared with the BMI data of the samples which was plotted as a boxplot (BMI status) and a dotplot (BMI values).

\item[Identification of samples with BMI data.]

The clinical data for all cancer types (33 in total) were downloaded from The Cancer Genome Atlas (TCGA) database and were checked whether the samples had both height and weight data.

\item[BMI calculation and categorisation.]

    BMI were calculated using the samples' height and weight data, where:
    \begin{equation*}
    BMI = \frac{Weight(kg)}{Height^2(cm^2)}
\end{equation*}
Samples were then categorised into normal weight (BMI \textless{ }25), overweight (25 $\le$ BMI $<$ 30) and obese (BMI $\ge$ 30).

\item[Assessing whether the metagene is associated with BMI.]

Once the metagenes were obtained for each cancer type, the metagene was plotted against BMI status and raw BMI value.

\item[Processing the cancer data.]

RNA-seq data for all of the cancer types with both height and weight data (8 in total) were downloaded from the International Cancer Genome Consortium (ICGC).
All of these data were loaded into R/RStudio and were processed into a matrix form.
The data were then logged to the base 10, and standardised so that each gene had a mean of 0 and standard deviation of 1.

\item[Transforming the cancer data.]

Transformation matrix was used to obtain the metagene for the other cancer data.
In order to transform the data, the transformation matrix from the original data must have the same number of genes as the transforming data, so the genes present in both the Creighton \textit{et al.} data and all of the other cancer types were identified.
Transformation matrix was made from the Creighton \textit{et al.} data using these common genes, and then this transformation matrix was used to transform all of the cancer types.

\end{description}

%\newpage

\subsection*{Results and Discussion}

Since the project was based on the results presented by Creighton \textit{et al.} (2012), the methods used in this project was applied to the raw data from their paper and the results were compared to validate whether the methodolgy to be used in this project is applicable to other data.

After the Creighton \textit{et al.} data were normalised using RMA method, singular value decomposition was applied to the data to obtain the obesity metagene and plotted on a heatmap (fig1a).
The resulting heatmap showed that the overall expression of the obesity-specific genes correlated with the metagene.
To assess whether the metagene was actually correlating with the sample BMI, the metagene was plotted against the sample BMI status and sample BMI value (fig1b and c).
These plots clearly showed that the metagene was able to identify and separate the obese samples from the normal and overweight samples.
Together, these plots confirmed that the methodology applied to the data set was able to recapitulate the results presented by Creighton \textit{et al.}.

The metagenes for other cancer types were generated by using the obesity-specific genes from breast cancer.
Transformation matrix was applied to each cancer type to generate the metagene and was plotted on the heatmap (fig2).
All of the cancer types showed significant correlation with the metagene created from the transformation, where the high metagene values corresponded with high overall gene expression, and vice versa.

To confirm whether the association was in fact due to BMI of the samples, plots of metagene against BMI status and BMI value were created.
In contrast to the Creighton \textit{et al.} plots, no cancer types showed any significant separation between the obese samples and the normal/overweight samples.
These results suggest that the obesity-specific genes from breast cancer were not specific enough to be used as a genetic marker for other cancer types.

Although these current results do not show any significant correlation of the obesity-specific genes and the sample BMI in different cancers, there are many places for improvements.
The data processing of the raw RNA-seq data was carried out using a program written by myself, which could have processed the data incorrectly.
There are other programs to process these raw data, so the data processing procedure should be repeated and compared with the current results.

The obesity-specific genes in each cancer type was not checked with the BMI values of the samples.
The results presented here only showed the association of the metagene with the BMI value and status, and it would be good to double-check whether the expression of the genes show any correlation with the BMI of the samples.

For future experiments, differential expression analysis for each cancer type should be carried out to identify the genes that are dysregulated in obese samples but not in the normal or overweight data.
These genes can then be compared with the genes from other cancer types to identify genes that are  commonly up or down regulated in different cancers.
More recent paper by Fuentes-Mattei \textit{et al.} (2014) have also found obesity-specific genes in breast cancer, and these genes should be investigated as well.

%\citep{Hanahan2011}

%%Referencing:
%\bibliographystyle{BiocRefStyle}
%\bibliography{../../../References/BibTeX/MSc}

\end{document}
