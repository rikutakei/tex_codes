\section{History of ATP Synthase}

For years, scientists were not able to find any ``energised'' intermediates for substrate level phosphorylation.
Peter Mitchell proposed the chemiosmotic coupling theory, which is a theory that the ETC forms a proton gradient between the inner mitochondrial membrane, and this proton gradient drives the ATP synthesis.
This theory was proposed on the basis that the inner mitochondrial membrane was impermeable to proton, ETC was pumping protons into the intermembrane space from the matrix, and this proton gradient drove the ATP synthesis.

\section{Proton Motive Force}

\begin{center}
% \includegraphics[width=0.6\textwidth]{pmf}
\end{center}

Proton Motive Force is the electrochemical gradient formed by the proton gradient, and this drives the synthesis of ATP.
This proton gradient produces 2 kinds of energetic gradients: pH (or chemical) gradient sue to the difference in proton gradient, and electrical gradient as a result of the charge difference across the membrane (positive (\ch{H3O+}) in intermembrane space, and negative (\ch{OH-}) in the matrix).
The energy of this gradient (about 180mV) drives the ATP synthesis.

\section{Experimental Proof of pmf}

\subsection{Experiment with Mitochondria with no Outer Membrane}

When there was no outer membrane of the mitochondria, there was no ATP produced, even though the ETC was still active.
If an energised intermediate was present, then ATP would have still been produced.

\subsection{Experiment using an Uncoupler}

Uncouplers dissipate the pmf by shuttling the protons across the inner mitochondrial membrane, and thus uncouples the proton gradient from ATP synthesis.
When uncouplers were introduced, ATP synthesis was stopped, showing that the pmf was necessary for ATP synthesis.

\subsubsection{Dinitrophenol (DNP)}

\begin{center}
% \includegraphics[width=0.6\textwidth]{dnp}
\end{center}

The use of dinitrophenol (an uncoupler) causes your body to heat up, as the proton gradient is dissipated as heat.
You would feel lethargic as well, as you're not making as much ATP as you're supposed to (due to no proton gradient for ATP synthesisdue to no proton gradient for ATP synthesis).
There is no known lethal dose, but it's probably a good idea not to use it.

\subsection{Liposome with Light-Inducible Proton Pump}

\begin{center}
% \includegraphics[width=0.3\textwidth]{lipomt}
\end{center}

In this experiment, a liposome was created with a light-inducible proton pump and F$_o$F$_1$ ATP synthase.
When the light was shone on the liposome, the proton was pumped into the liposome, forming a proton gradient.
When this gradient was formed, they observed the production of ATP by the ATP synthase.
This experiment showed that only the pmf was sufficient to drive ATP synthesis, and does not require ETC (as long as there was a proton gradient).

\section{F$_o$F$_1$ ATP Synthase}

\begin{center}
% \includegraphics[width=0.4\textwidth]{atpsynthase}
\end{center}

F$_O$ refers to the membrane part of the ATP synthase, and the F$_1$ part refers to the matrix part of the ATP synthase.
ATP synthase is literally a molecular motor, driven by the proton motive force.

\subsection{Mechanism of ATP Synthesis}

There are 2 parts to the ATP synthase: the rotor subunits which rotate/move ($\gamma$ subunit), and the stator subunits which do not move/rotate ($\alpha$ and $\beta$ subunits).
The proton flow through the rotor subunits drives the rotor movement, and this in turn causes a conformational changes in the stator subunits and drives the ATP synthesis.

The $\alpha/\beta$ pairs of F$_1$ binds to ATP or ADP, depending on its conformation.
When it's in an open (O) state, the pair is able to release ATP or bind ADP + Pi.
When it's in a tight (T) state, the catalysis of ATP is able to occur.
When it's in the loose (L) state, the pair is able to hold the ADP and Pi in preparation for catalysis in the T state.
The conformations rotate from O, then to L, then to T, and repeats.

\section{Energy Accounting}

ATP synthase makes an ATP per 4 protons being shuttled across the membrane.
So, for every NADH, 2.5 ATP are made, and for every FADH$_2$, 1.5 ATP are made.

