\section{Digestion and Absorption of Food}

Food macromolecules have to be broken down into smaller molecules for absorption, so it can be used by the body.

\section{Organs Involved in Digestion}

\subsection{Salivary Glands}

Produces saliva -- neutral pH and contains mucous and amylase that starts the digestion of carbohydrates.

\subsection{Stomach}

For food storage (controlled release into the duodenum) and mixing of the food with gastric juices.
Stomach release highly acidic acid (0.1M HCl) that denature the proteins.
It also secretes pepsinogen (zymogen) for protein digestion, and also mucous layer for protection.

The parietal cells produce the acid (HCl), and the chief cells produce the pepsinogen, which is activated by the acidic environment.

\subsection{Pancreas}

Pancreas is slightly alkaline pH 7, and secretes most digestive enzymes, including amylase, lipase and many proteases.

\subsection{Liver}

Synthesis of bile salts/acids (stored in gall bladder), which is important for fat digestion.

\subsection{Small Intestine}

Where the final phase of digestion and absorption occurs.

\section{Beaumont's Experiment}

Beaumont tested a range of food for digestion using the stomach acid from St Martin.
He showed that the digestion of food was dependent on temperature, mixing, stress, etc.
He also noted that he wasn't able to completely reproduce the acid, most likely due to the enzyme component of the gastric juice.

\section{Digestion}

There 2 main phases involved in digestion:
\begin{enumerate}
\item Hydrolysis of bonds connecting the monomer units in food macromolecules.
\item Absorption of products from the gut into the body.
\end{enumerate}

In the case of carbohydrates, proteins and TAGs, the bond to cleave would be the glycosidic bonds, peptide bonds and ester bonds, respectively.

\section{Carbohydrate Digestion}

Carbohydrates make up 40 to 50\% of the energy intake.
Carbohydrates include starch (amylose and amylopectin), glycogen, simple sugars and fibre (i.e. cellulose).
Cellulose cannot be digested in humans, as we lack the enzyme for breakdown of cellulose.

\subsection{Enzymes Involved in Carbohydrate Digestion}

The digestive process begins in the mouth where the salivary amylase starts breaking down the starch molecules into smaller sugars.
When it enters the small intestine, the pancreatic amylase (secreted from the pancreas) further breaks down the starch molecules into maltose and isomaltose.
The intestinal epithelial cells secretes maltase, isomaltase, lactase and sucrase, which breaks down the disaccharide molecules into monomers (glucose, fructose and galactose).
These monosaccharides are then absorbed into the body via the intestinal villi.

\begin{reactions*}
Maltose/Isomaltose &->[Maltase][Isomaltase] 2 Glucose\\
Sucrose &->[Sucrase] Fructose + Glucose\\
Lactose &->[Lactase] Galactose + Glucose
\end{reactions*}

\subsection{Lactose Intolerance}

This is caused by lactase enzyme deficiency.
Since lactose is not digested, it accumulates inside the intestine where the gut microbes can use it for fermentation.
Lactose intolerance causes bloating, flatulence and diarrhoea due to the fermentation.
Individuals with this should avoid lactose consumption.

\section{Carbohydrate Absorption}

Monosaccharides are absorbed into the body via the villi and microvilli (brush border) of the intestinal wall.
The villi and microvilli provides a large surface area for absorption.

\begin{center}
% \includegraphics[width=0.8\textwidth]{glucoseabsorption}
\end{center}

\subsection{Glucose Transport}

Sugars are water soluble, and therefore require a transporter to pass through the cell membrane into the cell.
SGLT1 protein is present in the intestinal side of the cell membrane and transports glucose into the intestinal cell together with the Na ion (symport).
Movement of Na ion from the intestine into the cell is down the concentration gradient, which provides the energy for the glucose transport up the gradient into the cell.

On the other side of the cell membrane, GLUT2 protein is present.
Since the concentration of glucose in the cell is higher than the blood, the glucose can move down the concentration gradient through the GLUT2 transporter.
Note that there is Na/K pump on this side of the membrane to maintain the Na concentration gradient.

Once transported into the blood, other tissues can uptake glucose via GLUT3 (brain) and GLUT4 (muscle and adipose).

\subsection{Coeliac Disease}

Disease of the small intestine, where the body reacts against the wheat protein gluten.
This causes the villi to be flattened and nutrients are not absorbed properly.
