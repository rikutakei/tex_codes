\section{Starvation}

During starvation, your body needs to supply the brain with fuels (i.e. 120g of glucose per day).
Fatty acids can be utilised by other tissues.
Proteins can also be used as a fuel, but we must conserve as much proteins as possible during starvation, as it would affect the structure and function of the protein, leading to death/trouble.

Glucagon is the main hormone responsible during starvation, and is produced in the $\alpha$ cells of the pancreas.

\subsubsection*{Quick Calculation}

Imagine that we have about 10 kg of fat in our body, and the atwater factor of fat is 38kJ/g (40kJ/g to make the calculation easier).
This means that we have about 400000kJ of energy available to us.
Since we require about 100000kJ of energy per day, we can survive for about 40 days with just fat.
However, our brain needs glucose, not fat, for fuels, so we need an alternative mechanism to provide the brain with energy.

\section{Liver Glycogen}

There is about 90 to 120 g of glycogen that we can mobilise it back to glucose (stimulated by glucagon).
This glycogen store can be utilised and provides enough glucose for the brain for one day.
Glycogen can be broken down into glucose by the enzyme glycogen phosphorylase and debranching enzyme.
The resulting G1P is then converted into G6P, which can then be fed into glycolysis (in muscle) or be used to make new glucose (in liver).
Note that there are glycogen in muscle cells, but these cannot be converted back into glucose as muscle cells don't have the enzyme glucose-6-phosphatase enzyme that is present in the liver cells.

\begin{center}
% \includegraphics[width=0.8\textwidth]{gluconeogenesis}
\end{center}

\section{Gluconeogenesis}

As mentioned earlier, this occurs mainly in liver (some in kidney cortex), and is stimulated by glucagon.
Glucose is synthesised using three fuel molecules: lactate (from RBC and muscle), alanine (from muscle protein), and glycerol (from TAG in adipose tissue).
Gluconeogenesis is an energy requiring process, and this energy is provided by fatty acid oxidation.
Most of the glucose produced is used by the brain.

\begin{center}
% \includegraphics[width=0.8\textwidth]{gluconeogenesis2}
\end{center}

Just a quick note that both lactate and alanine feed in at the pyruvate level, and so requires energy input to make glucose, but glycerol can feed in at the DHAP level and require no energy to make new glucose.

\section{Maintaining Fuel Supply to the Brain}

Our glycogen store is able to support the brain for a day -- what do we use after we've used up the glycogen?
We use up about 200g of TAGs per day when we are starving, and we can use 20g of glycerol from TAGs to make new glucose (which accounts for about 20g of glucose).
This means that we still need about 100g of glucose to fuel our brain -- where does it come from?

\subsection{Protein Stores}

We can break down protein to provide the source of new glucose.
We have got about 10 to 15kg of protein inour body, but we don't have any specific storage proteins for fuel.
This means that some proteins must be degraded to make glucose.
However, this can only last for up to 2 weeks before we die from other complications(i.e. we cannot rely on protein too heavily).
So, we can break down proteins, but we have to conserve as much protein as possible to prevent structural and functional damage to your body.

\subsection{Ketone Bodies}

\begin{center}
% \includegraphics[width=0.7\textwidth]{moleculesfasting}
\end{center}

An experiment done Cahill and Owen in 1967 showed that the blood glucose in your body was maintained during fasting and there was an increase in FFA concentration due to TAG breakdown.
They also noted that $\beta$-hydroxybutyrate and acetoacetate concentration have increased by 100 times during fasting, and showed that about 50g of these ketone bodies were used by the brain.
This meant that only 50g of glucose was needed to be made from the breakdown of protein.

\subsubsection{Ketone Body Synthesis}

Ketone bodies are synthesised in the liver from fatty acids.
Fatty acids are activated into acyl CoA, which is then broken down into acetyl CoA via $\beta$-oxidation.
Acetyl CoA is then converted into ketone bodies (acetoacetate and $\beta$-hydroxybutyrate) in the process of ketogenesis.
These ketone bodies can cross the blood brain barrier, converted back into acetyl CoA in the brain, which is then used in the CAC to make energy.
Note that these ketone bodies can be spontaneously decomposed into acetone (hence the sweet smell in you breath).

\begin{center}
% \includegraphics[width=0.6\textwidth]{ketogenesis}
\end{center}

\section{Metabolic Adaptations to Starvation}

To summarise, the body uses all of the mechanisms described above to keep the body alive and adapt to starvation.
Fatty acids are used by all aerobic tissues (except brain) for energy.
Once we use up all of the glycogen stores, excess fatty acids are converted into ketone bodies for the brain to use, and therefore prevent/conserve the amount of proteins broken down (and therefore stay alive for longer).

