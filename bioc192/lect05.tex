\section{Allosteric Effectors of Haemoglobin}

As discussed earlier in Lecture 4, binding of O$_2$ molecule to the haem group causes a conformational change, and thus promotes the R or T state.
This conformational change is caused by the shift in the Fe$^{2+}$ atom of the haem, which is attached to the His F8 of the haemoglobin subunit, which then cause a slight shift that propagates through the protein.

However, there are other molecules that bind to the haemoglobin \textbf{at a different site as where the O$_2$ binds to}.
Since these molecules have effects on haemoglobin, yet it binds to a distant position of the haemoglobin, these molecules are called the allosteric effectors.

\vspace{0.5cm}

\begin{center}
\textbf{Allosteric effectors are molecules that bind to the allosteric protein at a place other than/different to the normal substrate binds to.}
\end{center}

\section{2,3-Bisphosphoglycerate (BPG)}

2,3-Bisphosphoglycerate (BPG) is a small, negatively charged (-5 charge) molecule that binds to the deoxy-haemoglobin at the center of the tetramer.
Binding of BPG causes the haemoglobin to be in T state, i.e. in the dexygenated state, and decreases the haemoglobin's binding affinity for O$_2$, meaning that the haemoglobin with BPG bound to it will require higher pO$_2$ to oxygenate it.
Binding of BPG at the muscle/tissue site ensures that the O$_2$ molecules are released at the muscles/tissues.

O$_2$ binding to the haemoglobin at the lungs causes the release of BPG molecule, allowing the haemoglobin to carry O$_2$ to the muscles/tissues.

\section{H$^+$ and CO$_2$}

Both H$^+$ and CO$_2$ binds to haemoglobin at a different site as the substrate (O$_2$) and affects the oxygen affinity of haemoglobin.
Both molecules favour the T state of haemoglobin, and therefore favours the deoxygenated state (i.e. decreased O$_2$ affinity).
Note that both of these molecules are related to the pH of the environment, and the blood buffer system is also involved in controlling the concentration of these molecules.

\subsection{The Bohr Effect}

Technically speaking, the Bohr effect is the effect of pH (i.e. [H$^+$]) and [CO$_2$] on the oxygen affinity of haemoglobin.
As pH decreases (therefore [H$^+$] increases), the net charge of haemoglobin changes due to different amino acids getting protonated, which leads to slight conformational changes in the haemoglobin structure, and therefore decrease in haemoglobin's oxygen affinity.
Similarly, as CO$_2$ concentration increases, CO$_2$ is more likely to bind to the haemoglobin and decrease the oxygen affinity of haemoglobin.

\subsection{The Blood Bicarbonate Buffer System}

In the blood, there is an equilibrium between bicarbonate ([HCO$_3^-$]) ion and CO$_2$.

\begin{center}
\ch{H$_2$O + CO$_2$ <=> HCO$_3^-$ + H$^+$}
\end{center}

As you can see, the [H$^+$] concentration and the [CO$_2$] concentration is related to one another due to this buffer system.
This means that at the tissue/muscle where the relative concentration of CO$_2$ is high, it is going to be making the bicarbonate ion, together with H$^+$ ion, and therefore the pH is decreased.
However, at the lungs where the concentration of CO$_2$ is low, there will be less H$^+$ ion present, and therefore have higher pH.

Together with the Bohr effect, this means that the pH is high at the lungs and have low CO$_2$ concentration, and therefore haemoglobin will have high oxygen affinity, and thus get saturated with O$_2$ (oxygenated).
In contrast, the pH will be lower at the tissues/muscles and have high CO$_2$ conference, and therefore haemoglobin will have low oxygen affinity, and thus releases O$_2$ at the tissue/muscle (deoxygenated).

Also note that BPG is released at the lungs when O$_2$ binds to the haemoglobin, and binds at the tissue/muscle when O$_2$ is released and CO$_2$ and H$^+$ binds to it, contributing to the binding and release of O$_2$.(insert the saturation curve of +/- allosteric effectors)

