\section{Integration of Metabolism}

\subsection{Brain Metabolism}

Brain aerobically oxidises glucose and ketone bodies (during starvation).
The key pathways used by the brain are glycolysis, CAC, ETC and oxidative phosphorylation, in the presence of oxygen.

\subsection{Skeletal Muscle Metabolism}

Skeletal muscle can aerobically or anaerobically oxidise fuel molecules, depending on the energy requirement.
It can aerobically oxidise glucose, FAs, ketone bodies, and TAGs (to make FAs).
During anaerobic metabolism, lactate is produced and is released into the blood (goes to liver and used as fuel or gluconeogenesis).
Proteins can be broken down into amino acids under starving condition and used as fuel, or it can be released into the blood.

\subsection{Hear Muscle Metabolism}

Heart is completely aerobic and able to oxidise pretty much all fuels -- glucose, FAs, ketone bodies and lactate (which is converted into pyruvate for CAC).
Key pathways include glycolysis, $\beta$-oxidation, CAC, ETC, and oxidative phosphorylation.

\subsection{Adipose Tissue Metabolism}

This is the major TAG store, and also where TAGs are stored/mobilised.
Mobilisation of FAs are done by hormone sensitive lipase and FFA are released into the blood for other cells to use, whereas the glycerol backbone is sent to the liver for gluconeogenesis.
Adipose tissue also secretes various hormones such as leptin, but it can also synthesise and secrete many other hormones (referred to as adipokines).

\subsection{Liver Metabolism}

Major metabolic processing organ, and is the first place to have access to the fuels after digestion of food.
It also synthesise and stores glycogen, and is also the major site of fat synthesis (which is exported as TAGs in VLDL).
Glycogenolysis and gluconeogenesis also occurs in the liver.
Synthesis of ketone bodies and conversion of toxic excess ammonia to urea is also done here.

\section{Signalling Your Body}

\subsection{Endocrine System}

Chemical signals that are synthesised in the endocrine gland, and is released into the blood to act on distant target tissues.
These regulates metabolic activities and cellular processes.

\subsection{Nervous System}

Neurotransmitter chemicals are synthesised and released from the nerve endings into the synapse.
These act on the post synaptic cells (e.g. nerve/muscle).

\subsection{Types of Hormones}

\begin{description}
\item[Steroid hormones] e.g. estrogen, testosterone, etc.
\item[Polypeptide/peptide hormones] e.g. insulin, glucagon, etc.
\item[Amino acid derivatives] most neurotransmitters.
\end{description}

\subsection{Functions Regulated by Signal Molecules}

Pretty much everything that the cell needs to do to keep us alive -- making energy, growing, wound healing, metabolism, etc.
Hormones and other chemical signals:
\begin{description}
\item[Stimulate/inhibit the activity of enzymes or other proteins in cells] It can regulate metabolic pathways and secretory processes.
\item[Alter transcription of specific genes] and therefore regulate the amount of enzymes and proteins produced.
This means that it regulates growth and differentiation.
\end{description}

\subsection{Signal Transduction and Diseases}

\begin{description}
\item[Type II Diabetes] Insulin signal transduction is impaired due to resistance/loss of insulin.
New drugs target the signal transduction pathway of insulin to stimulate the pathway.
\item[Cancers] Overactive signal transduction stimulates the cells to divide and leads to tumour.
New drugs to block unwanted signal transduction.
\end{description}

