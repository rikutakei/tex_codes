\section{Enzyme Kinetics}

It is important for biochemists to be able to compare different enzymes, so they can compare how well an enzyme is to the rest.
One way (or maybe the only way) to do this is to compare the enzyme kinetics -- how well an enzyme can catalyse a reaction.

\section{Measuring Enzyme Catalysis}

Enzyme catalysis of a reaction can be measured from the progress curve of a reaction.
Progress curve is the graph of how much product is made (or how much substrate is used up) over time.
It is important to measure the initial reaction velocity of the reaction from this progress curve in order to calculate some other important measures of enzyme kinetics.

This initial reaction velocity (V$_i$) is the slope of the tangential line of the initial part of the progress curve.
Only the initial part is used, because as you progress through the reaction, some factors start to limit the reaction and becomes unreliable.

\section{Effect of Enzyme Concentration on the Rate of Reaction}

Consider the situation where there is an excess substrate (i.e. unlimited source of substrate).
As the concentration of the enzyme increases, the rate of reaction increases.
\textbf{This means that the velocity of the reaction (V) is proportional to the enzyme concentration, when there is excess substrate.}

This is expected, since there are more enzymes to catalyse the reaction -- as more enzymes are introduced to the system, more enzymes can react with the substrate, and therefore produce more products.

\section{Effect of Substrate Concentration on the Rate of Reaction}

Consider the situation where the concentration of enzyme is fixed.
As we increase the concentration of substrate, the rate of reaction will initially increase linearly, but as more and more substrates are introduced, the rate of reaction reaches a `maximum'.
This is because all of the active site of the enzymes become occupied as the substrate concentration increases -- it reaches the `maximum capacity' of the enzyme, and the enzymes cannot increase the rate of reaction no matter how hard it tries.

In kinetics term, the enzyme shows first-order kinetics initially (where the rate depends on the substrate concentration), but then shows zero-order kinetics (where the rate does not depend on/changes with substrate concentration).
When the velocity of the reaction is plotted with the substrate concentration, this is quite apparent from the hyperbolic curve of the graph.

\subsection{V$_{max}$ and K$_M$}

From this V vs S curve, we can identify two kinetic parameters: V$_{max}$ and K$_M$.
V$_{max}$ is defined as the maximum velocity of the reaction when an enzyme is saturated with substrate.
K$_M$ is defined as the \textbf{substrate concentration} at which the velocity of the reaction is half the maximum velocity ($V = \frac{V_{max}}{2}$).
K$_M$ is also called the Michaelis constant.

Note: do NOT get this mixed up with ``K$_M$ = $\frac{V_{max}}{2}$'' -- remember that K$_M$ is the \textbf{substrate concentration} when V = $\frac{V_{max}}{2}$.









