\section{Oxidative Phosphorylation -- Part 2}

\begin{center}
% \includegraphics[width=0.6\textwidth]{etc}
\end{center}

\subsection{Complex I}

Complex I is where the electrons from NADH are accepted/transferred (i.e. where NADH is oxidised).
Two electrons are fed into the ETC from NADH, via this complex.
During this process of transferring the electrons, four protons are pumped into the intermembrane space of the mitochondria.

\subsection{Complex II}

Complex II is where the electrons from FADH$_2$ are accepted/transferred (i.e. where FADH$_2$ is oxidised).
Two electrons are fed into the ETC from FADH$_2$, via this complex.
However, no proton is pumped through this complex -- just electrons through the ETC.
Note that succinate dehydrogenase from CAC is associated with this complex, allowing the enzyme-bound FADH$_2$ to be oxidised and feed into the ETC.
Another thing to mention is that the FADH$_2$ from $\beta$-oxidation (done in the matrix) transfers its electrons via the electron transferring protein, which then feeds into the ubiquinone and not through complex II.

\subsection{Co-Enzyme Q (Ubiquinone)}

Co-enzyme Q (or ubiquinone) is a mobile electron carrier that transfers electrons from complex I/II to complex III.
Co-Q has a characteristic of being able to transfer one electron at a time, which NADH or FADH$_2$ cannot do.
Since complex III requires the electrons to be transferred one at a time, CoQ is an ideal electron carrier from complexes I/II.
The state of ubiquinone when only one electron is accepted is called semi-quinone.

\subsection{Complex III}

Complex III accepts one electron at a time from CoQ and release/transfer this electron to cytochrome c (again, one at a time).
During this transfer process, complex III pumps four protons into the intermembrane space.

\subsection{Cytochrome c}

Cytochrome c contains a haem, and carries one electron at a time to complex IV via a reversible Fe$^{2+}$/Fe$^{3+}$ redox reactions.

\subsection{Complex IV}

Biologically, the production of water occurs as follows:

\begin{center}
\ch{4 H+ + 4 e- + O2 -> 2 H2O}
\end{center}

This shows that the reaction requires four electrons to carry this reaction out.
Complex IV is able to ``hold'' four electrons and wait for the electrons to accumulate, and then do the reaction to produce water.
This complex also pumps two protons into the intermembrane space.

\subsection{Energy Accounting}

To account for the energy, the number of protons must be counted for each reduced co-enzymes.
For NADH, two electrons are transferred to complexes I, III, and IV, meaning that 10 protons in total are pumped into the intermembrane space.
For FADH$_2$, two electrons are transferred to complex II, III, and IV, meaning that 6 protons in total are pumped into the intermembrane space.

\section{Poisoning the ETC}

\subsection{Rotenone}

Rotenone inhibits electron transfer from complex I to coenzyme Q

\subsection{Cyanide}

Cyanide binds to cytochrome a3, which is in complex IV.

\subsection{Carbon Monoxide}

Carbon monoxide binds to where oxygen binds on complex IV, and therefore prevents the electron transfer.

\subsection{Consequences}

All of these will stop the electrons from flowing through the ETC.
This will in turn prevent the oxidation of NADH and FADH$_2$, which means that these co-enzymes cannot get recycled back into its oxidised form (NAD$^+$ and FAD).
If there are no NAD$^+$ or FAD around, $\beta$-oxidation and citric acid cycle shuts down as well.
As a result, protons don't get pumped and therefore no (or decreased) ATP produced.
Reactive oxygen species can also be produced when the ETC is inhibited, as the electrons have to be passed onto somewhere.
Since this transfer is from another complex, the electron transfer is incomplete and results in the production of highly damaging reactive oxygen species.

