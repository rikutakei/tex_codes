\section{Minerals}

\subsection{Selenium}

There is about a mg to \si\micro g of selenium per kg of body weight.
Selenium functions as a cofactor for enzymes such as glutathione peroxidase and iodothyronine deiodinase.

Selenium deficiency in animals lead to white muscle disease in sheep, where the sheep is still born or die within a few days, and shows unsteady gat and arched back.
In humans, selenium deficiency can cause the Keshan disease, and is also associated with cancer.

NZ soil has selenium deficiency, and can be problematic (use fertiliser, etc.).
50 to 60\si\micro g of selenium per day is required, and the RDI is about 60 to 70\si\micro g per day.
The upper limit is about 400\si\micro g per day.

\subsection{Iodine}

Iodine deficiency can lead to goitre, and iodine cannot be stored in the body (and therefore excreted).
Over 100\si\micro g per L is adequate.
Less than this can lead to hypothyroidism, goitre, impaired mental function and eventually cretinism and abortions/still births.

\subsection{Iron}

Most common deficiency around the world.
Iron is important for Fe$^{2+}$/Fe$^{3+}$ states, oxygen transport, and assists many enzymes in their reactions.

RDI = 8 to 18mg per day (for men and women, respectively).

Factors that influence the iron intake are supply (amount of iron consumed, haem vs non-haem, and promoters/inhibitors of iron absorption) and requirements for iron (increased absorption is iron status is low, menstruation, growth, pregnancy, infections, etc.).

\begin{center}
\begin{tabular}{c | c | c}
& \% intake & \% absorbed \\
\hline
Haem & 10-15 & 25-35 \\
Non-Haem & 85-90 & 2-10\\
\end{tabular}
\end{center}

From this table, you can see that the haem iron is more available and better absorbed than the non-haem irons.

Iron absorption promoters include vitamin C (non-haem) and eating vegetables with meat (non-haem).
Inhibitors include phytates, compounds in coffee/tea, EDTA (preservative) , calcium, zinc and manganese.

\subsection{Zinc}

Zinc is required as a co-factor for over 300 enzymes.
RDI = 8 to 14mg per day (women and men, respectively).
Zinc is low in vegetables, and therefore deficiency is often seen in vegetarians.

Mild zinc deficiency leads to stunted growth in children, decreased taste sensation, and impaired immune functions.
Severe deficiency leads to dwarfism, delayed sexual maturation, and hypopigmented hair.
Causes of zinc deficiency is from diet that is low in meat, staple based on beans, unleavened bred, other wholegrain foods that are high in fibre and phytate.
