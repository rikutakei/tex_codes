\section{Transport of Non-polar Molecules Through the Membrane}

Transport of non-polar molecules through the membrane is dependent on both the hydrophobicity and the concentration of the molecule.
Unlike polar molecules, non-polar molecules are relatively easier to diffuse through the membrane due to its hydrophobic nature of the molecule.

\subsection{Permeability Coefficient and Oil-Water Partition Coefficient}

\begin{center}
% \includegraphics[width=0.5\textwidth]{permeability}
\end{center}

The permeability coefficient of a molecule describes how well the molecule is able to pass through the membrane -- i.e. how permeable the molecule is.
The smaller the coefficient is, the harder for the molecule to pass through the membrane, and therefore less permeable.

The oil-water partition coefficient describes how hydrophobic a molecule is.
This means that the higher the oil-water partition coefficient is for that molecule, the more hydrophobic the molecule is.

When these two factors are graphed for different kinds of molecules, you can clearly see that the higher the oil-water partition coefficient (i.e. more hydrophobic the molecule is), the higher the permeability coefficient.
This shows that the more hydrophobic a molecule is, the easier for that molecule to pass through the membrane.

\subsection{Concentration Gradient vs. Flux}

\begin{center}
% \includegraphics[width=0.5\textwidth]{flux}
\end{center}

When the  flux  is plotted against the concentration gradient of the molecule across the membrane, it is clear that the flux is proportional to the concentration gradient.
This means that the higher the concentration gradient across the membrane, the higher the flux of that molecule through the membrane (i.e. faster/more will go through).

\section{Transport of Polar Molecules Through the Membrane}

Unlike the non-polar molecules, the lipid bilayer represents an energy barrier for the transport of polar molecules across it.
So, for the polar molecules to pass through the membrane, the energy barrier must be decreased.
Proteins within the membrane make small holes in the membrane to allow polar molecules to pass through.

\subsection{Channels}

\begin{center}
% \includegraphics[width=0.4\textwidth]{channels}
\end{center}

Channels, when opened, are open to both the intracellular and the extracellular space (i.e. open to both ends).
These channels can be opened or closed spontaneously, and are ``gated''.
Since the mechanism of the channels are fairly simple, the rate of transport of molecules across the membrane when opened can approach the rate of diffusion -- about 10$^7\sim$10$^8$ molecules per second.
The structure of the holes of the channels provide some specificity of the molecules that can go through the membrane.

\subsection{Transporters}

\begin{center}
% \includegraphics[width=0.5\textwidth]{transporter}
\end{center}

Unlike the channels, transporters are only open to either the intracellular or the extracellular space.
The binding of the molecule that is to be transported induces a conformational change in the transporter which opens the opposite end, allowing the release of the molecule to the other side.
The release of the molecule causes another conformational change that reverts the transporter back into its original state and the process repeats.
Since the process of transporting the molecule requires conformational change of the transporter itself, the rate of transport is much slower compared to the channels -- about 10$^2\sim$10$^3$ molecules per second.

Another characteristic thing about transporters is that they are very much like an enzyme -- transporters are very selective of what they move through the membrane, and it also shows a Michaelis-Menten-like kinetics.
For example, glucose and mannitol are two very similar structured molecules that have similarly low permeability coefficient without a transporter.
However, when GLUT transporter is present, the permeability coefficient increases a lot for the glucose, but not for mannitol, showing that the transporters are selective in which molecule it transports.

\begin{center}
% \includegraphics[width=1.0\textwidth]{kinetics}
\end{center}

The kinetics of the transporter is represented by the equation below, where J is the flux and [A] is the concentration gradient of the molecule across the membrane.

\begin{center}
\large{$J_A = \frac{J_{max}[A]}{K_M + [A]}$}
\end{center}

\section{Types of Transport Processes}

\begin{description}
\item [Non-mediated] Requires no protein for transport -- free diffusion.
\item [Mediated] Requires a protein. Also called facilitated transport.
\item [Passive] Down a concentration gradient and no energy is required for the transport.
\item [Active] Up a concentration gradient and requires an input of energy. Energy can come from ATP hydrolysis or from the co-transport of another molecule moving down its concentration gradient.
\item [Co-transport] Movement of two molecules by a transporter.
\begin{description}
\item [Symport] Both molecules move through the transporter in the same direction.
\item [Antiport] The two molecules move in the opposite directions to one another.
\end{description}
\end{description}

\vspace{0.5cm}

\noindent
\textbf{Note: for symport and antiport, both molecules may not be going down the concentration gradient -- one molecule may be going up the concentration gradient, whereas the other maybe going down the concentration gradient.}


