\section{Membrane Lipids and Proteins}

Membrane lipids allow the cells to make compartments within the cell.
This allows the cell to set up specific functions for different compartments (e.g. setting different pH environment, isolate function/group of proteins).
However, there are some issues with membrane lipids -- they act as a barrier for the molecules to pass through the membrane.
To understand how molecules pass through the membrane, you need to  know what the membrane is made up of.

\section{Membrane Composition}

\begin{center}
\includegraphics[width=0.4\textwidth]{fattyacids}
\end{center}

Membrane lipids are made from amphipathic molecules, meaning that the molecules have both charged and uncharged parts.
The hydrocarbon tails are hydrophobic and are 16$\sim$22 carbons long.
In animals, the length of the hydrocarbon tails are usually even.
These hydrocarbon tails can be either saturated (no double bonds) or unsaturated (with double bonds).
Note that the double bonds introduce a kink in the tail.

In aqueous solution, the polar group of the molecules are exposed to the solution, whereas the hydrophobic part forms the inner part of the membrane.
This forms a well-known lipid bilayer, which is energetically favourable for the molecules to form in aqueous environment.

\section{Membrane Lipid Structure}

Membrane lipids all have a glycerol backbone which two fatty acids and a phosphate group attaches to.
As mentioned, the fatty acids can be of different length, with or without double bonds.
The phosphate group provides the charge for the amphipathic molecule, and often have polar groups added to the phosphate.
This provides the group extra charge which can form patches on the surface of the membrane.

Note that there are other types of membrane lipids present in different tissue of the same organism, or maybe in tissues from another organism, depending on the function of the tissue in its organism.

\section{Membrane Fluidity}

Membrane fluidity is affected by temperature, the type of fatty acids present, and the presence of cholesterol.
All of these factors have an effect on the hydrophobic, non-covalent interaction between the lipids.
Reminder: non-covalent interactions are very weak on its own, but strong when there are a lot of them, and the strength of the interaction is also dependent on the distance/hoe close to each other.

\subsection{Temperature}

Temperature affects the membrane fluidity by disrupting the non-covalent interactions between the hydrocarbon tails.
As the number of the carbon increases in the hydrocarbon tails increases, the melting point also increases, and therefore less fluid.
This is because there are more carbons for hydrophobic interaction that strengthens the interaction.

\subsection{Type of Fatty Acid}

The type of fatty acids, or more like how saturated a fatty acid is, affects the membrane fluidity.
When the hydrocarbon tails are unsaturated, that is, it has double bonds in it, it has a kink.
This kink causes the hydrocarbon tails to be more separated and the packing of the membrane lipids less ordered.
Since the kink is going to spread the hydrocarbon tails apart further apart, this increases the distance between the tails, and therefore weakens the hydrophobic interaction between them (therefore more fluid).

\subsection{Presence of Cholesterol}

Cholesterol has a rigid ring structure that can be used as a ``scaffold''.
This allows the lipids to anchor themselves to it and form a more ordered structure, and therefore makes the membrane less fluid.

\section{Lipid Bilayer is a 2D Fluid}

\begin{center}
\includegraphics[width=0.4\textwidth]{lipidmovement1}
\includegraphics[width=0.4\textwidth]{lipidmovement2}
\end{center}

Lipid bilayer is not a solid, static structure -- it is a fluid.
Lipids are able to migrate rapidly in the plane of the membrane, and ``flip-flopping'' between the planes are very rare.
This is because the polar part will have to move through the non-polar part of the membrane.

\section{Membrane Proteins}

The membrane is jam-packed with proteins.
Proteins are also able to move in the membrane -- it is dynamic.
There are many functions of membrane proteins -- cell to cell contact, surface recognition, cytoskeleton contact, enzymes, transporters, receptors and signalling.

Proteins can be either integral (all or some part of the protein is embedded in the membrane), or peripheral (on the surface of the membrane).
Peripheral proteins can be held in pace on the membrane surface by polar patches from, for example, the polar groups attached to the phosphate group, forming an electrostatic interaction.

Note that the proteins have ``sides'' on the membrane, and are asymmetric.
This means that not all proteins that are exposed to the extracellular surface are also exposed on the intracellular surface.
For example, most (if not all) of the proteins that are glycosylated are exposed on the extracellular surface, and not on the intracellular surface.

\section{How Are Proteins Attached to the Membrane?}

Non-polar amino acids of the protein resides within the lipid bilayer -- it is unfavourable for hydrophobic amino acids to move out of the membrane, so the proteins is anchored to the membrane.
Proteins can also have an anchoring domain that can be used to anchor the domain to the membrane.

Lipidation, a type of post translational modification, adds fatty acid groups to the protein which can insert it into the membrane for anchorage.

\begin{center}
\includegraphics[width=0.7\textwidth]{tmproteins}
\includegraphics[width=0.6\textwidth]{anchoredprotein}
\includegraphics[width=0.7\textwidth]{lipidation}
\end{center}








