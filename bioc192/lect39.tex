\section{Obesity}

New Zealand is the third most obese country in the world.
Body Mass Index (BMI) is measured as:
\begin{center}
$BMI = \dfrac{Weight (kg)}{Height^2 (m^2)}$
\end{center}
and define people with BMI $>$ 30 obese, 25 $<$ BMI $\leq$ 30 overweight, 20 $<$ BMI $\leq$ 25 normal, and BMI $\leq$ 20 underweight.

\begin{center}
\includegraphics[width=0.9\textwidth]{energyflow}
\end{center}

\section{Metabolic Rate}

Metabolic rate is fixed, and partially genetic.
However, if we can increase the metabolic rate, then we would be able to lose weight without exercising.
How do we do this?

\section{Uncoupling Protom Motive Force from Oxidative Phosphorylation}

By uncoupling the proton motive force from the oxidative phosphorylaiton will dissipate the energy as heat and burn more fat.

\subsection{Dinitrophenol}

Dinitrophenol can be used to uncouple the proton motive force by shuttling the proton across the inner mitochondrial membrane.

\subsection{Brown Adipose Tissue}

Brown adipose tissue (BAT) is a special thermogenic tissue found in hibernating animals, babies, and some in adults.
BAT keeps the animals warm by generating heat (by uncoupling the pmf).
These cells contain many mitochondria and small fat droplets.

\subsubsection{Uncoupling Protein (UCP)}

UCPs were found in BAT, and are present in the inner mitochondrial membrane.
UCPs are regulated proton channels in the membrane and uncouples the ATP synthesis.
This means that heat is generated due to this uncoupling, and increases the BMR.

\subsubsection{Diet-induced Thermogenesis}

In rodents, activated UCP in BAT burnd off excess dietary energy.
Humans have active BAT, which may be used to burn fat/excess energy.
There are also related UCPs in WAT and muscles, which could potentially be used to burn fat and excess energy.

\subsubsection{Strategies of BAT Usage}

\begin{enumerate}
\item Stimulate existing BAT
\item Switch on BAT differentiation and growth
\item Transplantation of engineered BAT
\end{enumerate}

\section{Genetic Component of Obesity}

There are three genes (ob = obese, db = diabetic obese, and fa = fatty) that make mice fat.

\subsection{Obese Gene (Leptin)}

The ob/ob gene codes for leptin -- hormone secreted from the ``fat'' adipose tissue cells.
Leptin acts on the hypothalamus to let the body know that it is ``full'', and therefore require no more food.
This leads to a decrease in appetite and stimulates energy expenditure (therefore maintains normal body weight).

\begin{center}
\includegraphics[width=0.8\textwidth]{leptin}
\end{center}

In a mouse model, they injected leptin into the ob/ob (obese) mouse.
This stopped the mouse to eat more, and the body weight returned to normal (i.e. lost weight).

Defects in either leptin or leptin receptor (in hypothalamus and other tissue) can lead to increased weight (absent in db/db and fa/fa mouse).

\subsection{Leptin in Humans}

Humans also have leptin and its receptors, and mutations in these can lead to obese phenotype.
This means that the individual will have no leptin or non-responsive to leptin, and therefore always feel hungry.
Most obese humans appear to be resistant to leptin.

\section{Treatments}

\begin{description}
\item[Food breakdown] Target pancreatic lipase (Xenicol) so less fat is absorbed.
\item[Satiety signals] Increase the leptin levels or target leptin receptors.
\item[Mitochondria and brown fat] Uncouple oxidative phosphorylation from electron transport chain by upregulating UCPs and/or increase functional BAT.
\end{description}

