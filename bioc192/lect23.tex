\section{Vitamins}

\subsection{Fat Soluble Vitamins}

\subsubsection{Vitamin A}

Vitamin A is required for vision, integrity of epithelial cell, embryonic development, and  maintenance of immune system.
Xerophthalmia (``dry-eyes'') is a vitamin A deficiency condition that affects your vision/eyes -- causes ulceration and eventually scarring.
Vitamin A (or retinol) is important for your vision, as it is the substrate that binds to the GPCR that sends visual signals to the brain.

Preformed vitamin A (retinol) can be obtained from meat, or it can be converted from provitamin A (carotenoid) into retinol.
$\beta$-carotene (carotenoid) can be cleaved by dioxygenase into two vitamin A.
The activity of dioxygenase varies between species, due to the difference in the diet and therefore the amount of vitamin A obtained.
Herbivores only get carotenoids so they have high dioxygenase activity, whereas cats have no dioxygenase activity, since all the vitamin A is from meat.
A twelfth of carotenoids from food are converted into vitamin A.

Overdosage of vitamin A can lead to spontaneous abortion and fetal abnormalities, and therefore women should not take retinoid therapy during pregnancy.
The maximum dosage per day as retinol is 3000\si\micro g, but there are no upper level for carotenoids (as it gets excreted).

\subsubsection{Vitamin D}

Obtained/synthesised mainly from sunlight (about 20000IU).
It can also be consumed from your diet (vitamin D$_{2/3}$), although half of dietary vitamin D is lost when fried.
In NZ, you require about 5 to 15 \si\micro g per day, where 1IU = 0.025\si\micro g.
Vitamin D intake decreases during winter and it is also dependent on the ethnicity -- darker skin colour makes it more difficult to produce vitamin D.

Vitamin D deficiency can lead to rickets in children.
People who are at risk are people who have low or limited exposure of sun, have dark skin, and/or people who have fat malabsorption.
Overdosage of vitamin D can lead to hypercalcaemia, but there is no set concentration that is known to have an adverse effect.
However, the ``safe'' concentration is about 100ng/mL (250nmol/L).

In NZ, it is a recommended to expose infants to sunlight for at least 5 to 20 minutes (light and dark skins, respectively) before 11am or after 4pm during spring/summer time (Oct--Mar), unless you have sunprotection in between 11am to 4pm.
During winter, infants should spend time outside.

\subsubsection{Vitamin E}

Vitamin E, or tocopherol, can be obtained mainly from plant oil and in fatty tissues of animals.
Vitamin E is important for the membrane integrity by preventing oxidation of unsaturated fatty acids.
Vitamin E Deficiency is rare, and the RDI is 7 to 10 mg per day (women and men, respectively), and the upper limit is 300mg per day.

\subsubsection{Vitamin K}

Vitamin K is important for blood coagulation, and is offered at birth.
Vitamin K deficient bleeding (VKDB) disorder leads to spontaneous bleeding, and is uncommon.
Adequate intake is about 60 to 70\si\micro g per day (women and men, respectively).

\subsection{Water Soluble Vitamins}

\subsubsection{Vitamin B1}

Vitamin B1, or thiamin, is a co-enzyme for the TPP, which is involved in energy metabolism.
25 to 30mg are stored/used in your body, and excess thiamin is excreted out (therefore no overdosage).
Thiamin deficiency leads to Beri-Beri.

\subsubsection{Vitamin B2}

Vitamin B2, also known as riboflavin, is not linked to any characteristic disease and is therefore not fatal.
1.1 to 1.3mg (women and men, respectively) are the recommended intake.

\subsubsection{Vitamin B3}

Vitamin B3, or niacin, can be made from tryptophan, and therefore not technically a vitamin.
Niacin deficiency leads to pallagra and the 4Ds (diarrhoea, dermatitis, dementia and death).

\subsubsection{Vitamin B5}

Vitamin B5, or pantothenic acid, deficiency is rare, but it can cause fatigue, GI and neurological disturbances.

\subsubsection{Vitamin B6}

Vitamin B6 (pyridoxine) is part of the PLP and PMP co-enzymes.
Deficiency is rare, as it occurs in association with other vitamin B group.
RDI = 1.1mg per day.

\subsubsection{Vitamin B7}

Vitamin B7, or biotin, deficiency is rare and leads to non-specific symptoms.
Avidin (found in raw egg white) is a molecule that binds to biotin and prevents absorption.

\subsubsection{Vitamin B9}

Vitamin B9 (folate) is involved in the one carbon metabolism.
Deficiency is rare in developed countries.
Deficiency leads to megaloblastic anaemia (immature RBC and not enough of them) and neural tube defect in infants.

To protect infants from NTDs, supplements can be used to increase the folate in your body.
This will allow correct dose at the correct time, have max-protection and will have no risk to non-target population.
However, half of all pregnancies are unplanned, and it may be hard to consume the supplement at the right time.

Fortification of the food is another option where people can voluntarily consume products with folate.
Food manufacturer can fortify their food (mandatory fortification) so that their products will have low-level of folate in it (prevent excess intake in off-target groups).

\subsubsection{Vitamin B12}

Vitamin B12 (cobalamin) functions with folate in the one carbon metabolism, and also amino acid and odd chain fatty acid catabolism.
Dietary source is from animal products and deficiency leads to megaloblastic anaemia and nerve damage.

\subsubsection{Vitamin C}

Vitamin C is an antioxidant that is important for hydroxylation.
Deficiency leads to scurvy.

