\section{Diabetes}

Symptoms of diabetes include:
\begin{itemize}
\item fatigue
\item weight loss
\item intense thirst
\item frequent urination
\item hyperglycaemia
\item glucosuria
\item ketones
\end{itemize}

There are two types of diabetes:
\begin{description}
\item[Type I diabetes] is where the individual has an autoimmune disease of the pancreatic $\beta$ cells, where insulin is made.
Since the patient is responsive to insulin (just that they can't make it themselves), treatments include insulin injection.
This type is mainly genetic.
\item[Type II diabetes] is where the individual is not responding to the insulin (i.e. insulin resistance).
This means that insulin injection would not work on these patients and require alternative medication (e.g. hypoglycaemic drugs), if there is any (or just change lifestyle).
This type is both genetic and environmental.
\end{description}

When the blood glucose level is too low, it can cause serious problems (coma and death).
Although unless you have very high blood glucose for a long period of time, increase in blood glucose is not too serious.
However, it can cause many problems chronically (e.g. glycation of proteins, nreve impairment, retinopathy, etc.).
Glucose tolerance test can be used to test the effect of insulin in the patient.

\section{Insulin}

Insulin is a peptide hormone synthesised from the $\beta$ cells in the pancreas, and is secreted in response to increase blood glucose.
Since insulin is a peptide hormone, you cannot ingest it (as the GI would break it down).
Insulin acts on the whole body, but mainly the liver, muscle and adipose tissue to stop the fuel breakdown/mobilisation and stimulates the uptake/synthesis of fuel stores.

Consequences of no or very little insulin are:
\begin{description}
\item[Decreased glucose uptake and storage] due to no action by insulin.
\item[Increased glycogen mobilisation] as your cells are not taking up any glucose.
\item[Increased glucose synthesis] as the liver thinks that you're starving.
\item[Increased lipolysis] to try and increase the energy production.
\item[Increased ketogenesis] to compensate for the ``lack'' of glucose by your brain.
\item[Decreased removal of TAGs from blood] as no signal from insulin.
\item[Increased protein breakdown] due to the body thinking it's starving.
\end{description}

\begin{center}
% \includegraphics[width=0.8\textwidth]{insulin}
% \includegraphics[width=0.8\textwidth]{effectofinsulin}
\end{center}

















































