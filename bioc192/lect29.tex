\section{Amino Acids as Fuel}

Amino acids can be used as a fuel source, but its amino group must be removed in order to use it as a fuel.

\section{Deamination of Amino Acids}

Deamination of an amino acid generates a carbon skeleton and a free amino group.
The carbon skeleton can then be used as a fuel.
There are 2 ways of deaminating amino acids: direct deamination into solution by an enzyme, or by transamination of the amino group to a keto acid to form another amino acid.

Direct deamination of the amino acids are simple -- the amino group is removed and released into the solution by an enzyme catalysed reaction.
However, transamination process is a little bit complicated.
In transamination, the amino group of one amino acid is transferred to a keto acid to form a different amino acid, and the leftover carbon skeleton (a keto acid) can be used as a fuel.
For example, glutamate can transfer its amino acid onto a keto acid to form $\alpha$-ketoglutarate and a different amino acid.
These transamination reactions are catalysed by aminotransferases, and requires pyridoxal phosphate (derived from vit B6) to transfer the amino group from one amino acid to another.

\begin{center}
% \includegraphics[width=0.8\textwidth]{transamination}
% \includegraphics[width=\textwidth]{aminocac}
\end{center}

\subsection{Common Amino/Keto Acid Pairs}

There are common amino acid/keto acid pairs that you should know about.
These are:
\begin{reactions*}
Glutamate &<=> $\alpha$-ketoglutarate\\
Aspartate &<=> Oxaloacetate\\
Alanine &<=> Pyruvate
\end{reactions*}

Different amino acid carbon skeleton can feed into the citric acid cycle at different places, which may or may not produce energy.
For example, aspartate can form oxaloacetate, but going into the CAC here will not directly give you energy (it will provide more oxaloacetate for the cycle).

\section{Removal of Ammonium}

Free ammonium released from the direct deamination of an amino acid is toxic, and must be removed from the cell.
This can be done by adding the free ammonium to a keto acid ($\alpha$-ketoglutarate) to form an amino acid (glutamate).
This amino acid can then transaminate the amino group onto pyruvate to form alanine, which can then be tranported to the liver.
In the liver, alanine is transminated onto $\alpha$-ketoglutarate, and then directly deaminated from glutamate, and the liver cell can process this ammonium into urea.

\begin{center}
% \includegraphics[width=0.7\textwidth]{ammonium}
\end{center}

