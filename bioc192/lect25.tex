\section{Metabolic Pathways and ATP Synthesis}

This part of the lecture is about how the energy from food is released in metabolic pathways and how they are used to make ATP.
As a reminder:
\begin{description}
\item[Anabolism] is where you use energy from ATP hydrolysis to make stuff and keep you alive.
\item[Catabolism] is where you break down food molecules and use that energy to regenerate the used up ATP.
\end{description}

\subsection{Conservation of Energy}

Remember that you can't make or destroy energy -- energy must be converted from one form to another.
If $\Delta$G is less than 0, the reaction is favourable.
If $\Delta$G is equal to 0, the reaction is at equilibrium.
If $\Delta$G is greater than 0, the reaction is unfavourable (but favourable if it goes in the opposite direction).

\subsection{Energy Coupling}

In cells, we need some unfavourable reactions to occur in the pathway, as it could be essential for that pathway.
To do this, you can couple two reactions together so that the total $\Delta$G is less than 0 (and therefore favourable).

\begin{reactions*}
A &-> B\\
C &-> D\\
A + C &-> B + D\\
\end{reactions*}

Enzymes often couple these reactions to drive necessary, but unfavourable reactions.

\begin{center}
\ch{ATP + H2O <=> ADP + Pi + H+} ,  \bioGibbs[exponent=0']{30.5}
\end{center}

As you can see, ATP hydrolysis (and therefore release of energy from the hydrolysis) is favourable, and allows unfavourable reactions to occur if coupled together.
However, this also means that it requires that much energy to regenerate ATP.

\subsubsection{Coupling ATP Hydrolysis with Hexokinse}

\begin{center}
\ch{Glucose + Pi ->[Hexokinase] Glucose-6-phosphate + H2O}
\end{center}





























































