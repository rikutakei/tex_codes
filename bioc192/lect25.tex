\section{Metabolic Pathways and ATP Synthesis}

This part of the lecture is about how the energy from food is released in metabolic pathways and how they are used to make ATP.
As a reminder:
\begin{description}
\item[Anabolism] is where you use energy from ATP hydrolysis to make stuff and keep you alive.
\item[Catabolism] is where you break down food molecules and use that energy to regenerate the used up ATP.
\end{description}

\subsection{Conservation of Energy}

Remember that you can't make or destroy energy -- energy must be converted from one form to another.
If $\Delta$G is less than 0, the reaction is favourable.
If $\Delta$G is equal to 0, the reaction is at equilibrium.
If $\Delta$G is greater than 0, the reaction is unfavourable (but favourable if it goes in the opposite direction).

\subsection{Energy Coupling}

In cells, we need some unfavourable reactions to occur in the pathway, as it could be essential for that pathway.
To do this, you can couple two reactions together so that the total $\Delta$G is less than 0 (and therefore favourable).

\begin{reactions*}
A &-> B\\
C &-> D\\
A + C &-> B + D\\
\end{reactions*}

Enzymes often couple these reactions to drive necessary, but unfavourable reactions.

\begin{center}
\ch{ATP + H2O <=> ADP + Pi + H+} ,  \bioGibbs[exponent=0']{30.5}
\end{center}

As you can see, ATP hydrolysis (and therefore release of energy from the hydrolysis) is favourable, and allows unfavourable reactions to occur if coupled together.
However, this also means that it requires that much energy to regenerate ATP.

\subsubsection{Coupling ATP Hydrolysis with Hexokinse}

\begin{center}
    \ch{Glucose + Pi ->[Hexokinase] Glucose{-} 6 {-}phosphate + H2O}
\end{center}

The reaction above has a \bioGibbs[exponent=0']{+14}, which means that the reaction is unfavourable.
However, this reaction can occur by coupling this reaction with ATP hydrolysis to make the overall \bioGibbs[exponent=0']{-16}.

\subsection{Oxidising Fuels for Energy}

Both carbohydrates and fats release energy, but we don't want all of the energy to be released/burnt at once.
Therefore, our body requires a stepwise oxidation of carbohydrates and fats to capture the energy.

In your body, there are 2 main kinds of reactions involved in energy metabolism: redox reactions and phosphorylation reactions.

\subsubsection{Redox Reactions}

Redox reactions are where the fuel molecules get oxidised, and the energy from this oxidation is captured by the co-enzymes which gets reduced from this reaction.
Eventually, these reduced co-enzymes are used to make ATP.

\subsubsection{Phosphorylation Reactions}

Phosphorylation reactions are for making/generating ATP by directly or indirectly phosphorylating the ADP.
There are two kinds of phosphorylation reactions:
\begin{description}
    \item[Substrate level phosphorylation] This is where the energy to phosphorylate the ATP comes directly from the reactions that break down the molecules.
        Substrate level phosphorylation occurs in glycolysis and citric acid cycle.
    \item [Oxidative phosphorylation] This is where the energy to phosphorylate the ATP comes indirectly from the breakdown of the molecules, via co-enzymes.
        In the redox reactions, the co-enzymes get reduced, and the energy from these reduced co-enzymes can be fed into oxidative phosphorylation.
\end{description}

\subsection{Reducing Equivalents}

\begin{center}
    \ch{H = H+ + e-}
\end{center}
Biological reactions often involve the transfer of hydrogens (protons + electrons), instead of just electrons in purely chemical reactions.
The co-enzymes NAD and FAD interact with various enzymes to accept/donate H$^+$/e$^-$.
Enzymes that catalyse these reactions are often called dehydrogenase, as it removes hydrogen.

\subsection{Co-enzymes}

\subsubsection{NAD}

NAD (nicotinamide adenine dinucleotide) is made up of niacin (vitamin B3).
NAD accepts hydrogens and electrons in metabolic pathways such as glycolysis, fatty acid oxidation, and citric acid cycle.

\begin{center}
\ch{NAD+ + 2 H+ + 2 e- <=> NADH + H+}\\
\vspace{0.5cm}
\setatomsep{2em}
NAD$^+$ + \chemfig{R-C(-[2]H)(-[6]OH)-R'}
\ch{<=>} NADH + \chemfig{R-C(=[6]O)-R'}
+ H$^+$
\end{center}

As you can see, NAD$^+$ gains a hydrogen atom and two electrons(hence no charge after the reaction).

\subsubsection{FAD}

FAD (flavin adenine dinucleotide) is made from riboflavin (vitamin B2), and it accepts hydrogens in pathways such as fatty acid oxidation and citric acid cycle.
Flavin co-enzymes are tightly bound to the proteins with which they interact, whereas NAD is free in solution and dissociates after the reaction.

\begin{center}
    \ch{ FAD + 2 H+ + 2 e- <=> FADH2 }\\
    \vspace{0.5cm}
    \setatomsep{2em}
    FAD + \chemfig{R-C(-[2]H)(-[6]H)-C(-[2]H)(-[6]H)-R'}
    \ch{<=> FADH2} +
    \chemfig{R-C(-[2]H)=C(-[2]H)-R'}
\end{center}

\subsubsection{Co-enzyme A}

Co-enzyme A (CoA) are made from pantothenic acid (vitamin B5), and acts as a carrier of acyl groups (i.e. a fatty acid carrier).
There are 2 forms: CoASH form (free form) and AcCoA (acyl group attached) form.

\begin{center}
    \setatomsep{2em}
    \ch{H+ + CoASH +} \chemfig{CH_3-C(=[2]O)-\chemabove{O}{\scrm}}
    \ch{<<=> H2O + } \chemfig{CH_3-C(=[2]O)-SCoA}
\end{center}

