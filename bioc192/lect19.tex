\section{Energy Metabolism and ATP}

We require approximately 1kg of ATP per kg of body fat for the day.
Obviously, we can't carry that much ATP every day, and so we have to make it as we need it.
Anabolism is the use of energy to form/make macromolecules from building blocks (e.g. making proteins).
Metabolism is the opposite of anabolism, where the macromolecules are broken down into smaller constituent/component/building blocks to be oxidised to produce energy, or to be used in anabolic processes.

From the first law of thermodynamics, the total energy within a system is constant -- i.e. energy can neither be created or destroyed.
This means that the energy must be converted from one form to another.

\section{Energy Balance in the Body}

\begin{center}
E$_{intake}$ = E$_{expended}$ + E$_{stored}$\\
\vspace{0.5cm}
E$_{expended}$ = Basal metabolism + Activities
\end{center}

As you can see from the equation, all of the energy taken into the body must be either used up/expended, or stored.
Clearly, to lose weight, you must either decrease your intake, or increase your energy expenditure.
You can increase the energy expenditure by increasing your basal metabolism and/or the amount of exercise/activities.
(Note that small amount of energy is lost in faeces, urine and sweat.)
Weight gain is a problem in developed countries, as they tend to have higher percentage of obese people.

The rest of this lecture will be focusing on each aspect of the equation.

\subsection{Subtopic: Leptin Receptor}

Leptin deficient mice eats too much.
Leptin is a hormone that binds to the leptin receptor, which acts so that the host feels full, and therefore stops eating.
So, if a mouse is leptim deficient, then the mouse will not feel full, and therefore continues eating.

\section{Units of Energy}

Joule (J) is the unit of energy, defined as the \textbf{energy required to push against 1N of force for 1m}.
An older metric is the calories, where 1cal is the energy required to heat 1g of water from 14.5$^\circ$C to 15.5$^\circ$C.
1cal is equivalent to 4.184J.
Daily energy intakes and expenditure usually measured as MJ.

\section{Measuring Energy Intake}

\subsection{Bomb Calorimetry}

You can measure the energy contained in the food by burning the food and measuring the change in temperature of the water.

\begin{center}
\ch{C6H12O6 + 6 O2 -> 6 CO2 + 6 H2O + heat}\\
\ch{C16H32O2 + 23 O2 -> 16 CO2 + 16 H2O + heat}
\end{center}

Carbohydrates releases 2813kJ/mol (glucose), whereas fatty acids release 10024kJ/mol (palmitate).
Since fatty acids are more reduced, it can produce more energy from oxidation.
One thing to note is that not all energy in food is available for the body (e.g. cellulose and nitrogen, which are excreted out).
Therefore, these factors must be taken into account for calculations.

\section{Atwater Factor}

Atwater factors are used to measure how much energy each constituents have.
Fat has 38kJ/g, carbohydrates and proteins have 17kJ/g and ethanol has 29kJ/g.
These atwater factors can be used to calculate how much energy is contained in the food.

\section{Measuring Energy Expenditure}

Bomb calorimetry of a human would be good, but not feasible for obvious reasons.
There are other ways to measure energy expenditure.

\subsection{Direct Calorimetry}

This method relies on measuring heat output from individual in a whole body calorimeter.
This is good for determining the basal metabolic rate (BMR) -- i.e. at rest.
However, it is quite hard to measure the energy used in more physical exercises, due to space constraints and complexity of the calculation and machinery.

\subsection{Indirect Calorimetry}

This method measures the energy expenditure based on the oxygen consumption and carbon dioxide production, which is measured using a respirometer.
Since certain amount of energy is associated with every litre of oxygen consumed (20.9kJ/L), you are able to calculate how much energy was used.
This allows us to calculate the energy expenditure for a range of activities, and also allows the calculation of the respiration exchange ratio (RER).

\subsection{Respiration Exchange Ratio}

\begin{center}
\large{RER = $\frac{CO_2 produced}{O_2 consumed}$}
\end{center}

RER tells us about what kind of fuels are used -- carbohydrates or fat.
Note that the stoichiometry of the oxygen and carbon dioxide is different for carbohydrates and fat.
Since the stoichiometry of oxygen to carbon dioxide is 1:1 in carbohydrates, RER = 1, but this not the case for fat, and therefore have RER = 0.7.

\section{Basal Metabolism}

Basal metabolism is the energy required for the maintenance of life -- general metabolism, muscle concentration, etc.
This is usually defined as the energy expenditure at rest.
BMR is affected by:
\begin{itemize}
\item gender
\item age 
\item body size
\item body composition
\item genetics
\item hormonal status
\item stress levels
\item disease status
\item certain drugs
\end{itemize}

BMR can be increased by: athletic training, fever, late pregnancy, drugs (e.g. caffeine) and hyperthyroidism.
BMR can be decreased by: malnutrition, sleep, drugs (beta blockers) and hypothyroidism.

\subsection{Hyper and Hypothyroidism}

\begin{description}
\item [Hyperthyroidism] Increased metabolic rate due to excessive production of thyroid hormone, which acts to increase the effect of insulin, etc. Symptoms include heat production, anxiety, etc.
\item [Hypothyroidism] Decreased metabolic rate due to thyroid hormone deficiency (from iodine deficiency). Symptoms include coldness, depression, etc.
\end{description}
