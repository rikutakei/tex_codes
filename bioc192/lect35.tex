\section{Fuel Stores}

Energy must be made on demand at the cells.
Fuel stores are required since we cannot store ATP in our body -- ATP must be made in the cell at the time it's needed, at the rate it's needed by oxidising fuels.
We need fuel stores to:
\begin{itemize}
\item To maintain a supply of glucose between meals (especially for brain)
\item To provide immediate fuel for increased activity
\item For long periods when food intake may be inadequate (i.e. starvation)
\end{itemize}
In a fit, 70kg person, there are about 15kg of fat, 200 to 400g of glycogen, and almost no circulating glucose.

\section{Fat Storage}

TAGs are stored as fat droplets in adipose tissues.
Excess fat and carbs from diet can be converted to stored fat, and note that there is no limit to the amount of fat being stored in your body.
TAGs can be various form as well -- different length fatty acids, double bonds, etc.

\section{TAG Synthesis}

Fatty acids from chylomicrons and VLDL are transported to the adipose tissue, with the help of lipoprotein lipase.
The fatty acids are activated by forming acyl-CoA with free CoASH in the adipose tissue (note that this process uses 2 ATP worth of energy, as it makes AMP).
These activated fatty acids are then added to the glycerol backbone (glycerol 3 phosphate), which is made from DHAP in glycolysis (note that it's different to glyceraldehyde 3 phosphate, which goes through the rest of the glycolysis).
All of these processes are under the control of the hormone insulin.

\section{Mobilisation of TAGs}

To mobilise the TAGs, the fatty acids need to be cleaved off the TAGs by hormone sensitive lipase, which is stimulated by adrenaline and glucagon.
This allows the release of free fatty acids and glycerol.

\section{Glycogen}

Glycogen is a branched polysaccharide that is connected to one another via the $\alpha_{1-4}$ and $\alpha_{1-6}$ linkages.
Glycogen is mainly stored in liver and muscles as granules.

\section{Glycogen Synthesis}

Occurs mainly in liver and muscle, immediately after a meal.
Glycogen synthesis requires energy inputs as ATP/UTP.
Glucose is activated to form the high energy precursor UDP-glucose, and this process is done by the glycogen synthase enzyme.
Branching enzyme cleaves long glycogen branches and adds it to the growing branch.
Again, this process is stimulated by insulin.

\begin{center}
\ch{Glucose + ATP ->[Hexokinase] G6P + ADP <=>[Mutase] G1P + UTP -> UDP-Glucose + PPi}\\
\ch{Glycogen + UDP-Glucose ->[Glycogen\ synthase] Glycogen(n+1) + UDP}
\end{center}

Since UDP-glucose is a high energy intermediate, the addition of this to the glycogen is irreversible and requires a different enzyme to break it down.
As mentioned before, the branching enzyme cleaves a chain with more than 6 glucose molecules, and attaches it to the main branch.
This is stimulated by insulin.

\section{Mobilisation of Glycogen}

Glycogen is degraded by glycogenolysis.
Liver glycogen is released as glucose into the blood, to be used by other parts of the body.
Muscle glycogen releases the fuel for glycolysis within the muscle cells.

\section{Formation of Fatty Acids from Glucose}

Excess glucose carbon is converted to fatty acids, which occurs mainly in liver.
This is a complex and energy-requiring process, which involves the addition of acetyl-CoA onto the fatty acids (until 16C long)
These fatty acids are exported as TAGs (16C long) in VLDL, and stored las TAGs in the adipose tissue.
This is stimulated by insulin.

