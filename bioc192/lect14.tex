\section{Pharmacology and Toxicology}

Pharmacology is the study of therapeutic drugs and design of these drugs.
It is the study of how drugs work and how it is metabolised in the cell.

Toxicology is the study of poisoning/toxins.
It is the study of the harmful uses drugs in humans, animal and the environment.

\section{Drugs}

A drug is a chemical that changes the behaviour or function of an individual system, organ, tissue or invading organism.
In other words, a chemical that changes the behaviour of the cell.

\subsection{Sources of Drugs}

You can extract and purify drugs from plants and animals, but have some issues around contamination, etc.
Current sources of drugs are made synthetically, so there will be no contamination from purification.
Drug companies can screen for millions of chemicals, and choose the best one and study it (this process is very expensive).
Rational drug design is where you study the actual target and the disease so that you can make a rational decision about how to make or design a drug for that disease.
Another way to make drugs is to use computer to model what kind of drug will be the best to use.
Once identified, it can then be tested in the lab.

\subsection{Targets of Drugs}

Drugs can bind to a molecule and cause an effect on the cell -- the molecule it binds to will be its molecular target.
Understanding the pathways of the disease or the targets of the drug leads to better drug design.

\section{Agonists and Antagonists}

Agonists are the drugs or molecules that binds to a receptor and can produce a full biological response.
On the other hand, antagonists are drugs or molecules that binds to a receptor and prevents the response to occur (i.e. opposite of agonists).

\section{Receptors}

There are two types of receptors described in these lectures -- drug receptors and physiological receptors.

\subsection{Drug Receptor}

Drug receptor is ANYTHING that the drug binds to.
This can be a proteins, lipids, DNA, RNA, or whatever as long as the drug binds to it and have an effect on the cell through the molecule.

\subsection{Physiological Receptor}

\begin{center}
% \includegraphics[width=0.4\textwidth]{physrec1}
% \includegraphics[width=0.4\textwidth]{physrec2}
\end{center}

Physiological receptors are normal receptors that are present on the cell surface.
Most of these are transmembrane, meaning that it spans the whole membrane.
These receptors allow signalling to occur from the outside (extracellular) to the inside (intracellular) of the cell.
They also regulate cell function by responding to stimuli such as hormones, neurotransmitters or growth hormones and produce a signal for the cell to do something.
Most drugs activate or inhibit these kind of receptors.

There are four types of physiological receptors: ligand-gated/voltage-linked ion channels, G-protein coupled receptors (GPCR), kinase-linked receptors, and Nuclear receptors.

\begin{center}
% \includegraphics[width=1.0\textwidth]{fourreceptors}
\end{center}

\section{Ligand-Gated Ion Channels}

\begin{center}
% \includegraphics[width=0.7\textwidth]{ligandgated}
\end{center}

Ligand-gated ion channels are proteins present in the cell membrane.
This receptor has a receptor and an ion channel domains, and is made up from 5 subunits.
Binding of the ligand induces a conformational change in the protein, which opens up the channel and allow the ions to pass through.

Specific receptors have specific activating molecules, and only certain ions can pass through.
These receptors control the intracellular ion concentration which gives different charges to the cell.
The response times for these receptors are within milliseconds, which is ideal for fast response (e.g. in nerves).

Using nicotinic acetylcholine receptor as an example, normal substrate/agonist of this receptor is acetylcholine.
Another agonist is nicotine, and an antagonist is tubocurarine.
Tubocurarine is an irreversible inhibitor, so it binds to the receptor and won't release, hence you can eat the hunted animal.

\section{Voltage-Gated Ion Channels}

\begin{center}
% \includegraphics[width=1.0\textwidth]{voltagegated}
\end{center}

These receptors are made fromm four subunits, and opened in response to changes in voltage across the membrane (e.g. voltage change caused by ligand-gated ion channels).
The mechanism itself is very similar to the ligand-gated channels.
The drugs that can block these receptors literally blocks the ion channel and prevents the ions to move through the pores (e.g. tetrodotoxin).

\subsection*{What you should get out of this lecture}

\begin{itemize}
	\item Drug is a chemical that changes the behaviour of the cell
	\item Agonists are the drugs or molecules that causes the full biological response
	\item Antagonists are the drugs or molecules that prevents the full biological response (opposite of an agonist)
	\item Drug receptor is anything that a drug binds to (DNA/RNA/protein/peptide)
	\item Physiological receptor refers specifically to the  receptors located on the cell surface
	\item Know how the gated channels work
\end{itemize}

