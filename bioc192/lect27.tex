\section{Fatty Acid as a Fuel}

Fatty acids are the preferred fuel for most tissues in your body.
Fat is the primary energy reserve in your body, making up about 5 to 25\% of body weight in (mammals).
This is because fats are much lighter to store (as it requires no water for storage, and also are more reduced than carbs -- therefore more efficient energy store).

Note that excess glucose and fat are stored in your body as TAGs, and these TAGs are oxidised to provide energy.
However, this does not mean that you can just cut sugar and burn fat, as organs like brain requires glucose as their energy source, not fat.

\section{Delivery of Fatty Acids}

\subsection{From Adipose Tissue to Cells}

In adipose tissue, TAG is broken down into free fatty acids and glycerol by lipase.
FFAs diffuse into blood passively and binds to albumin for transport to other tissues.
Glycerol is transported to the liver.
FFAs get into the cells by diffusion or helped by transporters, and in the cell, FFA binds to the fatty acid binding protein (FABP).

\subsubsection{Fatty Acid Activation}

For fatty acid (or $\beta$) oxidation, the fatty acids must be transported into the mitochondria, as $\beta$-oxidation occurs in the mitochondria.
In order to transport the fatty acids into the mitochondria, FAs are activated in the cytoplasm by attaching CoA onto the FAs.
This generates fatty acyl-CoA, and the energy to do this comes from the hydrolysis of ATP.
Note that the ATP is hydrolysed into AMP, meaning that 2ATP worth of energy was used to accomplish this activation.

\subsection{From Cytoplasm to Mitochondria}

\begin{center}
\includegraphics[width=1.0\textwidth]{fatomt}
\end{center}

Oxidation of fatty acyl-CoA occurs in the mitochondria, but to do this, acyl-CoA must be transported into the mitochondria.
\begin{enumerate}
\item FA-CoA is transported into the intermembrane space via a carrier.
\item FA-CoA is converted into fatty acyl carnitine by carnitine acyl transferase I.
\item Fatty acyl carnitine is carried into the mitochondrial matrix.
\item Fatty acyl carnitine is converted back into FA-CoA by carnitine acyl transferase II.
\end{enumerate}

\subsubsection{Carnitine Supplements}

Note that carnitine is required for the transport of fat into the mitochondrial matrix.
There are supplements of carnitine to try and increase the carnitine concentration (so more fats are transported and burned).
However, you can already make carnitine in your body, and therefore require no extra carnitine (levels of carnitine will be compensated).

\section{$\beta$-Oxidation}

\begin{center}
\includegraphics[width=1.0\textwidth]{betaoxidation}
\end{center}

$\beta$-oxidation (or fatty acid oxidation) oxidises even number carbon fatty acids with no double bonds in it.
No ATP is directly synthesised from this, but energy is captured by co-enzymes (which will eventually be fed into electron transport chain).
The product of $\beta$-oxidation (acetyl-CoA) is fed into citric acid cycle for further oxidation and generation of energy.

$\beta$-oxidation is basically a cycle of oxidation, hydration, oxidation and cleavage, but you end with fatty acid with 2C shorter than what you began with.
Every cycle of $\beta$-oxidation generates FADH$_2$, NADH and an acetyl-CoA (except for the last round, where 2 acetyl-CoA are released).

\begin{center}
$No. of rounds = \frac{no.of C}{2}-1$
\end{center}

