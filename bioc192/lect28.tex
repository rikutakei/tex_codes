\section{Citric Acid Cycle}

Acetyl-CoA from glycolysis and $\beta$-oxidation is fed into the citric acid cycle, which occurs in the mitochondrial matrix.
All but one reaction (involving FADH$_2$) occurs in the mitochondrial matrix.
Citric acid cycle is a true cycle, as it starts and ends with the same molecule (which is oxaloacetate).

In CAC, there are 4 redox reactions (3NADH and 1 FADH$_2$) and one substrate level phophorylation (GTP/ATP).
Basically, there are two halves in the CAC -- carbon dioxide releasing half, and the rearranging the 4C molecule back to oxaloacetate half.

\subsection{First Half -- Release of CO$_2$}

\subsubsection{Condensation of Acetyl-CoA and Oxaloacetate}

This is the first step of the CAC, where the acetyl-CoA is attached to the oxaloacetate to make the 6C citrate.
Energy to do this comes from the hydrolysis CoA from the acetyl-CoA.
Since CAC is a cycle, the added 2C must be removed at some point of the cycle.

\subsubsection{Isomerisation of Citrate}

Citrate is converted into cis-Aconitate, and then into isocitrate, and these reactions are catalysed by an enzyme called aconitase.
This rearrangement makes the molecule more susceptible for decarboxylation steps.
Poisons such as fluoroacetate and sodium fluoroacetate (1080 poison) targets this step of the CAC.
Fluoroacetate is metabolised to fluorocitrate by binding to the oxaloacetate (similar to the first step of the CAC).
Fluorocitrate then binds to aconitase and inhibits the enzyme, which causes the inactivation of aconitase.
As a result, fluoroacetate decreases the amount of oxaloacetate available AND the activity of aconitase.

\subsubsection{First Oxidative Decarboxylation}

The first decarboxylation step is done in two steps -- oxidation of isocitrate, then the decarboxylation of oxalosuccinate.
The energy from the oxidation is conserved as NADH, and carbon dioxide is released from oxalosuccinate to form $\alpha$-ketoglutarate.

\subsubsection{Second Oxidative Decarboxylation}

This reaction is catalysed by $\alpha$-ketoglutarate dehydrogenase, and is similar to the pyruvate dehydrogenase reaction to add CoA onto pyruvate.
Again, the energy is conserved as NADH, and also releases carbon dioxide to yield a 4C molecule (succinyl-CoA).
This 4C molecule must be rearranged back into oxaloacetate to restart the cycle.

\subsection{Second Half -- Rearranging it back to Oxaloacetate}

\subsubsection{Succinyl-CoA to Succinate}

The conversion of succinyl-CoA to succinate is like the reverse of fatty acid activation.
Removal of CoA releases enough energy to synthesise GTP, which is then used to generate ATP (this reaction has a $\Delta$G close to 0, and therefore close to equilibrium).
This means that this reaction is the third substrate level phosphorylation since glycolysis.

\subsubsection{Succinate to Oxaloacetate}

This conversion is very similar to the $\beta$-oxidation pathway, where succinate is oxidised into fumarate, which is then hydrated into malate, which in turn gets oxidised into oxaloacetate.
The energy from the first oxidation reaction is conserved as FADH$_2$ and the energy from the second oxidation reaction is conserved as NADH.
Note that the first oxidation reaction (which uses succinate dehydrogenase) is shared between CAC and the electron transport chain.
This is in the inner mitochondrial membrane, and is due to the fact that FAD is tightly bound to enzyme, and therefore for ETC to use FADH$_2$, it makes sense to have the enzyme in the inner mitochondrial membrane where ETC occurs.

\begin{center}
\includegraphics[width=1.1\textwidth]{cac}
\end{center}

