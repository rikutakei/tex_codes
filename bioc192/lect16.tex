\section{G-protein Coupled Receptors (GPCR)}

G-protein Coupled Receptors (GPCRs) are transmembrane receptors that are associated with G-proteins, as the name suggests.
GPCRs acts/responds in the timescale of seconds -- slower than the ligand/voltage gated ion channels, but faster than the nuclear receptors.
GPCRs are important as it is the largest group of the receptors, and therefore an easy target for drugs.
Also, GPCRs are activated by diverse ligands.
About 30 to 50\% of the drugs target the GPCRs, but still only 10\% of the known GPCRs are targeted.

\section{Mechanism of GPCRs}

The ligand binds to the transmembrane receptor on the cell surface.
Before the ligand is attached to the receptor, the receptor is not associated with the G-protein.
Once the ligand binds to the receptor, the conformation of the receptor changes, which allows the G-protein to interact with the receptor.
Interaction with the receptor causes the G-protein to change conformation as well, which increase or decrease the activity of a particular enzyme.
Depending on the activity of the enzyme, the amount of second messenger molecule produced by the enzyme is altered, and therefore causes some cellular response.

\section{Dopamine Receptor}

Dopamine receptor is an example of GPCR.
When the ligand (dopamine) binds to the receptor, it causes a conformational change in the G-protein which decrease the enzyme activity and therefore decreases the amount of second messenger (cAMP) produced.

\subsection{Parkinson's Disease}

Parkinson's disease is caused by decreased level of dopamine, due to the damage in the neural cells that produce dopamine.
Parkinson's can be treated by using levodopa -- a prodrug that is processed into dopamine.
The problem with levodopa is that it is hard to balance/control the effect of levodopa, and leads to psychological effects (such as hallucination/psychosis) and motor complications.

\subsection{Psychosis}

Psychosis occurs when there is too much dopamine around, and over activates the dopamine receptor.
In this case, antipsychotics (antagonists of dopamine receptor) is used to prevent the dopamine to have too much effect on the receptor.
This means that the level of dopamine is stall high, but the effect is inhibited by using antagonists.
Again, however, if you use too much of antipsychotics, you get a Parkinson's like syptoms, signifying the fact that it's very hard to balance the effects of these drugs.
































