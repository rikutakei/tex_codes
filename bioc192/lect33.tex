\section{Photosynthesis}

Photosynthesis is the process in which the energy of sunlight is used by organisms (especially green plants) to synthesise carbohydrates from carbon dioxide and water.
The overall simplified equation of this reaction is:
\ch{6 CO2 + 6 H2O ->[Light{} energy] C6H12O6 + 6 O2}

The energy stored in the carbohydrate is then available to the biosphere through respiration (i.e. carbs used to respire).

\section{History of Photosynthesis}
\begin{description}
\item[Joseph Priestly] Found that plants made ``air''.
When the plant was placed with a candle and/or a mouse, both remained alive.
However, when there was no plant in the jar, the mouse died.
\item[Jan Ingenhousz] Showed that light was required for photosynthesis and suggested the equation: \ch{CO2 + H2O ->[light] (CH2O)n + O2}.
\item[Cornels van Neil] Hypothesised that the oxygen came from the water (or its equivalent part), not the carbon dioxide.
\item[Robin Hill] Experimentally proved that van Neil's hypothesis was right by doing an experiment on isolated thylakoids.
He removed carbon dioxide from the system and added Fe$^{3+}$.
This caused a reaction water and produced oxygen (Hill reaction).
\item[Sam Ruben and Martin Kamen] Used radio-labelled oxygen to track where the oxygen was coming from, and therefore provided direct evidence that the oxygen was coming from the water, not carbon dioxide.
\item[Calvin and Benson] Used 14C atom to track the path of the carbon and proposed the famous Calvin-Benson cycle.
In this cycle, 3 Ribulose-1,5-bisphosphate (RuBP) is carboxylated to make 6 3-phosphoglycerate (PGA or 3PG), which is catalysed by rubisco (the most abundant protein in the world).
3PG are then rearranged into 1,3-BPG, then into glyceraldehyde-3-phosphate (PGAL or G3P).
One of these G3P is a net gain, where as other 5 are use to regenerate the starting RuBP.

\begin{center}
% \includegraphics[width=0.8\textwidth]{cbcycle}
\end{center}

\item[Sam Wildman] Discovered the enzyme Rubisco, which catalyzes the first reaction of the Calvin-Benson cycle.
\end{description}

\section{Mechanism of Photosynthesis}

For the first part of the Calvin-Benson cycle to occur, it needs ATP and NADPH -- where do they come from?
The Calvin-Benson cycle is ``coupled'' to the photosystems that utilises light to split water (the light reaction).
There are different types of antenna pigments in different photosynthetic organisms.
This allows them to exploit different niches in the electromagentic spectrum.

\subsection{Antenna and Reaction Center Concept}

Emerson and Arnold came up with the antenna and reaction center concept.
They found that they required 10 photons to split a single oxygen molecule, and there were about 2500 cholorphylls present to split the oxygen.
The idea was that the antenna complex (made of multiple pigment molecules) transferred the energy from the light to the reaction center, where the electron is transferred to molecules.

\subsection{Z Scheme}

\begin{center}
% \includegraphics[width=0.8\textwidth]{lightreaction}
\end{center}

The Z scheme was proposed by Robin Hill (the same guy who did the Hill reaction), where the light energy was transferred to specialised pigments in photosystems I or II (note that different wavelengths for different photosystems).
Once the energy is transferred to the photosystem, it can use this to transfer electrons into the electron transport chain (and split water in the process, if it's photosystem II).
Eventually, this electron is accepted by NADP$^+$ and forms NADPH, which is used in the first set of reactions in Calvin-Benson cycle.
The electrochemical gradient formed from the splitting of water and the electron transport chain is then used to make ATP to drive the first set of reactions.

\subsection{Chemiosmosis Theory}

The first evidence of this theory proposed by Peter Mitchel was from an experiment with chloroplasts.
Isolated thylakoids were added to a pH 4 solution and equilibrised.
The thylakoids were then transferred into a pH 8 solution and observed in dark (to prevent any ATP production via light reaction).
This showed that the pH gradient was enough to make ATP.

