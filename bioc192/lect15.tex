\section{Nuclear Receptor}

Nuclear receptors, unlike the transmembrane voltage/ligand-gated channels, reside in the cytosol of the cell (i.e. within the cell).
This is partly because the nuclear receptors act in the nucleus, where it interacts with the DNA for gene transcription.
These nuclear receptors binds and interacts with the DNA at the response element for that receptor, which requires multiple regulation mechanisms.
The binding of the nuclear receptor-ligand complex to the response element causes transcription/gene activation.
Since the effect of the cellular response is observable only after the protein is made, the response time is quite slow -- takes hours to days for the response.

\section{Mechanism of Nuclear Receptor Signalling}

For a ligand to enter the cell, the ligand must be lipophilic/hydrophobic, otherwise it will not be able to enter the cell through the membrane.
Normally, the receptor is bound to a heat shock protein (Hsp; a chaperone), that prevents the receptor from binding to the DNA or other receptors.
When the ligand binds to the receptor, the Hsp is displaced/removed, and the receptor dimerises (i.e. binds to another receptor).
The receptor cannot bind to the DNA unless it is dimerised.
Once bound to the DNA, the dimer recruits/associates with many other proteins to activate the gene (e.g. RNA polymerase) and transcribes the gene for cellular response.

\subsection{Regulation of Nuclear Receptor}

There are multiple regulatory steps in this mechanism to prevent unwanted signal and cell response.
Firstly, the receptor is located inside the membrane, meaning that only the lipophilic/hydrophobic ligands can access/activate the receptor.
The Hsp that is bound to the receptor is also important, because without them, the receptor may be able to bind to the DNA cause unwanted gene activation.
Dimerisation is another level of regulation -- it requires the receptor to be dimerised, or else it's not going to activate the gene.

\section{Structure of Nuclear Receptor}

\begin{center}
% \includegraphics[width=0.7\textwidth]{nuclearreceptor}
\end{center}

There are three parts to the receptor.
The first part is the ligand binding site (steroid hormones).
Second part is the DNA binding domain, which is the site where the receptor binds to the DNA.
This part contains the zinc finger motif for DNA binding.
The last part is the transcription regulatory domain, where the transcription factors and other machinery binds to for gene activation.

\section{What Genes are Activated by Nuclear Receptor}

The genes activated by the receptor depends on the tissue site the receptor resides in, what cell type it's in, and what kind of ligand/drug binds to the receptor.
This means that the receptor can activate multiple different genes, and therefore have different cellular response depending on the ligand, tissue site, etc. (i.e. it's not a one-gene-to-one-receptor thing).
For example, estrogen receptor have different cellular response depending on the tissue site.

\section{Nuclear Receptor and Cancer}

Cancer is a very common disease and have high death rates.
Cancers are the cells that divide uncontrollably and proliferate while avoiding cell death.
The growth of the cancer cells are usually driven by the hormones, and therefore driven by the receptors of these hormones.

\subsection{Breast Cancer}

In some breast cancer, the cause is from over activation of estrogen receptor.
In pharmacology terms, estrogen is the agonist of the ER, and over dosage of this can lead to cancer.
Tamoxifen, on the other hand, is an antagonist of ER and prevents the cellular response from occurring.
However, although tamoxifen reduces the growth of breast cancer, it has an agonist side effects in bone (strengthening of bone) and in uterus (increased risk of endometrial cancer).

\subsection{Androgen Receptor and Prostate Cancer}

Testosterone is formed from the testis and adrenal gland (mostly testis).
Testosterone is the ligand for the androgen receptor.

Prostate cancer is (at least initially) by over activation of the androgen receptor by testosterone.
One way to treat it is to remove the testicles.
Another way is to use drugs that reduces the androgen level.
Bicalutamide is one such drug -- it is an androgen receptor antagonist and it also accelerates the AR degradation.
However, it also has a lot of side effects.

\section{Effect of Hormones on Living Organisms}

In a waterway with low levels of estrogen, some male fish developed in to female-like fish.
This was probably due to the chronic exposure to low level of estrogen in the water.
This raises an important question where low levels of hormone may affect the development of humans, if exposed chronically.
Although there has only been speculations, this theory could be true.

\subsection*{What you should get out of this lecture}

\begin{itemize}
	\item Nuclear receptors are located inside the cell, and because of this, the ligands are usually hydrophobic
	\item Know the basic structure of nuclear receptors
	\item Know the mechanism in which the nuclear recpeptor acts
\end{itemize}
