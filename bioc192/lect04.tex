\section{Myoglobin}

Myoglobin is the monomeric form of haemoglobin, and it resides in the tissue/muscles for O$_2$ storage.
Myoglobin consists of a globin protein that consists of 8 $\alpha$-helices (numbered for clarity), with a haem group in the middle.
Myoglobin is able to reversibly bind O$_2$, and have high affinity for O$_2$ as well.

\section{Haem (Heme)}

Haem consists of four pyrrole rings that are linked together via methane bridges, with the ferrous (Fe$^{2+}$) attached at the center of the molecule.
This Fe$^{2+}$ is slightly below the plane of the porphyrin ring when O$_2$ is not bound, and is shifted into the plane when it binds (causes conformational change).

The heme is attached to the myoglobin via the His F8 (histidine on the helix F at position 8), and His E7 is on the opposite side to influence the O$_2$ binding affinity.
The His E7 prevents the strong binding of O$_2$ to the heme (right angle) by kinking the angle at which the O$_2$ binds to the heme.
This allows weaker and therefore reversible binding of the O$_2$ molecule to the myoglobin.

Note that cyanide binds to the Fe$^{3+}$ form, not the Fe$^{2+}$ form.
Also note that cyanide's main inhibitory effect is at the cytochrome Fe$^{3+}$, and affects the electron transport system.

\section{Haemoglobin}

Haemoglobin is the tetrameric (4 subunit) form of myoglobin that consists of two $\alpha$ and two $\beta$ subunits, and it is a blood O$_2$ transporter.
Since haemoglobin is a tetramer, it has 4 haem group in total (one on each subunit), and therefore able to carry 4 O$_2$ molecules.
Slight conformational change occurs when the O$_2$ binds to the haemoglobin molecule, which will become important in explaining the cooperative effect of the subunits.

\section{O$_2$ Binding Property}

\subsection{Myoglobin}

Myoglobin resides in muscles and tissues, and have high affinity for O$_2$.
This is shown in the O$_2$ saturation curve.
At low partial pressure of O$_2$ (pO$_2$), myoglobin is mostly saturated with O$_2$ (i.e. O$_2$ is bound to it).
This means that myoglobin only releases the O$_2$ molecule when the pO$_2$ is very low.

\subsection{Haemoglobin}

In contrast, haemoglobin's O$_2$ binding affinity is not that high compared with myoglobin.
As you can see from the O$_2$ saturation curve, it requires quite a high pO$_2$ for haemoglobin to be saturated with O$_2$.
This makes sense as haemoglobin has to bind to O$_2$ at the lungs where pO$_2$ is relatively high, but also has to release the O$_2$ molecules at the tissue/muscle, where the pO$_2$ is low.
This allows haemoglobin to be an effective O$_2$ carrier.

One thing to note here from the curve is that the shape of the curve for haemoglobin is sigmoidal (S-shape).
This indicates that haemoglobin is allosteric (or cooperative), where the subunits assist one another to facilitate the binding/release of O$_2$ molecules.
Of course, there are other factors that contribute to the allosteric binding/release of O$_2$ molecules.

\section{Sequential and Concerted Model of \-Allosterism}

When O$_2$ binds to a subunit of haemoglobin, it causes a slight conformational change to the subunit.
This conformational change assists the binding of O$_2$ molecules to the other subunits by cooperativity/allosterism.
For clarity, the tense (T) state is the state of the subunit when O$_2$ is not bound (low O$_2$ affinity), and the relaxed (R) state is the state of the subunit when O$_2$ is bound (high O$_2$ affinity).

There are two models, sequential and concerted models, of allosterism.
It is still unknown which model is correct, as there are many observations supporting both models.

\subsection{Sequential Model}

Sequential model is when one O$_2$ molecule binding to a single subunit causes the conformational change in the subunit it binds, and also the subunits that are adjacent to it as well.
This conformational change makes the other subunits easier to bind to O$_2$ molecules, which will also change conformation of itself and the others.

With respect to the R/T states, binding of O$_2$ molecule causes a single subunit to be in R state, and also changes the conformation of the adjacent subunits so that it is easier to bind to O$_2$ molecules, therefore easier for the subunits to turn into the R state.

\textbf{In sequential model, binding of one substrate changes the conformation of other subunits, which makes it easier for the other subunits to bind the substrate}

\subsection{Concerted Model}

In concerted model, all of the subunits are either in the T state, or the R state, and the states are at equilibrium (i.e. no `in between' state).
The O$_2$ molecules can bind to either state at any time.
When there are more substrate (i.e. O$_2$) is bound to haemoglobin, the greater the chance for the subunits to be in the R state, which means that the equilibrium is favoured towards the R state when there are more O$_2$ molecules bound to haemoglobin.

\textbf{ In concerted model, the T and R states are at equilibrium and substrates can bind at any time to any conformational state. The binding of substrate causes the equilibrium to shift to the R state and the substrates are more likely to bind to the protein.}

