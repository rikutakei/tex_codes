\section{Myoglobin}

Myoglobin is the monomeric form of haemoglobin, and it resides in the tissue/muscles for O$_2$ storage.
Myoglobin consists of a globin protein that consists of 8 $\alpha$-helices (numbered for clarity), with a haem group in the middle.
Myoglobin is able to reversibly bind O$_2$, and have high affinity for O$_2$ as well.

\section{Haem (Heme)}

Haem consists of four pyrrole rings that are linked together via methane bridges, with the ferrous (Fe$^{2+}$) attached at the center of the molecule.
This Fe$^{2+}$ is slightly below the plane of the porphyrin ring when O$_2$ is not bound, and is shifted into the plane when it binds (causes conformational change).

The heme is attached to the myoglobin via the His F8 (histidine on the helix F at position 8), and His E7 is on the opposite side to influence the O$_2$ binding affinity.
The His E7 prevents the strong binding of O$_2$ to the heme by kinking the angle at which the O$_2$ binds to the heme.
This allows weaker and therefore reversible binding of the O$_2$ molecule to the myoglobin.

Note that cyanide binds to the Fe$^{3+}$ form, not the Fe$^{2+}$ form.
Cyanide's main inhibitory effect is at the cytochrome Fe$^{3+}$, and affects the electron transport system.






