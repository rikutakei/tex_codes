\section{Supersecondary structure}

Secondary structures such as $\alpha$-helix, $\beta$-sheets, and turns combine together to form a specific supersecondary structures, or `motifs'.

\section{Protein Domains}

Several supersecondary structures then come together to form protein domains (tertiary structure), which often possess a specific biological function.
These domains have hydrophobic core and hydrophilic surface, and is stabilised by non-covalent interactions (but some proteins may have covalent disulfide bonds to stabilise it).
Note: Hydrophobic interactions drive the protein folding process.
Large proteins are often multi-domained, whereas smaller proteins can be made up of a single domain.

\section{Protein Families}

Proteins can be grouped based on the tertiary structure (e.g. $\alpha$/$\beta$-barrel).
There are a lot of protein families.

\section{Quaternary Structure}

Quaternary structure refers to the non-covalent interactions between different domains, which forms a larger protein structure.
The domains are called `subunits', and these assemblies are called dimer (2 subunits), trimers (3 subunits), etc.

\section{Protein Folding Pathway}

Protein folding is directed predominantly by the hydrophobic forces from the non-polar residues.
This is NOT a random process.
Since the proteins are submerged in water, the non-polar residues try to stay away from water, and therefore aggregate together to form secondary structures.
This initial folding process is referred to nucleation.
The hydrophobic regions of the secondary structure comes together to form a larger structure, and stabilised by small non-covalent interactions.

\vspace{0.5cm}

\noindent
The process of protein folding is summarised below:
\begin{enumerate}
\item Short, small secondary structures are formed, driven by hydrophobic forces (nucleation)
\item Nuclei come together to form larger domains
\item Larger domains come together
\item Small conformational changes/adjustments occur to give compact native structure
\end{enumerate}

\section{Stabilisation of Protein Folding}

Proteins are stabilised by non-covalent interactions, which are weak individually, but strong when there are a lot of them.
Covalent bonds such as disulfide bonds can stabilise the protein as well.

\section{Chaperones and Chaperonin}

Some proteins require additional accessory proteins to assist them in folding the proteins properly, or to fold it faster.
If the proteins are not folded properly, it can lead to serious diseases like Alzheimer's and prion.

