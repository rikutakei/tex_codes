\section{Enzyme Inhibitor}

Inhibitor is a small molecule that can bind to the enzyme and reduce its catalytic activity.
These inhibitors are important in various ways: natural inhibitors regulate the metabolism in the cell; many drugs, poisons and toxins target enzymes; and is also used for studying metabolic pathways and mechanisms of an enzyme.
There are two types of enzyme inhibitors: reversible and irreversible inhibitors.

\section{Irreversible Inhibitors}

Irreversible inhibitors are inhibitors that covalently binds to the enzyme, to a specific amino acid side chain.
This means that the inhibitor permanently inactivates an enzyme, as it cannot be unbound.
Examples of irreversible inhibitors include penicillin (targets glycopeptide transpeptidase in bacterial cell wall) and organophosphates (inhibitor of acetylcholinesterase).

\section{Reversible Inhibitors}

There are two types of reversible inhibitors: competitive and non-competitive inhibitors.
Both of these inhibitors bind to the enzyme non-covalently, and can be `knocked off'.
Some inhibitors can be knocked off more easily than the others, depending on how tight the inhibitor binds to the enzyme.

\begin{center}
\ch{ E + I <=> EI }\\

\vspace{0.5cm}

\ch{K$_i$ = $\frac{[E][I]}{[EI]}$}
\end{center}

K$_i$ is the dissociation or inhibition constant -- just like the K$_M$ with normal substrate.
The lower the K$_i$, the stronger the inhibitor binds to the enzyme.
K$_i$ \textless{} $10^{-9}$ mol L$^{-1}$ is strong binding.

\subsection{Competitive Reversible Inhibitor}

\begin{center}
\includegraphics[width=0.6\textwidth]{compinh}
\end{center}

These inhibitors bind to the enzyme at the active site, and competes with the substrate molecule.
This means that (usually) the inhibitor will have similar structure as the substrate, as it can bind to the active site.

\begin{center}
\textbf{Competitive inhibitors increases the K$_M$, but does not change the V$_{max}$ of the enzyme.}
\end{center}

This means that the apparent affinity of an enzyme for the substrate is reduced (as the K$_M$ is increased).
The V$_{max}$ is not affected, as the binding of the inhibitor is NOT going to affect the catalytic activity of the enzyme -- the enzyme can still reach the V$_{max}$, it just needs more substrate to out-compete the inhibitor (hence the increase in K$_M$).
(The binding of the inhibitor is not going to affect how fast the reaction proceeds, once the real substrate binds to the enzyme).

\subsection{Non-competitive Reversible Inhibitor}

\begin{center}
\includegraphics[width=0.6\textwidth]{noncompinh}
\end{center}

These inhibitors bind to elsewhere to the active site of the enzymes and causes a conformational change, and therefore affects the efficiency of the catalysis.

\begin{center}
\textbf{Non-competitive inhibitors decreases the V$_{max}$, but does not change the K$_M$ of the enzyme.}
\end{center}

When the inhibitor binds to the enzyme, it does not occupy the active site where the substrate binds to.
This means that the active site is free for the substrate to bind -- it's just that the reaction does not proceed as efficient as without the inhibitor bound to it.
Hence the decreased V$_{max}$, but unchanged K$_M$.
(Another way to think about it -- V$_{max}$ is proportional to the total enzyme concentration and the k$_{cat}$, so if k$_{cat}$ decreases, then V$_{max}$ will also decrease).

\section{Methods of Enzyme Regulation}

\subsection{Allosteric Enzyme Regulation}

\begin{center}
\includegraphics[width=0.6\textwidth]{allosterism}
\end{center}

Allosteric enzymes generally display co-operativity and usually multi-subunit.
The enzyme's reaction curve shows a sigmoidal curve.
Activity of allosteric enzymes can be regulated by binding molecules (effectors) away from the active site, which can activate or inhibit the enzyme.
Regulatory effect is achieved by conformational changes caused by the effectors binding to the enzyme.
Note that the inhibitors shift the curve to the right, whereas the activator will shift the curve to the left.

\subsection{Covalent Modification of Enzymes}

In this lecture, only phosphorylation was mentioned.
Phosphorylation and de-phosphorylation is the most common, and is catalysed by kinases and phosphatases, respectively.
Phosphorylation state of an enzyme determines whether the enzyme is `on' or `off'.
However, phosphorylation does not necessarily mean that the enzyme is on -- it really does depend on the enzyme you're studying.

\subsection{Activation of Enzymes by Cleavage}

Some enzymes are synthesised as zymogens -- the inactive form of the enzyme.
The enzyme is only active once the enzyme is cleaved at a certain amino acid residue.
Many of the enzymes involved in the digestive processes and blood clotting proteins are made as zymogens.

\section{Isoenzymes/Isozymes}

Multi-subunit enzymes may exist in multiple forms in an organism -- these enzymes are called isoenzymes.
Isoenzymes differ in physical, kinetic and immunological properties, but all of them catalyse the same reaction.
Isoenzymes are formed from two different genes that make slightly different subunit for the enzyme.
Isoenzyme is a form of transcriptional regulation of cellular activity, as some cells in your body has to catalyse the reaction differently to another part of your body.
