\section{The Michaelis-Menten Equation}

Using V$_{max}$ and K$_M$, we can define the Michaelis-Menten equation:

\begin{center}
\large{V = $\frac{V_{max} [S]}{K_M + [S]}$}
\end{center}

Few assumptions must be made in order to derive this equation:
\begin{center}
\ch{E + S <=> [k_1][k_{-1}] ES -> [k_2][slow] E + P}
\end{center}
\begin{enumerate}
\item ES formation and breakdown is in rapid equilibrium, i.e. the ES concentration does not change over time ({$\frac{d[ES]}{dt}$} = 0).
\item Enzyme is saturated with substrate, therefore it is at V$_{max}$.
\item Initial rate is used, meaning that the substrate concentration does not change significantly.
\item Initial rate is used, meaning that there is no product present.
\end{enumerate}

Michaelis-Menten equation is a simple model of an enzyme catalysed reaction based on the assumptions stated above, and is applicable to most enzymes.

\section{Lineweaver-Burk Plot}

\begin{center}
% \includegraphics[width=0.8\textwidth]{lineweaverburk}
\end{center}

Since the V vs [S] curve is hyperbolic, you can only estimate the V$_{max}$ from it, and therefore you can only estimate the K$_M$ value.
In order to measure the V$_{max}$ and K$_M$, you can take the double reciprocal of the V vs [S] curve (i.e. take the reciprocal of both V and S).
This gives us the Lineweaver-Burk plot, which you can measure the V$_{max}$ and K$_M$ from the y- and x-intercept, respectively.

\section{Significance of K$_M$}

\begin{center}
\ch{E + S <=> [k_1][k_{-1}] ES -> [k_2][slow] E + P}
\end{center}

K$_M$ characterises a single enzyme-substrate pair.
K$_M$ is independent of the substrate concentration, but is dependent on the substrate, meaning that K$_M$ is different for different substrate.

K$_M$ is defined as the ratio of the breakdown of the ES complex and the formation of the ES complex:

\begin{center}
\large{K$_M$ = $\frac{breakdown[ES]}{formation[ES]}$ = $\frac{k_{-1} + k_2}{k_1}$ = $\frac{k_{-1}}{k_1}$}
\end{center}

Usually, k$_2$ is much slower than k$_{-1}$, so we can ignore k$_2$.
K$_M$ is also called the ES dissociation constant, and its unit is the same as the concentration.

Low K$_M$ indicates that k$_{-1}$ is low and k$_1$ is high, and therefore it has a high affinity between the enzyme and the substrate, and binds to one another tightly, forming a strong ES complex.
High K$_M$ indicates that k$_{-1}$ is high and k$_1$ is low, and therefore it has a low affinity between the enzyme and the substrate, and binds to one another weakly, forming a loose/weak ES complex.

Therefore, you can use K$_M$ as an indicator of substrate preference of an enzyme.

\section{Physiological Significance of K$_M$}

In the cell, the substrate concentration is often below the K$_M$ for a particular enzyme-substrate reaction.
The physiological substrate concentration lies on the either side of the K$_M$, on the linear part of the V vs S curve.
This means that as the substrate concentration increases in the cell, you get a relative increase in the rate of reaction as well, and vice versa.
Since the physiological substrate concentration is around K$_M$, the enzyme is able to tolerate both increase and decrease in the cell's substrate concentration.

When you think about it, if the physiological substrate concentration was at the point where enzyme is working at its V$_{max}$, the enzyme will not be able to cope with the increase in substrate concentration -- it is already working full-on, and cannot process the extra substrate.
This can lead to the accumulation of the substrate in the cell, and could have some effects on the cell.

\section{V$_{max}$ and k$_{cat}$}

\begin{center}
\ch{E + S <=> [k_1][k_{-1}] ES -> [k_2][slow] E + P}
\end{center}

\begin{enumerate}
\item k$_2$ is the rate-limiting step of the reaction
\item The rate of the reaction is proportional to the total enzyme concentration ([E$_{total}$]).
\end{enumerate}

This gives us the definition of k$_{cat}$ and V$_{max}$, relative to one another:

\begin{center}
V$_{max}$ = k$_2$ + [E$_{total}$] \ch{->} k$_{cat}$ = $\frac{[E_{total}]}{V_{max}}$
\end{center}

k$_{cat}$ is the turnover number, or the catalytic rate constant of an enzyme.
The unit is in s$^{-1}$.
\textbf{The number of moles of substrate converted into product, per mole of enzyme per second.}

\section{k$_{cat}$, K$_M$ and Catalytic Efiiciency}

\begin{center}
Catalytic Efficiency = $\frac{k_{cat}}{K_M}$
\end{center}

The ratio of k$_{cat}$ and K$_M$ is the measure of the enzyme efficiency -- how quickly a product is formed relative to how well the substrate binds.
The catalytic efficiency will therefore have the highest value when k$_{cat}$ is high (high turnover), and low K$_M$ (high binding affinity).
