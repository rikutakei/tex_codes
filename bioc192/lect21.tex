\section{Protein Digestion and Absorption}

Proteins can be used for energy, but it is more for regenerating the depleted amino acids.
It is also important for supplying the body with essential amino acids.
Proteins are the major source of nitrogen for purines, pyrimidines and haem.
The carbon skeleton of the amino acids can be used as fuel/energy production, where the amino group (i.e. nitrogen) is converted into urea and excreted.
Decrease/malnutrition of certain amino acid can lead to Kwashiokhor.

\section{Essential Amino Acids}

We cannot synthesise all 20 amino acids -- require 8 essential amino acids from the diet.
These amino acids are: L, K, T, Y, I, M, F and V.

\section{Regulation of Protein Digestion}

Obviously, we don't want to ``eat ourselves'', so the process is highly regulated by polypeptide hormones.
These hormones are: gastrin (stomach), secretin (duodenum), and cholecystokinin (duodenum).

\section{Enzymes Involved in Protein Digestion}

Before we move onto how proteins are digested, we need to know what kind of enzymes are involved.
The enzymes present are: pepsin (stomach), trypsin, chymotrypsin and carboxypeptidase (3 from pancreas), and aminopeptidase and dipeptidase (2 from small intestine).
All of these enzymes act in the small intestine, except for pepsin.
Note that the protease specificity is determined by the adjacent amino acid side chains of the peptide chain it is breaking.
For example, pepsin, trypsin and chymotrypsin cleaves the polypeptide after aromatic, positively charged and aromatic side chains, respectively.

\section{Protein Digestion}

Digestion of protein involves the hydrolyses of specific peptide bonds, performed by several different proteases.
All of these proteases are secreted as inactive zymogens to prevent them from digesting our own cells, and they are all activated by cleavage of peptides from their structure.

There are two stages of protein digestion:
\begin{enumerate}
\item Endopeptidases attack the peptide bonds within the peptide chain. Pepsin, trypsin and chymotrypsin are all endopeptidases.
\item Exopeptidases attack the peptide bonds at the end of the peptide chain (i.e. from C-term or N-term -- hence amino or carboxypeptidase).
\end{enumerate}

\subsection{Pepsinogen Activation}

Pepsinogen (due to its structure) have its catalytic active site inhibited.
H$^+$ (low pH) changes the conformation of the pepsinogen and allows it to auto-cleave the peptide bond and activate itself.
The activated pepsin can then activate other pepsinogens.
Note that this process is all triggered by the ingestion of food, and the hormones associated with it.

\subsection{Small Intestine Peptidase Activation}

All 3 peptidases (trypsin, chymotrypsin and carboxypeptidase) are secreted in the inactive zymogen forms -- trypsinogen, chymotrypsinogen and procarboxypeptidases.
First, trypsinogen is activated by the membrane bound enterokinase that adds the phosphate group to it.
This causes a conformational change in the trypsinogen structure which allows it to auto-cleave and auto-activate itself.
The activated trypsin activates the chymotrypsinogen and procarboxypeptidase.

As a result of the activation of all of these proteases (including pepsin), dietary proteins are digested into single amino acids, dipeptides and tripeptides.
Di and tripeptidases then digest the di and tripeptides into single amino acid.

\section{Protein Absorption}

\subsection{Amino Acid Transport}

There are 5 main amino acids transporters to transport the amino acids across the luminal membrane.
These are: Neutral, basic, acidic, other (taurine), and Glycine, proline and hydroxyproline.
All of these amino acids are transported in a similar way to the glucose transport -- co-transport with Na.

\subsection{Peptide Absorption}

Very little absorption occurs for peptides that are longer than 4 amino acids.
Di and tripeptides are absorbed in the small intestine via co-transport with H$^+$ ions (PepT1).
These di/tripeptides are then digested in the cell and exported out into the blood.

\subsection{Absorption of Intact Proteins}

This only occurs in few circumstances -- e.g. in newborn for immunoglobulin absorption.

\section{Diseases of Inappropriate Enzyme Activation}

This is associated with inappropriate enzyme action (zymogen activation).
This can lead to the breakdown of mucosal layer and proceed to peptic ulcers.

Pancreatitis is where the pH is altered and degrades/inflame the pancreas.
It can also release the enzymes into the blood.

\section{Diseases of Malabsorption}

Cystic fibrosis patients have abnormally thick mucous secretions due to mutation in chloride transporter.
The thick mucous can block the pancreatic ducts, and therefore no protease activity.

Coeliac disease have been mentioned in the previous lecture.

\section{Digestion of Nucleic Acid}

DNA and RNA are subject to acid hydrolysis in the stomach.
Intestinal nucleases also hydrolyse the phophodiester bonds between the nucleotides.
The individual nucleotides are absorbed via nucleotide transporters (Na co-transport).

It is most unlikely that the DNA/RNA gets incorporated into our body, but it is a possibility as there are evidence of DNA being intact in the gut.

