\section{Post-Translational Modification (PTM)}

Most proteins, if not all proteins, in eukaryotic cells are post-translationally modified.
Post-translational modification increases the number of distinct forms of proteins without increasing the number of genes in the genome.
This means that the level of complexity provided by the proteins are also increased as well.

The diversity of the proteins can be expanded by mRNA splicing and PTM.
PTM can be categorised into two categories: the covalent addition of chemical group to amino acid side chain, or covalent hydrolysis of the peptide bond in a protein.

\section{Covalent Modification of Amino Acids}

There are 15 amino acids known to be PTMed by covalent addition of chemical groups.
For each of the amino acids, there are a variety of modifications that can be made to it.
Three PTMs are discussed in this lecture: phosphorylation, gamma-carboxylation of glutamic acid, and hydroxylation.

\section{Phoshphorylation}

Phosphorylation can occur on the hydroxyl (OH) group of the amino acid side chains.
Amino acids that have OH groups are Ser, Thr, and Tyr.
Phosphorylation of the side chains are done by an enzyme called kinase, which adds the large, negatively charged phosphate group to the side chain, and uses ATP during this process.
To remove the phosphate group, an enzyme called phosphatase is used.

The consequence of adding a phosphate group to these amino acid side chains is that it's going to alter the protein structure, due to the electrostatic attraction between the more negatively charged phosphate group with other postitively charged amino acid side chains.
Since the attraction is going to shift the amino acids slightly, it will cause a shift in the whole protein, and therefore alters the conformation of the protein and affects the biological properties of it.

Quite often, phosphorylation state is used to `switch' the conformational state of the protein to enable a biological process in a cell (e.g. facilitating interaction with other proteins).

\subsection{Insulin Receptor}

Binding of insulin to the insulin receptor causes a conformational change in the receptor, and this exposes the tyrosine in the cytosol side of the membrane.
The exposed tyrosine is then targeted by kinases, and get phosphorylated by it.
Phosphorylated tyrosine can then cause responses (e.g. signal transduction, GLUT4 transporter expression).

\begin{center}
% \includegraphics[width=0.7\textwidth]{insulinreceptor}
\end{center}

\subsection{Na$^+$/K$^+$ Pump}

Phosphorylation state affects which molecule (Na$^+$ or K$^+$) to bind to the pump, and therefore be transported.
This is also energy dependent process, as the pump is constantly being phosphorylated and dephosphorylated.

\begin{center}
% \includegraphics[width=0.7\textwidth]{nakpump}
\end{center}

\section{Gamma-carboxy Glutamic Acid (Gla)}

Gamma-carboxy glutamic acid occurs when an additional carboxyl group is added to the glutamic acid.
This PTM is done in some blood coagulation proteins, and it requires vitamin K as a co-factor.
Ten to twelve Gla enables the formation of bidentate Ca$^{2+}$ binding site and allows blood coagulation proteins to interact with platelets, as a part of blood clot formation process.

When you have vitamin K deficiency, you can't gamma-carboxylate the glutamic acids, since the reaction requires vitamin K as a co-factor.
As a consequence, this leads to bleeding disorder.

\section{Hydroxylation}

Hydroxylation mainly occurs on the 4' (or 3') position of the proline, but it can also occur at the 5' position of lysine.
Hydroxylation provides greater possibilities for non-covalent interactions, so it allows for more H-bonding to occur.
As a result, proteins are able to form higher order structure.

Vitamin C is important in hydroxylation, as it is required for the reaction to occur.
Vitamin C keeps the Fe$^{2+}$ in this oxidation form, which is essential for the hydroxylation process.
Since collagen structure requires hydroxylation to keep its higher order structure, low vitamin C causes disruption in this structure.
This disease is known as scurvy.
