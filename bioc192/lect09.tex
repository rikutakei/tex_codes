\section{Introduction to Biotechnology}

For therapeutic and clinical use of proteins, the proteins must be made and extracted from a source.
Biotechnology is the field of research to produce proteins in systems (sort of).

\subsection{The `Old' Way of Obtaining Proteins}

Therapeutic proteins were extracted from relevant sources (e.g.\ tissues, serum) and was used to treat the patients.
The problems with this approach was the risk of identified and unidentified pathogen infections from the source (e.g. HIV).

\subsection{Recombinant Protein System}

This is the more recent method to produce proteins for therapeutic use.
Plasmids that contain a required protein is expressed in a different organism/system, and the purified and collected.

Recombinant insulin is produced this way, by expressing two separate parts of insulin in \textit{E. coli}.
These parts are extracted and combined to form insulin.

The advantage of this is that it is cheap, has high yield, and it is pathogen free.
However, in a prokaryotic system, the cells are unable to post-trasnlationally modify the protein, which may be essential for the function of some proteins.
Another problem is that the proteins may not fold properly or only partially folded.

\subsection{Eukaryotic System}

This system solves the problem of PTM and protein misfolding, as the proteins will be expressed in a eukaryotic system.

An example of a protein that uses a eukaryotic system is erythropoietin (EPO).
EPO regulated haematopoiesis, and its expression is partially regulated by the blood oxygen level -- EPO acts so that it produce more red blood cells and haemoglobin.
EPO is used therapeutically as well as a performance enhancing drug.
EPO is produced in immortalised Chinese hamster ovary cells.

Different EPOs have different glycosylation profile, which allows us to detect athletes who uses EPO.

\subsection{Recombinant Antibodies}

Recombinant antibodies are designed to control the effects of some proteins in your body (e.g.\ some cell receptors).
Mouse AB can be made, but it is recognised as foreign material in humans and get rejected.
This lead to the formation of chimeric antibodies, so there will be no immune response.

\begin{center}
% \includegraphics[width=0.8\textwidth]{chimericab}
\end{center}

\subsection{Pharming}

The use of live animals to express/make recombinant proteins, for example, in milk.

\subsection*{What you should get out of this lecture}

\begin{itemize}
	\item Read through this lecture section notes
\end{itemize}

