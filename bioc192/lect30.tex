\section{Oxidative Phosphorylation -- Part 1}

Oxidative phosphorylation is the coupled reaction/process of the electron transport chain and the ATP generation by the ATP synthase.
These processes are coupled by the proton motive force (proton gradient).
The electron transport chain forms this gradient by pumping proton into the intermembrane space of the mitochondria, and the ATP synthase uses this gradient to generate ATP.

\section{Electron Transport Chain}

Electron transport chain (ETC) is located in the inner mitochondrial membrane, as the ETC must have access to all of the reduced co-enzymes from other pathways.
In addition to this, since the ETC must form a proton gradient, this inner mitochondrial membrane acts as the barrier for the gradient.
(Side note: you can centrifuge the cells to isolate certain components of the cell, and use this to do experiments on one particular organelle).

\begin{center}
\includegraphics[width=0.7\textwidth]{etc}
\end{center}

In ETC, electrons are transported through the ETC via the electron carriers (complexes I, II, III, IV and mobile carriers), through series of redox reactions.
The trasnported electrons are ultimately used to reduce oxygen into water, meaning that oxygen is the terminal electron acceptor.
Note that protons are pumped into the intermembrane space as electrons are transported through the complexes.

\subsection{Electron Carriers}

As mentioned before, electrons are passed through various electron carriers.
These carriers are arranged in the ETC as complexes, numbered from I to IV.
There are also 2 mobile electron carriers (mobile in the sense that they are able to communicate between the complexes), called ubiquinone (Co-enzyme Q) and cytochrome c.
Note that cytochrome c is associated with complex III on the intermembrane space side of the inner mitochondrial membranei, and is actally not free in solution.
This means that the electrons are transferred from cytochrome c to complex IV by direct contact (remember that membrane is a fluid).

\subsection{Movement of Electrons Through ETC}

The movement of the electrons from the reduced co-enzymes to oxygen through the ETC requires series of electron carriers (mentioned above) to undergo series of redox reactions.
Electrons move from low redox potential to higher redox potential -- i.e. from reduced co-enzyme to oxygen (highest redox potential).
During the electron movement process, energy is released in a stepwise manner through series of carriers, which allows us to capture energy for pumping protons into the intermembrane space.

\begin{center}
\includegraphics[width=\textwidth]{redoxpotential}
\end{center}

