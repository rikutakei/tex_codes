\section{Alcohol Metabolism}

\subsection{Alcohol}

\begin{center}
\setatomsep{2em}
\chemfig{H-C(-[2]H)(-[6]H)-C(-[2]H)(-[6]H)-OH}
\end{center}

Alcohol has 29 kJ/g of energy, and takes about 1 hour to get into the blood.

\subsection{Alcohol Dehydrogenase}

Alcohol is processed into acetaldehyde by alcohol dehydrogenase enzyme, which can process about 10g of alcohol per hour (i.e. one drink per hour).

\begin{center}
\setatomsep{2em}
\chemfig{H-C(-[2]H)(-[6]H)-C(-[2]H)(-[6]H)-OH} \ch{+ NAD+ ->[Alcohol{}Dehydrogenase]}
\chemfig{H-C(-[2]H)(-[6]H)-C(=[1]O)(-[7]H)} \ch{+ NADH + H+}
\end{center}

\subsubsection{Kinetics of Alcohol Dehydrogenase}

Alcohol dehydrogenase has a K$_M$ of about 1mM, where 2 to 3mM of alcohol is equivalent to about 2 to 3 drinks.
This means that alcohol dehydrogenase is already at V$_{max}$ after a couple of drinks, and therefore your body starts to accumulate alcohol.

\subsection{Acetaldehyde Dehydrogenase}

\begin{center}
\setatomsep{2em}
\chemfig{H-C(-[2]H)(-[6]H)-C(=[1]O)(-[7]H)} \ch{+ NAD+ ->[Acetaldehyde{}Dehydrogenase]}
\chemfig{H-C(-[2]H)(-[6]H)-C(=[1]O)(-[7]\chemabove{O}{\scrm})} \ch{+ NADH + H+}
\end{center}

Acetaldehyde is toxic to your body and is quickly converted into acetate mainly in the mitochondria (some in cytoplasm).
Glu487Lys mutation in mitochondrial acetaldehyde dehydrogenase causes asian flush, and affects 40\% of the eastern population.

\subsection{Acetyl-CoA Synthetase}

\begin{center}
\ch{Acetate + CoA + ATP -> Acetyl-CoA + AMP + PPi}
\end{center}

Acetate is then converted into acetyl CoA, which can then be used to generate energy, or it can be converted into fatty acids (and cause fatty liver).

\subsection{Fate of Acetyl-CoA from Alcohol}

The obvious path for the acetyl CoA is the CAC/ETC/oxidative phosphorylation for ATP synthesis.
However, if there is enough ATP/energy, the accumulated acetyl CoA is used to make fatty acids.
Since alcohol has no regulatory system, all alcohol is converted into acetyl CoA and it has to be used as energy or stored as fatty acids.

\subsection{Consequences of Alcohol Metabolism}

Alcohol metabolism causes an increase in NADH and ATP in the cell.
This causes the CAC, $\beta$-oxidation, ETC, pyruvate dehydrogenase, and glycolysis (PFK) to slow down.
It also causes the fatty acids to be esterified back to TAGs, and cause fatty liver.
The production of lactate is also favoured, as the NAD$^+$ source is depleted.
This then causes the pH to lower and inhibits gluconeogenesis in the liver (cause low blood glucose and coma).

\section{Metabolism of Alcohol as a Toxin}

Note that this is a different enzymatic system as the one described above.
This pathway uses the microsomal ethanol oxidising system, where the oxidase enzyme (e.g. cytochrome P450) converts alcohol into acetaldehyde.
This means that the speed of alcohol to acetaldehyde conversion is increased.
However, this also means that it metabolises some drugs faster (since alcohol can directly increase the transcription of cyt P450 enzyme, and therefore faster drug clearance).

\begin{center}
% \includegraphics[width=0.6\textwidth]{chronicalcmet}
\end{center}

