\documentclass[a4paper, 12pt]{report}

\usepackage{graphicx}
\usepackage[utf8]{inputenc}

\newcommand{\HRule}{\rule {\linewidth}{0.5mm}}

\newcommand{\mychapter}[2]{
    \setcounter{chapter}{#1}
    \setcounter{section}{0}
    \chapter*{#2}
    \addcontentsline{toc}{chapter}{#2}
}

\begin{document}

%%%%%%%%%%%%%%%%%%%%%%%%%%%%%%%%%%%%%%%%%%%%%%%%%%%%%%%%%%%%
% Title page for master book - code is pretty self explanatory
% Required packages: setspace
%%%%%%%%%%%%%%%%%%%%%%%%%%%%%%%%%%%%%%%%%%%%%%%%%%%%%%%%%%%%
\begin{titlepage}

\centering

\rule[2.0mm]{0.5\textwidth}{0.5mm}

\begin{doublespace}
    {\Huge \scshape My Tips Book}
\end{doublespace}

\rule[2.0mm]{0.5\textwidth}{0.5mm}

\vspace{2.0mm}
{\Large \scshape Riku Takei}

\vfill

%{\normalsize \scshape A Thesis Submitted For The Degree Of\\}
%{\Large \scshape Master Of Science\\}
%\vspace{2.0mm}
%{\normalsize \scshape At The University Of Otago, Dunedin, New Zealand \\}

\vspace{10.0mm}
{\normalsize \today}

\end{titlepage}


\tableofcontents
\newpage

\mychapter{0}{Introduction}

\mychapter{1}{Lecture 1}
\section{Protein}

Proteins are the workhorses of the cell -- they are required for every process a cell has to do (e.g. DNA replication, apoptosis, etc.)
Proteins are very small (50$\sim$100\AA{}, or 5$\sim$10nm big; a ``nanostructure''), yet they all have unique structure and functions that makes them important for cell survival and growth, and therefore often referred to as ``molecular machines''.

Biochemistry is the study of the molecular details of the biology of the cells/living organisms, including proteins, DNA and RNA.
To understand what components are involved and how they function, you need to know the structure of the thing you're trying to describe -- which, in our case, is the protein.

\section{Amino Acids}

Amino acids are the building blocks of a protein.
Proteins are made from multiple amino acids joined together as a long polypeptide chain, which in turn folds into a certain structure (ref to diff. chapter).

There are 20 amino acids in our cells that make up the proteins (insert image of 20 aa).
All of these 20 amino acids have an amino group (--NH$_3$ group), a carboxyl group (--COOH group), a hydrogen atom, and a variable side chain (--R group) attached to the central carbon atom (C$_{\alpha}$) (insert image of amino acid structure).
Due to the variable side chains, all amino acids except glycine are chiral, meaning that they cannot be mirrored (and hence have two forms: D-- and L-- amino acids).
The reason why glycine is the only amino acid that is achiral is because the glycine side chain group is just a hydrogen atom.

\section{Amino Acid Side Chains}

As mentioned earlier, all 20 amino acids have a common peptide backbone which is made up of the $\alpha$-carbon, carboxyl group, amino group, and a hydrogen atom -- the only thing different between the amino acids are the side chain, or the R group.
This means that each of the 20 amino acids have different chemical properties, depending on the side chain group it has.
Cells take advantage of this and arrange these amino acids in a certain way so the protein can, for example, carry out chemical reactions faster (i.e. it gives the protein its functionality, as well as its unique structure).

\section{Amino Acid Subgroups}

Due to the unique chemical properties of the amino acid side chains, they are grouped into subgroups.
There are four main groups: Non-polar, uncharged polar, negatively charged, and positively charged amino acids.

\subsection{Non-polar}

These amino acids have non-polar (or hydrophobic) side chain, meaning that the side chains cannot be charged under physiological pH.
The amino acids in this group are: alanine, valine, leucine, isoleucine, glycine, cysteine, phenylalanine, tryptophan, methionine, and proline.

\subsection{Uncharged Polar}

These amino acids have side chains that can be charged, but uncharged under normal physiological pH.
As you can see, the side chain contains --OH groups and/or --NH$_2$ groups, whichare able to de/protonate under certain pH.
The amino acids in this group are: serine, threonine, tyrosine, asparagine, and glutamine.


\subsection{Negatively Charged}

These amino acids have side chains that are negatively charged (i.e. deprotonated) side chains at physiological pH.
The side chain contains a deprotonated --COOH group, giving it a negative charge to the side chain.
The pK$_a$ of these amino acid side chains ranges from 4$\sim$5.
The amino acids in this group are: aspartic acid (aspartate) and glutamic acid (glutamate).

\subsection{Positively Charged}

These amino acids have side chains that are positively charged (i.e. protonated) side chains at physiological pH.
The side chain contains a protonated --NH$_2$ group, giving it a positive charge to the side chain.
The pK$_a$ of these amino acid side chains ranges from 10$\sim$12, with an exception of histidine, which has a pK$_a$ of 7$\sim$7.5.
The amino acids in this group are: lysine, arginine, and histidine.

\section{Charge of Amino Acids}

Amino acids have different charges on them, depending on the pH of the environment, as well as the chemical properties of the side chains of the amino acids.
The $\alpha$-amino group ($\alpha$ means that it is joined onto the C$_{\alpha}$) has a pK$_a$ of about 9, whereas the pK$_a$ of the $\alpha$-carboxyl group is about 2 (for all 20 amino acids).
When pH = pK$_a$, 50\% of the molecules are protonated and 50\% of the molecules are protonated.
Depending on the pH of the solution, there will be more or less protonated molecules.

The amount of de/protonated amino acids affect the net charge of the protein, and therefore the isoelectric point (pI) of the protein.
The isoelectric point (pI) is the pH at which the net charge of the protein is 0 (i.e. neutral).

When the pH is low (e.g. 2), the concentration of the hydrogen ion ([H$^+$]) increases, therefore the molecules are likely to become protonated.
Likewise, when the pH is high (e.g. 10), the concentration of the hydrogen ion ([H$^+$]) decreases, therefore the molecules are likely to become deprotonated.
So, in general, when the pH is below the pK$_a$, the molecule is protonated (i.e. H$^+$ ion added), and when the pH is above the pK$_a$, the molecule is deprotonated (i.e. H$^+$ ion removed).

\begin{center}
    \textbf
        {pH \textless{} pK$_a$, then the side chain is protonated \\
        pH \textgreater{} pK$_a$, then the side chain is deprotonated}
\end{center}

Under physiological condition (pH = 7), the --COOH and the --NH$_3$ groups are charged so the amino acid is uncharged.
However, some amino acid side chains may be charged under physiological condition and contribute to the net charge of the amino acid, and therefore the protein.
This is because different amino acid side chains have different pK$_a$ values, hence different charge at physiological pH.

\section{Amino Acid Modifications}

Some amino acids are modified after translation -- this is called the post-translational modification (PTM).
Types of PTM includes:
\begin{itemize}
    \item Disulfide bond formation between cysteines
    \item Phosphorylation
    \item Glycosylation
    \item Methylation
    \item Adenylation
    \item Iodination
    \item Metal binding
\end{itemize}

\section{Proteins, Peptides, and Peptide Bonds}

Amino acids are covalently linked together by peptide bonds -- the bond formed between the amino group of one amino acid with a carboxyl group of another amino acid.
A chain of amino acids linked together by peptide bonds are called peptides, and these are usually short.
Peptides that have a longer chain of amino acids with defined biological function are called proteins.
(Note: peptides are proteins as well -- it's just a little smaller)

\subsection{Amide (peptide) Bonds}

Amide (peptide) bond has a partial double bond characteristic due to the electron delocalisation/sharing between the oxygen atom and the nitrogen atom.
This means that the amide bond is rigid, and also have a dipole nature (oxygen is slightly negative, nitrogen is slightly positive).

The amide bonds are predominantly in \textit{trans}-configuration in the protein, although some can be in \textit{cis}-configuration, especially when the bond is preceded by proline.
This is due to ring structure that the proline side chain forms with its amino group (causes steric clashes).

By convention, the free amino group on the end of the peptide chain is called the N-terminus of the protein, and the free carboxyl group is known as the C-terminus.
This gives a `directionality' to the peptide sequence, from the N-terminus to the C-terminus, and we draw primary sequence and count the amino acids according to this directionality.

\mychapter{2}{Lecture 2}

\section{Bond Rotation}







\end{document}
