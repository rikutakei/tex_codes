\documentclass[a4paper, 12pt]{report}

\usepackage{graphicx}
\usepackage[utf8]{inputenc}

\newcommand{\HRule}{\rule {\linewidth}{0.5mm}}

\newcommand{\mychapter}[2]{
    \setcounter{chapter}{#1}
    \setcounter{section}{0}
    \chapter*{#2}
    \addcontentsline{toc}{chapter}{#2}
}

\begin{document}

%%%%%%%%%%%%%%%%%%%%%%%%%%%%%%%%%%%%%%%%%%%%%%%%%%%%%%%%%%%%
% Title page for master book - code is pretty self explanatory
% Required packages: setspace
%%%%%%%%%%%%%%%%%%%%%%%%%%%%%%%%%%%%%%%%%%%%%%%%%%%%%%%%%%%%
\begin{titlepage}

\centering

\rule[2.0mm]{0.5\textwidth}{0.5mm}

\begin{doublespace}
    {\Huge \scshape My Tips Book}
\end{doublespace}

\rule[2.0mm]{0.5\textwidth}{0.5mm}

\vspace{2.0mm}
{\Large \scshape Riku Takei}

\vfill

%{\normalsize \scshape A Thesis Submitted For The Degree Of\\}
%{\Large \scshape Master Of Science\\}
%\vspace{2.0mm}
%{\normalsize \scshape At The University Of Otago, Dunedin, New Zealand \\}

\vspace{10.0mm}
{\normalsize \today}

\end{titlepage}


\tableofcontents
\newpage

\mychapter{0}{Introduction}

\mychapter{1}{Lecture 1}
\section{Protein}

Proteins are the workhorses of the cell -- they are required for every process a cell has to do (e.g. DNA replication, apoptosis, etc.)
Proteins are very small (50$\sim$100\AA{}, or 5$\sim$10nm big; a ``nanostructure''), yet they all have unique structure and functions that makes them important for cell survival and growth, and therefore often referred to as ``molecular machines''.

Biochemistry is the study of the molecular details of the biology of the cells/living organisms, including proteins, DNA and RNA.
To understand what components are involved and how they function, you need to know the structure of the thing you're trying to describe -- which, in our case, is the protein.

\section{Amino Acids}

Amino acids are the building blocks of a protein.
Proteins are made from multiple amino acids joined together as a long polypeptide chain, which in turn folds into a certain structure (ref to diff. chapter).

There are 20 amino acids in our cells that make up the proteins (insert image of 20 aa).
All of these 20 amino acids have an amino group (--NH$_3$ group), a carboxyl group (--COOH group), a hydrogen atom, and a variable side chain (--R group) attached to the central carbon atom (C$_{\alpha}$) (insert image of amino acid structure).
Due to the variable side chains, all amino acids except glycine are chiral, meaning that they cannot be mirrored (and hence have two forms: D-- and L-- amino acids).
The reason why glycine is the only amino acid that is achiral is because the glycine side chain group is just a hydrogen atom.

\section{Charge of Amino Acids}

Amino acids have different charges on them, depending on the pH of the environment, as well as the chemical properties of the side chains of the amino acids.
The $\alpha$-amino group ($\alpha$ means that it is joined onto the C$_{\alpha}$) has a pK$_a$ of about 9, whereas the pK$_a$ of the $\alpha$-carboxyl group is about 2 (for all 20 amino acids).
When pH = pK$_a$, 50\% of the molecules are protonated and 50\% of the molecules are protonated.
Depending on the pH of the solution, there will be more or less protoneted molecules.

When the pH is low (e.g. 2), the concentration of the hydrogen ion ([H$^+$]) increases, therefore the molecules are likely to become protonated.
Likewise, when the pH is high (e.g. 10), the concentration of the hydrogen ion ([H$^+$]) decreases, therefore the molecules are likely to become deprotonated.
So, in general, when the pH is below the pK$_a$, the molecule is protonated (i.e. H$^+$ ion added), and when the pH is above the pK$_a$, the molecule is deprotonated (i.e. H$^+$ ion removed).

\begin{center}
    \textbf
        {pH \textless{} pK$_a$, then the molecule is protonated \\
        pH \textgreater{} pK$_a$, then the molecule is deprotonated}
\end{center}

Some amino acid side chains are also able to be charged, and again, the charge of the amino acid side chain depends on its pK$_a$ value and the pH of the environment.

\section{Amino Acid Side Chains}

As mentioned earlier, all 20 amino acids have a common peptide backbone which is made up of the $\alpha$-carbon, carboxyl group, amino group, and a hydrogen atom -- the only thing different between the amino acids are the side chain, or the R group.
This means that each of the 20 amino acids have different chemical properties, depending on the side chain group it has.
Cells take advantage of this and arrange these amino acids in a certain way so the protein can, for example, carry out chemical reactions faster (i.e. it gives the protein its functionality, as well as its unique structure).

\section{Amino Acid Subgroups}

Due to the unique chemical properties of the amino acids, they are grouped into subgroups.
There are four main groups: Non-polar, uncharged polar, negatively charged, and positively charged amino acids.

\subsection{Non-polar}




\subsection{Uncharged Polar}




\subsection{Negatively Charged}




\subsection{Positively Charged}



\mychapter{2}{Lecture 2}
\section{me}







\end{document}
