\section{Introduction to Enzymes}

In most cases, enzymes are proteins (in some cases, it's RNA enzymes) that acts as biological catalysts.
Enzymes increases the rate of reaction, but does NOT change the equilibrium, nor the thermodynamics of a reaction.
This means that, even though enzymes make the reactions to occur faster, it cannot shift the chemical reaction to favour one product over another.
Note that enzymes are named ``-ases''.

For a chemical reaction to occur, the substrates must be in close proximity of one another, and also have the right orientation.
Enzymes catalyse reactions by doing just that: bind the substrate in the right orientation in close proximity so a chemical reaction can occur at a faster rate.

\section{Active Site of an Enzyme}

Active site is the 3-dimensional site of an enzyme where the reaction occurs.
The active site has amino acid side chains projecting into it to help substrate binding.
The shape of the active site determines the specificity of the enzyme, since enzymes are (usually) very specific.
This means that enzymes only bind to certain substrate, and no other substrates.

\subsection{Geometric Specificity and Stereospecificity}

\begin{center}
\includegraphics[width=0.7\textwidth]{geo_stereo}
\end{center}
Geometric specificity is where the substrate must be made of same component/structure for it to be recognised by the enzyme.
Stereospecificity is where the substrate has the required components,but don't have the right orientation, and therefore not recognised by the enzymes.

\section{Enzyme-Substrate Interaction}

\begin{center}
\ch{ E + S <=> ES <=> [{T}] EP <=> E + P }
\end{center}

In an enzyme catalysed reaction, substrate binds to the active site and forms an enzyme-substrate complex (ES), which is stabilised by non-covalent interactions.
Note that this interaction is specific.
As the reaction proceeds, transition state (T), which is neither a substrate nor the product, is formed, then proceeds further into enzyme-product complex (EP).
This complex can then be dissociated to form the original enzyme ready for another reaction, and release the product.

There are two models that describes how the substrate binds to the enzyme: the lock-and-key model, or the induced-fit model.

\begin{center}
\includegraphics[width=0.6\textwidth]{enzymemodels}
\end{center}

\subsection{Lock-and-Key Model}

The concept of the lock and key model is fairly simple -- the substrate must have the shape/structure that perfectly matches the enzyme active site.
If the substrate doesn't then, the enzyme cannot react with it and no reaction occurs.
However, this is not really the case in nature as enzymes changes conformation when the substrate binds to it.

\subsection{Induced-fit Model}

This is how an enzyme actually works.
The substrate and the enzyme active site does not match perfectly as it did for the lock and key model.
When the substrate and the enzyme binds, it causes a conformational change in the enzyme, which allows the enzyme to properly bind to the substrate.

Note that one way to increase the rate of reaction is by stabilising the transition state of the substrate.
Enzymes can do this by preferentially binding to the substrate in a transition state conformation.
This is quite often the case, as enzymes usually stabilises the transition state of the substrate during the catalysis.

\section{Enzymes, Activation Energy, and \\Catalytic Mechanisms}

\begin{center}
\includegraphics[width=0.5\textwidth]{enzcatalreac}
\end{center}

Enzymes can decrease the activation energy of a chemical reaction by various catalytic mechanisms.
There are 6 ways to do this:

\begin{enumerate}
\item Acid-base catalysis
\item Covalent catalysis
\item Metal ion catalysis
\item Electrostatic catalysis
\item Proximity and orientation effects
\item Preferential binding of the transition state complex
\end{enumerate}

In this lecture, only the first three was discussed.

\subsection{Acid-base Catalysis}

This kind of catalysis involves the transfer of H$^+$ ions, for example by glutamic and aspartic acids (COOH group), or by lysine (NH$_3^+$ group).
Note that these side chains require certain pH for it to be in the right ionisation state, and therefore to be able to transfer H$^+$ ions.
In this sense, histidine is very useful as it has a pK$_a$ of about 6.5, which is close to the physiological pH.
This means that with a slight change in pH, it can be either a H$^+$ donor or acceptor, and therefore able to take part in the acid-case catalysis reaction.

\subsection{Covalent Catalysis}

In a covalent catalysis, the enzyme forms a highly reactive, short-lived intermediate which is covalently attached to the enzyme.

\begin{center}
\ch{A-B + E -> A-E + B ->  A + B + E}
\end{center}

The enzyme requires an amino acid with a side chain that can act as a nucleophile (e.g.ser, cys)

\subsection{Metal Ion Catalysis}

About a third of known enzymes require metal ions for their catalytic activity.
Metals can provide substrate orientation, binding energy, and/or sites for redox reactions.
Metals act as a cofactor in these reactions.

\section{Co-factors and Co-enzymes/Co-substrates}

Some enzymes require other factors to catalyse chemical reactions.
These factors can be metal ions or small organic molecules (co-enzymes).
Co-enzymes, or co-substrates, are molecules that help with the reaction, and is changed after the reaction occurs.
Co-substrates are usually derived from vitamin derivatives.
(An example used in the lecture was NAD)

\section{Enzyme Nomenclature}

Since there are MANY enzymes in the world, scientists needed to standardise the naming of the enzymes so other people knew which enzymes they were talking about.
The classes of enzymes are:

\begin{enumerate}
\item Oxidoreductases -- transfer electrons
\item Transferases -- transfer groups
\item Hydrolases -- hydrolysis reactions
\item Lyases -- Addition of groups to double bonds (or reverse)
\item Isomerases -- forms isomeric forms of a molecule
\item Ligases -- formation of C-C, C-S, C-O and C-N bonds (coupled with ATP hydrolysis)
\end{enumerate}
