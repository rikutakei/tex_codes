\section{Lipid Digestion and Absorption}

Lipid digestion starts and the mouth, which is continued in the stomach, and lastly in the small intestine (main site of fat digestion).
Lipid digestion is done by the lipase enzymes, present in all three location mentioned above, which breaks down the triacylglycerol (TAG) into glycerol and free fatty acids (FAs).

\section{Bile Acids/Salts}

Dietary lipids must be emulsified/solubilised in the aqueous solution.
This is done by the bile acids/salts, which are made from cholesterol.
Bile acids are synthesised in the liver and stored in the gall bladder as bile, which is secreted in response to cholecystokinin.

Bile acids are powerful detergents with hydrophobic  and hydrophilic surfaces (amphipathic), which allows it to form micelles with TAGs and increase the surface area for lipid digestion by the pancreatic lipase (also associated with the micelles).
Note that bile acids have similar structure to cholesterol, but have OH and COOH groups on them.

\section{Bile Composition}

Bile contains: water, bile acids, electrolytes, cholesterol, phospholipids, and bile pigments (bilirubin).
Note that the ration of electrolytes to cholesterol (electrolytes:cholesterol) is important to avoid gall stones in the gall bladder.

\section{Digestion of Lipids}

Firstly, the lipids are emulsified by the bile salts (in response to cholecystokinin) to form micelles.
Pancreatic lipase binds to the lipid/aqueous interface of the micelles and hydrolyses the TAGs at positions 1 and 3 of the glycerol backbone.
After this, smaller micelles form with bile salts, hydrolysed FFAs, MAGs, and cholesterol.
Micelles are then absorbed across the intestinal cell membrane into the blood.

\section{Fat Malabsorption}

This leads to excess fat and fat soluble vitamins in faeces, and are caused by conditions that interfere with bile or pancreatic lipase secretion (e.g. pancreatitis, gall bladder or liver disease).

\subsection{Orlistat (Xenicol)}

Potent inhibitor of pancreatic lipase, and therefore TAGs get excreted without getting absorbed.

\section{Process of Fat Absorption}

\subsection{Lipoproteins}

Lipoproteins are particles that are made of lipid core (TAGs and cholesterol) surrounded by phospholipids and proteins that sulubilise the lipids in the blood for transport around the body.
In general, the structure of lipoproteins are made of core TAGs and cholesterol esters, surrounded by phospholipids and apoproteins.
The apoprotein component of the lipoproteins provides the specificity of where the lipoproteins are transported.

There are four main classes of lipoproteins: chylomicrons, VLDL, LDL, and HDL.
\begin{description}
\item[Chylomicrons] are the least dense (in terms of TAG content) out of all four classes. These contain mainly TAGs, and the apoproteins associated are: apoB48, C2, C3, and E.
\item[VLDL] are the next least dense class, which contains mainly TAGs (but not as much as chylomicrons), with slightly higher phospholipid content. The main apoproteins associated are: apoB100, C1, C2, C3 and E.
\item[LDL] are second most dense class and consists mainly of cholesterol, with higher protein content. Apoprotein associated with this class is apoB100.
\item[HDL] are the most dense class and consists mainly of proteins. Main apoproteins involved in this class are: apoA1, A2, C1, C2, C3, D and E.
\end{description}

The two main functions of lipoproteins are to make the lipids soluble for transport in the blood, and to provide a delivery system for shifting the lipids in/out of the cells.

\subsubsection{Function of Apoproteins}

Apoproteins have various functions depending on the type of apoprotein.
Some apoproteins are structural for lipoprotein assembly (apoB).
ApoB and E are `ligands' for cell surface receptors (i.e. recognition for lipoprotein absorption).
Some are enzyme cofactors (e.g. apoC2 for lipoprotein lipase).

\subsection{Lipid Transport Pathways}

There  are two major lipid transport pathways.
The exogenous chylomicron pathway transports dietary fat from the GI tract to other parts of the body.
The endogenous VLDL/LDL pathway transports endogenously synthesised fat.

\subsection{Exogenous Pathway}

The FFAs and MAGs that are absorbed into the intestinal cell gets resynthesised back into TAGs.
These TAGs and other lipids combine with apoB in the ER to form chylomicrons and secreted from the intestinal cells into the blood via the lymphatic system.
FFAs are transported to adipose tissues and muscles, and the remnants are transported to the liver where it is recognised by the apoE receptor.

\subsubsection{Lipoprotein Lipase}

Lipoprotein lipase is an enzyme found on the endothelial surface that hydrolyses TAGs in the lipoproteins to glycerol and FFAs.
It has the highest activity in the heart, adipose tissues and skeletal muscles.
Lipoprotein lipase is activated by apoC2 protein, and mutation in apoC2 or lipoprotein lipase leads to increased chylomicrons and plasma TAGs.

\subsection{Endogenous Pathway}

Fat can also be synthesised in the liver as well, and therefore require transport out of the liver cell.
VLDLs are synthesised and exported into the blood from the liver cells, and contains apoC2 to interact with lipoprotein lipase for further TAG breakdown.
VLDL can get ``receycled'' into LDL -- the cholesterol rich lipoprotein that is considered to be ``bad'', and only contains apoB100.
These LDL can get absorbed/taken up by other tissues via  the LDL receptor, but if this receptor is mutated, it can cause increased cholesterol and LDL in the blood.

\subsection{Familial Hypercholesterolemia (FH)}

This is caused by defects in LDL receptor gene and leads to premature atherosclerosis.
It is a dominant disorder and so heterozygotes are also affected by this condition.
This leads to increased LDL level by about 2 to 3 times higher.
FH is treated with statins (drug that inhibits one of the enzymes involved in cholesterol synthesis).

