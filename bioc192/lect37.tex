\section{Exercise Metabolism}

Different pathways and fuels are used for different kinds of exercise.
There is about \SI{5}{\micro\mol} of ATP per g of muscle, which can generate enough energy for an exercise for a second.
This means that we need to make more ATP to keep doing exercise, and depending on the type of exercise, we have to utilise different pathways and fuels at different time.

\section{Anaerobic Exercise}

Anaerobic exercise is defined as the high intensity exercise that requires rapid generation of energy in short period of time where oxygen (although available) is not used as an electron acceptor (i.e. not making energy through CAC).
In anaerobic exercise, it mostly uses phosphocreatine and glycogen to make energy for the exercise.

\subsection{Phosphocreatine}

Phosphocreatine is the on site, fast fuel.
There is about 20mol of phosphocreatine per g of muscle, and it's a high energy compound.
Phosphocreatine is made in the liver from arginine and glycine, which is then transported to msucles.
The phosphate from the phosphocreatine is transferred onto ADP to make ATP, which is catalysed by creatine kinase.

Some studies show that if you are able to maintain the creatine in your muscle, you can exercise for a longer time.

\subsection{Glycogen}

Glycogen phosphorylase breaks down the glycogen into glucose-1-phosphate in muscle, which is converted back into glucose-6-phosphate, and fed into glycolysis.

\subsection{Anaerobic Glycolysis}

Muscle glycogen is used as fuel and no oxygen is required for this.
ATP is generated from the substrate level phosphorylation reaction, and NAD$^+$ is regenerated by the pyruvate to lactate conversion.
This allows rapid generation of energy, but only for a short time.

\subsection{Regulation of Glycolysis in Exercising Muscle}

Glycogen mobilisation is stimulated by Ca$^{2+}$ and adrenaline (when you're exercising you release calcium ions).
Phosphofructokinase activity is upregulated by AMP and inorganic phosphates (produced from ATP usage into ADP; Pi from making ATP from two ADP).
This allows increased flux through the glycolytic pathway.

\section{Aerobic Exercise}

Aerobic exercise depends more on the oxidation of glucose and fatty acids, using oxygen (i.e. CAC etc.).
After 20sec of anaerobic exercise, aerobic exercise kicks in.
At this point, 50\% of energy comes from glucose and 50\% of energy comes from fatty acids.

\begin{center}
% \includegraphics[width=0.7\textwidth]{exercisemetabolism}
\end{center}

\section{Aerobic Training}

Aerobically trained people are able to rely less on glycogen and therefore able to exercise for a longer period of time (i.e. don't have to rely on glycogen as much).
When you train, you increase the mitochondrial ability in the cells, and therefore able to oxidise lipids and carbohydrates more efficiently.

\section{Muscle Fibre Types}

\begin{description}
\item[Type I] is the red (slow twitch) fibre, and mainly uses oxidative phosphorylation for energy production.
\item[Type II] is the white (fast twitch) fibre, and mainly uses glycolytic paths for energy production.
\end{description}

\section{Myostatin}

Myostatin acts to inhibit muscle growth and differentiation.
People with defects in this gene therefore have increase muscle production/development (due to increase in certain transcription factors for muscle development).

