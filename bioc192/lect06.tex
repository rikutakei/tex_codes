\section{Haemoglobin Variants}

There are a lot of haemoglobin variants.
Some variants are naturally occurring, whereas other variants are caused by genetic mutations.

\section{Normal Variants}

These variants are normal in the sense that they are naturally occurring, and not caused by abnormal mutations.
These variants are different to the normal haemoglobin in that it uses different type of subunits to make the haemoglobin up.

\subsection{Foetal Haemoglobin (HbF)}

Foetal haemoglobin (HbF) is only present in the foetus blood.
HbF is made up of two $\alpha$ subunits and two $\gamma$ subunits, compared to the adult haemoglobin (HbA) which is made up of two $\alpha$ and two $\beta$ subunits.
Since the foetus has to obtain oxygen from the mother's blood via the placenta, HbF has a higher affinity for oxygen than the HbA.
This is because HbF binds less strongly to BPG due to the Ser143 in the $\gamma$ subunit (His143 in $\beta$ subunit), thereby increasing the oxygen affinity.

The graph below just shows how different subunits are made at different developmental stage.(insert the graph)
As you can see, there are subunits other than $\alpha$, $\beta$ and $\gamma$ subunits.

\section{Abnormal Variants}

These variants are caused by genetic mutations that alter the properties of haemoglobin, and can lead to anaemia or other pathological symptoms.

\subsection{Sickle Cell Haemoglobin (HbS)}

This variant is caused by a single point mutation on the $\beta$ subunit of haemoglobin.
This point mutation causes an amino acid substitution from glutamic acid to valine (E6V mutation), which is on the surface of the protein.

The consequence of this mutation is the formation of deoxy-Hb aggregate/crystal in the red blood cell.
Since valine is a non-polar amino acid, it can stick/bind to other haemoglobin at specific site (in a hydrophobic pocket).

\subsection{Met-Haemoglobin (HbM)}

Methaemoglobin (HbM) is caused by mutations that changes the oxidation state of the iron atom in the haem from Fe$^{2+}$ to Fe$^{3+}$.
This means that the oxygen cannot bind (or have very low affinity for oxygen) to the haem iron, as it is in a different oxidation state.

\subsection{Haemoglobin C (HbC)}

Haemoglobin C (HbC) is common in West Africa, and is caused by a point mutation that causes a substitution mutation.
The substitution is at the same place as HbS: glutamic acid to lysine at position 6 (E6K).
As a consequence of this substitution, the charge is swapped from negatively charged glutamic acid to positively charged lysine.

Malaria is caused by the infection of red blood cells by \textit{Plasmodium \\falciparum}.
Infected red blood cells will contain malaria parasite proteins that produce knobs on the surface of the red blood cells and causes it to adhere to the capillaries, causing tissue damage due to reduced oxygen supply.
However, individuals with homo or heterozygous HbC have less adherence, and thus have some resistance to malaria infection.





