\newpage
\begin{center}
\textbf{Lecture 2 Questions}
\end{center}

\begin{enumerate}

\item Define primary, secondary, supersecondary, tertiary, and quaternary structure of a protein.

%\begin{description}

%\item [Primary structure] is the actual amino acid sequence of the protein. All the information required for the protein to fold is contained in this primary sequence.
%\item [Secondary structure] is the small structure created from H-bonding between the peptide backbones of the amino acid sequence. Structures include alpha helix and beta sheets.
%\item [Supersecondary structure] is the structure made from multiple secondary structure to form a ``motif''. Also known as domains.
%\item [Tertiary structure] is the 3D structure of the protein, made from the supersecondary structure. This structure often has a functional role in the cell. May also be called as subunits.
%\item [Quaternary structure] is the structure made from multiple subunits (i.e. the formation of multi-subunit protein).

%\end{description}

\item Describe some properties of $\alpha$-helix.

%H-bond between the n$^{th}$ carbonyl oxygen and n+4$^{th}$ NH group. Amino acid side chains project outwards (therefore can have charged or non-polar side). Due to the dipolar nature of the peptide bonds, the helix is also have charges. Proline and glycine are usually the helix breakers.

\item Describe some properties of $\beta$-sheets.

%H-bond between carbonyl and NH group of the peptide backbones of different beta strands. Can be parallel or antiparallel, depending on the directionality. Side chains point above and below the sheet, again providing different characteristics on different surface. It is pleated (i.e. not flat).

\item Describe some properties of turns.

%About 30\% of protein consists of turns. Usually made of prolines and glycines.

\item Why are proline and glycine ``helix-breakers'', and often found in turns?

%Proline is an imino acid and has a unique R side chain that connects back to the amino group of the alpha carbon. Due to this, proline has a unique rigid structure that disrupts the secondary structure such as alpha helix.
%Glycine has an H atom as its R group, which provides little steric hindrance, and thus very flexible and destabilises the secondary structure.

\item Define $\phi, \psi, \omega,$ and $\chi$ angles.

%Phi is the angle between the alpha carbon and the N atom involved in the peptide bond.
%Psi is the angle between the alpha carbon and the C atom involved in the peptide bond.
%Omega is the angle of the actual peptide bond (i.e. between one alpha carbon to the next). It can only be 0 or 180 degrees.
%Chi is the angle of rotation of the amino acid side chain relative to the alpha carbon.

\item What is the significance of the Ramachandron plot?

%Ramachandron plot shows all possible phi/psi angle pairs an alpha carbon can have. This plot signifies the fact that the secondary structures lie within the predicted allowed region of the plot. However, some of the bond angles do not conform to the allowed region, due to flexible glycine residues.

\item Explain how the side chains of the amino acids are oriented in proteins.

%See the answers to alpha helix and beta sheets. Hydrophobic side chains are buried inside the core of the protein, whereas the polar residues are usually on the outside, facing the solvent.

\item What is the significance of primary sequence and the tertiary/quaternary structure of the protein on its function?

%Primary sequence has all the amino acids required for the protein to fold and function properly -- it's not in the right position. Tertiary and quaternary structure brings the important amino acids together and allow the proteins to have its functionality.

\end{enumerate}
