\newpage
\begin{center}
\textbf{Lecture 3 Questions}
\end{center}

\begin{enumerate}

\item What is a ``structural domain''?

%\textit{Structural domain is the structure formed from multiple secondary structures to make a characteristic domain or ``motif''. It's a supersecondary structure.}

\item List some interactions that stabilise the tertiary structure of a protein.

%\textit{Hydrophobic interactions, electrostatic, H-bond between side chains, disulfide bond between two cysteine residues, and metal ion binding.}

\item How are the secondary structures represented in a diagram?

%\textit{Helices are cylindrical or twisted,  and beta-sheets are shown as arrows.}

\item What do the terms `native' and `denatured' mean in terms of protein folding?

%\textit{Native means that the protein is in the form that is normally found in the cell. Denatured means that the protein is not in the native structure -- usually in a partially, if not fully, unfolded state.}

\item Describe the basic principal of the Anfinsen's experiment and the significance of the experiment in terms of primary and secondary/tertiary structure.

%\textit{Anfinsen completely denatured a protein (ribonuclease) with urea and mercaptoethanol (thiol reagent). The urea denatured the protein, and the mercaptoethanol broke the disulfide bonds. When urea was removed (but not the mercaptoethanol), the protein was able to fold, but not into its native structure. However, when both urea and mercaptoethanol were removed, ribonuclease was able to fold back into its native structure. \\ This experiment showed that proteins were able to fold spontaneously, and that all the information required to fold into its secondary/tertiary structure is from the primary structure.}

\item Describe some of the ways in which proteins are folded.

%\textit{Proteins can fold naturally on its own, or it can use other proteins such as chaperones and chaperonins}

\item Describe a typical folding process a protein will have, going from the primary structure to its tertiary structure.

%\textit{Hydrophobic interaction of the non-polar amino acids drives the initial folding of the secondary structure. These secondary structures then come together to form supersecondary structure, forming domains. These domains then come together to form the tertiary structure of the protein, and small, distant non-covalent (and maybe covalent) interactions stabilise this final structure.}

\item Name some diseases that are caused by protein misfolding.

%\textit{Alzheimer's disease, Creutzfeldt-Jakob (prion) disease.}

\end{enumerate}
