\begin{center}

\textbf{Lecture 1 Questions}

\end{center}

\begin{enumerate}

\item Name all of the groups attached to the $\alpha$-carbon of an amino acid.

%\textit{The H atom, an amino (NH$_2$) group, a carboxyl (COOH) group, and an R group (variable for every amino acid).}

\item What is the only achiral amino acid? Why?

%\textit{Glycine, because the R group is an H atom.}

\item Name the 4 categories used to group the amino acids.

%\textit{Non-polar, polar uncharged, positively charged and negatively charged amino acids.}

\item What are the implication(s) of these groups in protein function and/or structure?

%\begin{description}
%\item [Non-polar amino acids] These amino acids are hydrophobic, and therefore the major player in protein folding. These amino acids are usually present in the middle/core of the protein.
%\item [Polar-uncharged amino acids] These amino acids are uncharged, but in the right environment, it can be charged.
%\item [Positively and Negatively charged amino acids] These amino acids are usually charged in physiological conditions and contribute to the protein's charge. These amino acids can also contribute/take part in the active site for chemical reactions.
%\end{description}

\item Fill the gaps:

\begin{tabular}{|c|c|c|}
\hline
\textbf{Full name} &\textbf {3 letter abbreviation} & \textbf{1 letter abbreviation}\\
\hline
Alanine & & \\
\hline
Valine & & \\
\hline
Leucine & & \\
\hline
Isoleucine& & \\
\hline
Glycine& & \\
\hline
Cysteine& & \\
\hline
Phenylalanine& & \\
\hline
Tryptophan& & \\
\hline
Methionine& & \\
\hline
Proline& & \\
\hline
Serine& & \\
\hline
Threonine& & \\
\hline
Tyrosine& & \\
\hline
Asparagine& & \\
\hline
Glutamine& & \\
\hline
Aspartic Acid& & \\
\hline
Glutamic acid& & \\
\hline
Lysine& & \\
\hline
Histidine& & \\
\hline
Arginine& & \\
\hline

\end{tabular}

\item At physiological pH, what are the ionisation states of the amino group and carboxyl group? Note the approximate pKa of the groups, if you know them.

%\textit{Amino group (pKa = approx. 3) will be protonated, carboxyl group (pKa = approx. 8 to 10) will be deprotonated. It is in the zwitterion form.}

\item Draw the peptide bond with any significant features about it.

\item What is post-translational modification?

%\textit{PTM is a modification on amino acids and/or proteins after it is translated from mRNA. This allows the protein to have different functins and/or roles in the cell, and also provides diversity without increasing the genome.}


\end{enumerate}
