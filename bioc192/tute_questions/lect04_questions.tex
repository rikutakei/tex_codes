\newpage
\begin{center}
\textbf{Lecture 4 \& 5 Questions}
\end{center}

\begin{enumerate}

\item Describe the structural features of haemoglobin.

%\textit{Haemoglobin is made from four globin subunits (two alphs and two beta), with each subunit containing a molecule in the middle of the globin subunit (attached to the HisF8 residue).}

\item How is the iron bound to the haem moeity?

%\textit{Haem iron is slightly out of the plane of the haem plane. When oxygen binds to the haem iron, the iron is pulled into this plane and also causes a conformational change of the subunit. }

\item How does the redox state of the haem iron affect oxygen binding to the haemoglobin?

%\textit{Oxygen can only attach to the haem iron when it is in the Fe$^{2+}$ state, therefore the when it is oxidised to Fe$^{3+}$ state, the oxygen cannot bind to the iron.}

\item Explain the terms ``co-operativity'' and ``allosteric'' in the context of haemoglobin.

%\textit{Co-operativity means that the subunits co-operate with one another to make oxygen binding/release easier as more/less oxygen molecules bind to it (due to conformational change). Allosteric means that the activity/binding affinity of a protein can be altered by the binding of a small molecule at a distant site to the active site.}

\item Why are co-operativity and allosterism of haemoglobin physiologically important?

%\textit{Co-operativity is important for the release/binding of oxygen at the right place -- you want oxygen to fully bind to the haemoglobin at the lungs, whereas you would want it to release at distant tissue sites. Allosterism is important as it assists the release/binding by allosteric effect.}

\item Why is myoglobin not allosteric nor co-operative?

%\textit{It is not allosteric nor co-operative since myoglobin is a single subunit protein -- it doesn't have any other subunits to act allosterically/co-operatively.}

\item Name the main allosteric regulators of haemoglobin.

%\textit{2,3-BPG, H$^+$ (pH) and CO$_2$.}

\item Describe the effects of each of the molecules you have named in question 7.

%\textit{BPG binds to the middle of the haemoglobin protein and changes its conformation with its negative charges. This causes the haemoglobin to favour the T state rather than the R state, and therefore facilitates the release of oxygen. \\ H$^+$ (or the pH) affects the conformation of the haemoglobin, since the charge of the protein is going to change. As H$^+$ increases, the more protonated the haemoglobin becomes, and therefore alters the conformation so that the affinity for oxygen decreases. \\ CO$_2$ binds to the haemoglobin at distant site to the active site and decreases the affinity for oxygen. \\ In all three cases, the molecules work to decrease the affinity for oxygen and facilitates the release of oxygen.}

\item What is the significance of the blood buffer system?

%\textit{The buffer system shows how CO$_2$ and H$^+$ are related to one another. At distal tissue site, the relative concentration of CO$_2$ is high, which also means that there will be more H$^+$ present (i.e. decreased pH). This means that at distal tissue sites, haemoglobin is more likely to release oxygen at these site. In contrast, at the lungs, the concentration of CO$_2$ is low, and therefore less H$^+$ is present (i.e. higher pH). This means that the binding of oxygen is favoured at the lungs, which is expected.}

\item Describe the basic idea behind both the sequential model and the concerted model.

%\textit{In sequential model, binding of one substrate changes the conformation of the subunit it bound to, and also the subunits adjacent to that subunit. This allows the other subunits to bind to the substrate easier.\\ In concerted model, the T and the R states are at equilibrium (i.e. all subunits are either in T or R state), and the substrate can bind to the protein at any time, at any conformational state. When the substrate is bound, it favours the R state and becomes easier for the substrates to bind to other subunits.}

\end{enumerate}
