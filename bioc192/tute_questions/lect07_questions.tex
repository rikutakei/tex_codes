\newpage
\begin{center}
\textbf{Lecture 7 Questions}
\end{center}

\begin{enumerate}

\item What is post translational modification?

%\textit{It is a modification done on the protein after it has been translated from mRNA.}

\item Why is it important to have post translational modification?

%\textit{PTM allows the cell to increase the number of distinct forms of protein without increasing the genome size. This also increases the level of complexity (i.e. functionality) provided by the proteins, so have greater control over many functions of proteins.}

\item Describe what each of the following post translational modifications does, with some example(s) for each.
\begin{itemize}
\item[a)] Phosphorylation\\
%\textit{Phosphorylation is the addition of the phosphate group to a OH group of the Ser, Thr, and Tyr amino acid side chains by a kinase enzyme. This adds a large negatively charged group to the protein, which may cause some conformational change that can affect the activity of the protein. An example of this PTM is with the insulin receptor -- binding of insulin exposes the Tyr for phosphorylation, which can then cause some cellular response.}
\item[b)] Gamma-carboxylation of glutamic acids\\
%\textit{This PTM is where a carbocyl group is added to the glutamic acid and provides a bidentate Ca$^{2+}$ site in blood coagulation proteins. The process requires vitamin K as a co-factor.}
\item[c)] Hydroxylation\\
%\textit{Hydroxylation occurs mainly on proline, but can also occur on lysine. Hydroxylation provides extra site for H-bonding to form higher order structures (e.g. in collagen). This PTM requires vitamin C to assist the reaction, hence you get scurvy when low on vitamin C.}
\end{itemize}

\end{enumerate}
