\section{Normal Mode}

Normal mode can be accessed by pressing  the \verb|<esc>| or \verb|<ctrl-[>| command, and is probably the mode that you will be in most of the time.
This is because most of your editing will start in this mode, then switch into another mode to complete your editing task.
In some cases, you probably won't have to leave the normal mode at all (for example deleting a word or a line).

An analogy used in the book was of a painter -- painters, when drawing up a nice piece of artwork, do not leave their paintbrush on the canvas.
Instead, they rest their paint brush, look at it and think about it for a while, and then decide what to do next.
Whatever they do next can be anything from picking a different colour of paint to paint in, or it could be a different kind of brush to give different texture.
Vim's normal mode works just like that -- it gives you ``resting'' mode to think about what to write, what words to edit, how to edit, etc.

\subsection{Chunk your undos}

As mentioned already, you should chunk your undos as finely as you can by going back into normal mode.
Vim defines a ``change'' as any change in the text.
This could range from deleting a single letter to deleting the whole document (and everything you can think of in between those range).
When using insert mode, vim will define the change as whatever you type in insert mode until you go back into normal mode (i.e. until you have pressed the \verb|<esc>| key).

You should find your own definition of a chunk you are comfortable with and stick with it.
For example, I usually write a sentence up (or part of a sentence up), then always go back to normal mode to keep line-wise chunks rather than any smaller/bigger chunks.

By having smaller chunks for undoing your edit, you are able to go back to your previous text in a finer detail, rather than having to undo a huge chunk and rewriting most of the text you have just undone.

\subsection{Compose repeatable changes: Meet the Dot Command}

When you make a change to your text/document, vim allows you to repeat this change as well as undoing the change (as in previous section).
This means that any changes you make to the text can be repeated on a different part of the text, and therefore won't have to re-type the exact change or commands you have typed earlier (never retype!!).












