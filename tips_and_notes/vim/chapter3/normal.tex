\section{Normal Mode}

Normal mode can be accessed by pressing  the \verb|<esc>| or \verb|<ctrl-[>| command, and is probably the mode that you will be in most of the time.
This is because most of your editing will start in this mode, then switch into another mode to complete your editing task.
In some cases, you probably won't have to leave the normal mode at all (for example deleting a word or a line).

An analogy used in the book was of a painter -- painters, when drawing up a nice piece of artwork, do not leave their paintbrush on the canvas.
Instead, they rest their paint brush, look at it and think about it for a while, and then decide what to do next.
Whatever they do next can be anything from picking a different colour of paint to paint in, or it could be a different kind of brush to give different texture.
Vim's normal mode works just like that -- it gives you ``resting'' mode to think about what to write, what words to edit, how to edit, etc.

\subsection{Operator + Motion = Action}

This formula is the most powerful tool and rule you will use in normal mode.
In normal mode, there are keys known as the operator keys, and keys known as the motion keys.
Together, they can perform an action over certain parts of text, depending on the combination of the operator and the motion keys you have pressed.

\subsubsection{Operator keys}

Operator keys are the core components of all the editing commands you carry out in normal mode -- and there are a lot of them!
The easiest way to learn them (other than using the keys repeatedly) is to learn them by their mnemonics (or maybe what they look like).
\newline

\begin{tabular}[c]{c|l}
    c            & change\\
    d            & delete\\
    y            & yank\\
    g$\sim$      & swap case\\
    gU           & make uppercase\\
    gu           & make lowercase\\
    =            & autoindent\\
    \textgreater & shift to right\\
    \textless    & shift to left\\
\end{tabular}
\newline

Also note that vim will wait for you to press a motion key after you have pressed the operator key -- this is called the ``operator-pending mode''.
This means that you can press the operator key and have enough time to think about what motion you will need to do the right edit.

\subsubsection{Motion keys}

Motion keys are the keys that allow you to move around the document while in normal mode.
It is also the means to specify how far the operator commands should take action on.
Like the operator keys, you should learn it by their mnemonics.
\newline

\begin{tabular}[c]{c|l}
    h/j/k/l & default cursor movement\\
    w/W     & go to next word or WORD\\
    b/B     & go back a word or WORD\\
    e/E     & go to the end of the word or WORD\\
    p       & paragraph\\
    0/\$    & beginning/end of the current line\\
    gg/G    & start/end of the current file\\
    t/T     & (un)til the character, searching forward/backward\\
    f/F     & find the character, searching forward/backward\\
\end{tabular}

\subsubsection{Text objects}

Text objects define a regions of text by structure, and consists of two characters.
The first character is either an ``i'' or an ``a'', and the second character can be anything that describes either a bounded or delimited text object.

\begin{description}
    \item [Bounded text objects] are defined by boundaries (e.g. words, sentences, and paragraphs).
    \item [Delimited text objects] are anything in between a pair of matching symbols (e.g. (), [], \{\}, and xml tags).
\end{description}

The ``i'' stands for inside, whereas ``a'' stands for around/all.
With the bounded text objects, ``i'' will capture the current word and ``a'' will capture the word with a single white space (if any).
With the delimited text objects, ``i'' will capture everything inside the specified matching symbols and ``a'' will capture everything inside the matching symbols and the symbols you have specified.
Together with other motion keys or specific matching symbols, text objects can be very useful in defining and capturing parts of the texts that you want to edit.

\subsubsection{Action}

When you combine the operator and the motion keys and/or text objects together, it will edit the document as demanded.
For example, \verb|dw| will delete the word (from the current position of the cursor to the end of the word), and \verb|cw| will change the word (from the current position of the cursor to the end of the word).
If you use the text object, it's even better -- it will delete/change the word your cursor is currently positioned, instead of going from the current cursor position to the end of the word.

If you press the operator key twice (e.g. \verb|dd|), it will do the action on the current line.
So, \verb|dd| will delete the current line, and \verb|>>| will indent the current line to your right.

\subsection{Chunk your undos}

As mentioned already, you should chunk your undos as finely as you can by going back into normal mode.
Vim defines a ``change'' as any change in the text.
This could range from deleting a single letter to deleting the whole document (and everything you can think of in between those range).
When using insert mode, vim will define the change as whatever you type in insert mode until you go back into normal mode (i.e. until you have pressed the \verb|<esc>| key).

You should find your own definition of a chunk you are comfortable with and stick with it.
For example, I usually write a sentence up (or part of a sentence up), then always go back to normal mode to keep line-wise chunks rather than any smaller/bigger chunks.

By having smaller chunks for undoing your edit, you are able to go back to your previous text in a finer detail, rather than having to undo a huge chunk and rewriting most of the text you have just undone.

\subsection{Compose repeatable changes: Meet the Dot Command}

When you make a change to your text/document, vim allows you to repeat this change as well as undoing the change (as in previous section).
This means that any changes you make to the text can be repeated on a different part of the text, and therefore won't have to re-type the exact change or commands you have typed earlier.
Ideally, you should type one editing command (e.g. daw to delete a word) then repeat this command by using the dot (\verb|.|).

When you press \verb|.|, vim repeats the last editing command that altered the text.
For example, if you opened a new line and typed ``hello'' then escaped, the dot command will repeat your edit -- i.e. it will make a new line saying ``hello''.
Making a new line saying ``hello'' is not that helpful, but with other useful editing commands, the dot command can be very useful.

For example, if you wanted to make a word uppercase, you can do so by typing \verb|gUaw| or \verb|gUw| while the cursor is on the word you want to uppercase.
What if you wanted to uppercase a bunch of words, spread out throughout the document?
You are not going to retype \verb|gUw| every single time, but instead you can move to the word you want to change the case and press the \verb|.| -- it will remember the last editing command (\verb|gUw|) and uses it on the word.
Pretty neat, huh?

\subsection{Use counts to do simple addition and subtraction}

If you have numbers in your document which you would like to increment or decrement, use \verb|<ctrl-a>| and \verb|<ctrl-x>|.
When you have the cursor on the number, press \verb|<ctrl-a>/<ctrl-x>| to increment/decrement respectively.
If you are at the beginning of the line, pressing these keys will jump to the first number (if there is any), and does the incrementing/decrementing for you.

\subsection{Combine numbers with commands}

If you know exactly how many words/lines/paragraphs to edit, you can prefix the command with numbers to repeat the commands n times.
For example, when you type \verb|c2w|, you would drop into insert mode after deleting two words that are ahead of the cursor.
It may be quicker in some situations to use the dot command.

\subsection{Build custom operator and motion keys}

You can build your custom motion keys by using the \verb|omap| command in the vimrc (I think).
See \verb|:h :map-operator| and \verb|:h :omap-info|.
