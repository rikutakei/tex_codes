\chapter{Copy and Paste}

Copying and pasting text are essential features of any text editors.
I have already mentioned how to yank/change/delete/paste (\verb|y/c/d/p|) in vim, but there is a little bit more to it than just simply copy/deleting/pasting.

\section{Registers}

There are some useful features in vim that will allow you to store and paste multiple different pieces of text in your document.
The key to this feature are the registers.
Any key on the keyboard can be used as a register to store pieces of text, and you can pull out the text that is stored in the specified register.

To specify which register to store/use, you have to specify the register by pressing \verb|"{register}|, kind of like when you are setting marks in documents (see \textit{\ref{subsec:marks}:\nameref{subsec:marks}}).
For example, to yank a word into the register \verb|a|, you would press \verb|"ayiw| while the cursor is on the word you want to store.

The basic syntax of copying/pasting (and some other commands) using a register is \verb|"{register}y/p[motion]|.
When no register is specified for paste/copy, the \verb|"| register is used by default (known as the unnamed register).

\section{Named registers}

Vim has a register for each letter of the alphabet, meaning that we have 26 unique registers that we can store different pieces of text.
This is especially useful when you want to copy/paste/delete multiple text, but want to use it again another time.

\section{Special registers}

There are registers that have special function in vim and reserved for specific purposes.

\subsection{Unnamed register (\texttt{""})}

Unnamed register is used by \verb|x/s/d/c/y| commands, and stores the text that has been deleted or yanked as a result of the command.
So, whenever you remove a text from normal mode, the text will be saved to the unnamed register.

\subsection{Yank register (\texttt{"0})}

When \verb|y{motion}| is used, it saves not only to the unnamed register, but also saves it to the yank register (\verb|"0|).
This register is only used by the \verb|y{motion}|, and therefore unaffected by other deleting/substituting commands like \verb|s/d/x/c|.
Meaning that you are able to copy a piece of text, then change a different bit of text and paste the text you have yanked using the yank register.

\subsection{Black hole register (\texttt{"\_})}

The underscore register is the place where nothing is returned (i.e. nothing is saved).
This is useful when you don't want to overwrite a different register (e.g. unnamed register).

\subsection{System clipboard register (\texttt{"+})}

If you want to copy/paste to/from an external program, you need to access the system clipboard.
You can paste text from another place by using \verb|"+p| (or \verb|<ctrl-r>+| in insert mode).
If you want to use a text from vim, you can store the text in \verb|"+| and use paste as usual (\verb|<ctrl-v>|) in another place.

If it doesn't work on ubuntu properly, try installing the vim-gnome package.

\subsection{Expression register (\texttt{"=})}

When you access this register, it drops down into Ex command prompt.
Vim will evaluate the expression you have typed and stores the result in the register.
You can paste the result in insert mode using the \verb|<ctrl-r>=| keys.

\subsection{Read-only register}



