\chapter{Global commands}

The global command combines the Ex commands with Vim's pattern matching abilities.
It allows you to run Ex command on each line that matches a specific pattern.

\section{Syntax of global command}

{\large\verb|:[range] global [!] /{pattern}/ [range for cmd] [cmd]|}
\newline

The default range of the \verb|:global| command is the whole file (\verb|%|).
The \verb|{pattern}| can be left blank to use the previous search pattern.
\verb|[cmd]| can be any Ex command (except for another \verb|:global| command), and the default command is the \verb|:print| command.
The \verb|!| tells it to invert the command (i.e. do the \verb|cmd| on everything that doesn't match the pattern).

\section{Delete or keep matching lines}

\begin{tabular}{c|l}
\verb|:g/{regular exp}/d| & this would delete all the lines that contain the regular expression\\
\verb|:v/{regular exp}/d| & \verb|:v| stands for \verb|:vglobal| (inverted global, i.e. \verb|:global| with \verb|!|).\\
& This will delete all the lines without the regular expression.\\
\end{tabular}

\section{Collect certain lines from a file}

For example, you need to extract all ``TODO'' comments in a file.
You can do this by \verb|:g/TODO|, which prints it out.
If you want to save it in a register, do \verb|:g/TODO/yank A|, which will save it to the register \verb|a| (capital A tells it to append to the specified register).
Now you can paste or do whatever using this register.

\section{Alphabetise a selection}

You can do this by using the Vim's sort command.
Say if you needed to do this to a block of lines, and there were multiple blocks in a file, which will be difficult to do it one by one, you can hit all of them at once using the global command.
Note that the \verb|cmd| can take a range as well, independent of the \verb|range| for the \verb|:global| command.
The \verb|.| address stands for the line that the cursor is positioned, but in \verb|:g|, it stands for the line that matches the \verb|pattern|.

If the block was in between \verb|{| and \verb|}|, you can do \verb|:g/{/ .+1,/}/-1 sort|.
The \verb|+1| and \verb|-1| is the offset to specify only the content of the block.
Note that \verb|/{/| searches for the closing bracket.

In general: \verb|:g/{start}/ .,{finish} [cmd]| will do the work to each of the specified range.

\section{Muting messages from \texttt{cmd}}

Sometimes \verb|cmd| sends out a message, which can be annoying.
You can silent this by using \verb|:silent| (or \verb|:sil|) to suppress the messages.
For example, \verb|:g/{/sil .+1,/}/-1 >| will indent the lines specified with no messages.
