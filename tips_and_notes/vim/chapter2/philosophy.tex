\chapter{The ``Vim Way''}

\section{Get used to the Modal Interface}

Vim is set up in such a way that every key on the keyboard has a different meaning depending on the current ``mode'' of the text editor.
There are three major modes in vim (all of which we will talk about later):
\begin{description}
    \item [Normal Mode] The default mode of vim for editing text.
    \item [Insert Mode] Primary mode for inserting/writing text into the buffer/file.
    \item [Visual Mode] Allows you to select text in a buffer/file. The selected text can be used for copy/paste/manipulation/substitution/etc.
\end{description}

Each mode has a specific purpose and the keys are laid out to make it easier to carry it out (although most keys have the same, if not similar, meanings -- which makes it easier to remember than emacs!).
For example in insert mode, all the keys are set out so that we can type in the buffer.
On the other hand, the primary purpose of normal mode is to move around the buffer and manipulate the text, so each key has a different meaning (e.g. h/j/k/l to move around).

This allows us to focus on one task at a time, and also allows us to choose the right tool to carry out specific tasks.

\section{Never leave the home row}

One of the most annoying thing while programming is having to go back and forth between the mouse and the keyboard.
For example, you start coding for a bit, and you found a wee bug few line up the screen.
You pause and use your mouse to click where you want to delete/add text, fix your mistakes and click back to your original position to carry on coding.

This doesn't sound too bad.
But what if you compiled your program and found multiple errors within a file and also in other files as well?
Sure, you can open multiple screens or windows of your favourite text editor and use your mouse to scroll up and down the screen, clicking, and selecting for the piece of text you want to edit.
Important thing to note here is that every single time you leave the keyboard to reach for the mouse, you are not typing.
Going back and forth might not take up that much time, but it will add up.
Also, you'll probably get annoyed if you had to change a tiny bit of text at a time, especially if you had hundreds of them to go through.

Luckily, vim has a modal interface -- modes like normal mode and visual mode are designed so that you don't have to reach for your mouse (or the arrow keys) as much, if not at all.
We will meet these modes later on.

\section{Act, Repeat, Reverse}

When using vim, everything should be set up so that you can ``act, repeat, reverse''.
What I mean by this is that you should use vim in such a way that you edit a bit of text (act), then repeat the same edit on different parts of the text (repeat), and make all of his reversible (reverse).
We will meet the dot-command in the next chapter.

When going through this book, you should always think about how you are going to type less, automate tasks, and how to prevent yourself typing the same text over and over again.
The main goal is to do a single edit, then make this repeatable so that you don't have to re-type it again later, and if you happen to carry out the changes too many time, you should be able to reverse the change.

\section{Edit your text in chunks}

In vim, it is easy to write everything in one single burst in the insert mode (like in other text editors and MS Word), but this should be avoided.
Since vim's undo command (\verb|u|) reverts the last change made to  the text, if you type everything all at once you run the risk of undoing the whole text.
Rather than doing this, you should try and  change the text bit by bit to make sure your undo command doesn't undo everything, but only the small incremental changes you have made to the document.

\section{Navigate in chunks}

Hopefully, you know how to navigate simply in vim by now (the \verb|hjkl| keys).
When you're navigating yourself in vim, you should try and use more keys that skips more than one letter (i.e. use \verb|h| and \verb|l| keys less).
You should use the keys that allow you to skip multiple words or lines at a time, like the \verb|w/b| keys and \verb|<ctrl-f>/<ctrl-b>| keys.
