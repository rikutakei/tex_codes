\chapter{Quick Introduction}

\section{Why I started using Vim}

I used to use emacs during my undergrad years, but I never really mastered  it (or even become advanced at it).
The farthest I got with emacs was cutting/pasting lines, undoing mistakes, and moving cursor to the start/end of the line, which was enough to get through all the coding needs I had at the time.

I never hated emacs (I actually quite liked it), but I did notice the awkwardness of all the shortcuts that emacs provided for the user.
Most, if not all, shortcuts were prefixed with a \verb|<C-x>|, and some of the letters that followed were not as informative to remember.
Furthermore, emacs uses not only the control-key, but alt-key, shift-key, super-key, and other modifier keys as well (and each modifier key in combination with letters has different meanings!).
This really confused me (and annoyed me a bit), as you had to remember the right key combinations to carry out whatever you wanted it to do, and if you happened to accidentally press another modifier key, well, you are in for a surprise!

My point here is that emacs provides the user with so many shortcuts that will allow the user to do everything and anything by using all the different key combinations that you'll either forget or mistype.
There's just too many of them!!

When I began my postgrad study, I thought, ``If I was to keep coding throughout my study, I would need to be able to code faster''.
But when I started looking at the shortcuts that emacs provided, I was not motivated at all (for the very reason I mentioned earlier).
That is when I heard about vim and how efficient it was.

So, I started looking at how vim worked and what could be done with it.
What are the advantages?
How are they useful?
What can you do with it?
At first, I was weirded out by the way vim worked, like the modal interface, but I soon became used to it and agreed with the idea of not leaving the home row.
I also discovered multiple plugins that gave the features of the emacs that I loved in the days of undergrad comp sci courses.
The emacs features were based on the configuration file on the comp sci computer (which I could not reproduce myself), hence I was getting frustrated with the raw emacs that I have installed.

As I read more and more about vim, the urge to learn vim properly became larger.
I had plenty of other options to learn before vim (for example, another programming language), but I chose vim to be my first target - if I could master how to edit text faster, then whatever that follows will also (probably) become faster.
So, I decided to learn vim using the vim tips book ``Practical Vim'' written by Drew Neil.
This part of the book will mostly contain the tips that I thought was, or could be, very useful from Drew Neil's book, so if you want more details, go and read his book.

Also, just a note before I start - most sentences will probably be incomplete sentences (bullet-point like), as I am not writing this as a step-by-step guide to vim.
This book is just a bunch of tips I pulled out from his book.
