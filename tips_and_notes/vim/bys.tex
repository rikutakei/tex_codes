\section{Before you start}

Firstly, make sure you can touch type.
It doesn't have to be super fast or anything like that - just know where to find the keys that you are looking for (including special symbols like \%).
A good website I found was \url{www.keybr.com}.
It's free to sign up, and keeps all the record of what you have typed and what keys you need to work on.

Secondly, do the Vim tutor on command line (type \verb|vimtutor| in terminal).
This tutorial will get you through the basics of vim and how to get around files, save files, close files, etc.
Vim tutor will be a good starting point before using vim.

Lastly, have a google around for the spf13-vim plugin.
This plugin downloads other plugins that make vim awesome (e.g. tab completion).
Starting off with this plugin will make your life a lot easier, but remember to make your own .vimrc file once you get the hang of all the commands and (re-)mappings.
