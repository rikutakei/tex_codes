\chapter{Navigation within files}

\section{Basic navigation}

The idea here is to keep your fingers on the home row, no matter what.
Moving your hands from the keyboard to the mouse is going to waste your time.

\subsection{Moving letter-wise}

To move around in the document one cursor (i.e. one letter/character) at a time, press the \verb|h/j/k/l| keys, which corresponds to the left/down/up/right arrow keys.
It is easy to reach out onto your arrow keys instead of using these keys, but, like with reaching for the mouse, it will waste your time.

To force yourself to use the \verb|h/j/k/l| keys, you should map the arrow keys to do nothing: \verb|noremap <Up> <Nop>| (and do it for left/right/down keys).

\subsection{Moving word-wise}

You can move around bits of text character-wise, but moving word-wise would be faster.
The \verb|w/b/e/ge| commands will allow you to move word-wise through the text, each offering slightly different movement.
\verb|w/b| moves your cursor forward/backward to the start of the next word, and \verb|e/ge| moves your cursor forward/backward to the end of the next word.

\subsection{word vs WORD}

WORDS are sequence of any non-blank characters, whereas words are sequence of letters, digits, and underscores, or as a sequence of other non-blank characters.
For example, \verb|www.google.com| has 5 words, but only 1 WORD.
To move WORD-wise instead of word-wise, you will have to press the uppercase version of the word-wise commands (i.e. \verb|W/B/E/gE|).

\subsection{Moving to a specific position on a line}

Press \verb|0/\$| to move to the beginning/end of the line.
To go to the first non-blank character of the line, press \verb|^|.
