\chapter{Substitution}

Substitution is very much like search, but allows you to replace the search pattern with another string.

\section{Syntax of substitute command}

\begin{center}
    \LARGE{\verb|:[range]s/{pattern}/{string}/[flags]|}
\end{center}

\section{Tweaking substitute command using flags}

There are different flags you could use in the substitute command to suit the requirement(s).
Some of the flags are listed below (see \verb|:h :s_flags| for a full reference).
\newline

\begin{tabular}{c|l}
\verb|g| & use substitution command globally on the current line\\
\verb|c| & allow you to confirm or reject the changes each time\\
\verb|n| & suppresses the normal substitution and report the number of occurrences\\
\verb|&| & reuse the previous flags\\
\verb|e| & suppress error messages\\
\end{tabular}

\section{Special characters for the replacement string}

There are some special characters you can use in the replacement string.
Some of them are listed below (see \verb|:h sub-replace-special| for full reference).
\newline

\begin{tabular}{c|l}
\verb|\r|             & \verb|<CR>| \\
\verb|\t|             &  tab\\
\verb|\\|             & single backslash \\
\verb|\{digit}|       & submatches \\
\verb|\0|             & entire matched pattern\\
\verb|&|  & entire matched pattern\\
\verb|~|              &  use the \verb|{string}| from the previous substitution\\
\verb|\={vim script}| & evaluate vim script and replace \verb|{string}| with the result of script\\
\end{tabular}

\section{Find and replace every match in a file}

Note in the previous section that the \verb|g| flag only works on the current line.
To act on the whole file, you need to use \verb|%| (all the lines in a file) as the range.

\section{Check each substitution}

Use the \verb|c| flag to check/confirm if you want to substitute it or not (e.g. \verb|:%s/{pattern}/{string}/gc|)
\newline

\begin{tabular}{c|l}
\verb|y| & ``yes''/substitute\\
\verb|b| & ``no''/don't substitute\\
\verb|q| & quit substitution\\
\verb|l| & ``last'' substitution (substitute this match and quit)\\
\verb|a| & ``all'' (substitute this and any remaining matches)\\
\verb|<ctrl-e>| & scroll screen up\\
\verb|<ctrl-y>| & scroll screen down\\
\end{tabular}

\section{Re-using the last search pattern}

Leave the \verb|{pattern}| field empty to use the last search pattern.
If you want to, you can paste it into the substitute command with \verb|<ctrl-r>/| (register where the search pattern is stored).
This will keep the full record of what you have substituted on the command history, rather than a blank line.

\section{Replace pattern with the contents of a register}

You can do this by typing \verb|<ctrl-r>{register}|.
This may cause problems of the contents of that register contains special characters or have multiple lines (in which case you will have to escape each character).
Instead of pasting the contents directly into the substitute command, you can reference the register bu evaluating it as a vim script (e.g. \verb|:%s//\=@0/g|).

The \verb|\=| tells vim to evaluate a vim script expression, therefore the command tells vim to substitute the pattern using the contents of \verb|@0| (yank register).
By using the  register reference, you run the risk of substituting different patterns with different words when we originally ran the \verb|:%s//\=@0/g| command.
But we can take advantage of this behaviour by searching for a different pattern, then yanking a different replacement.
When you run the exact same command, it will replace your search pattern with the yanked text.

\section{Repeating the previous substitute command}

If you missed the \verb|%| symbol to make it a file-wise substitution, you can just press \verb|g&|, which is equivalent to \verb|:%s///~/&|.
Some useful commands are listed below (see pg 227 for good mappings):
\newline

\begin{tabular}{c|l}
\verb|:&| & repeat the last substitution command on this line (flag ignored)\\
\verb|:&&| & same as \verb|:&| but use the flags in previous substitution command.\\
& Note that this can be used with visual selection\\
\verb|:%&&| & same as \verb|:&&| but act file-wise (equivalent to \verb|g&|)\\
\end{tabular}
