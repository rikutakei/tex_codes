\chapter{Quick fix and location list}

\section{Compile code and use quick fix list}

You are able to run \verb|:make| in Vim, instead of having to run make in terminal.
When errors are given, vim automatically parses each line and extracts the filename, line number, and error messages.
For each warning, vim creates a record in the quick fix list.
You can jump through the quick fix list, which takes you to the location of the error.
Some of the commands are listed below (see \verb|:h quickfix| for full list):\\

\begin{tabular}{c|l}
    \verb|:cnext|     & jump to the next item\\
    \verb|:cprevious| & jump to the previous item\\
    \verb|:cfirst|    & jump to the first item\\
    \verb|:clast|     & jump to the last item\\
    \verb|:cfile|     & jump to the first item in the next file\\
    \verb|:cnfile|    & jump to the first item in the next file\\
    \verb|:cpfile|    & jump to the last item in the next file\\
    \verb|:cc N|      & jump to the N$^{th}$ item\\
    \verb|:copen|     & open the quick fix list\\
    \verb|:cclose|    & close the quick fix list\\
\end{tabular}
\newline

You can prepend the \verb|:cnext| and other commands with numbers to go through them $n$ times at a time (e.g. \verb|:5cnext| will go through 5 errors at a time).
If you want to skip all errors in current file and start with the next file, use \verb|:cnfile|.
You can go through and choose the errors visually by using the quick fix window, and use \verb|k/j| keys to scroll.
You can also use search within this window.
To close the window, type \verb|:q| or \verb|:cclose|.

Vim holds onto the last ten quick  fix list,  and you can load them up using \verb|:colder| or \verb|:cnewer| commands.
Again, this can be used with numbers to go back to the n$^{th}$ quick fix list.
