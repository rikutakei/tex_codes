\section{Insert Mode}

This mode will be self explanatory -- it is the mode you type text into the document.
However, there are tricks and work-arounds that you can use in insert mode so that you don't have to always go back into normal mode.
To go back into normal mode, you can either press \verb|<esc>| or \verb|<ctrl-[>|.

\subsection{Deleting from insert mode}

\begin{tabular}{c|l}
    \verb|<ctrl-h>| & delete one character\\
    \verb|<ctrl-w>| & delete one word\\
    \verb|<ctrl-u>| & delete one line\\
\end{tabular}

\subsection{Insert-normal mode}

Insert-normal mode is a mode that allows you to go back into normal mode for just one command, and then switches automatically back into insert mode.
To use it, while in insert mode, press \verb|<ctrl-o>|.
This will put you into insert-normal mode, and after you have entered a command in this mode, it will switch you back into insert mode.
\verb|zz| is a normal mode command that redraws the screen so that the current line is positioned in the middle of the screen.
This can be quite useful in insert-normal mode, especially if you want to reposition the screen and get back into the typing immediately.

\subsection{Paste text while in insert mode}

You can paste text from certain registers by using \verb|<ctrl-r>{register}|.
This will paste the text stored inside the register that you have specified into the current cursor position.
Use \verb|<ctrl-p>{register}| to paste the text literally without indentation, and other nonsense (e.g. carriage return).

\subsection{Use expression register to insert calculations and other complicated stuff}

The ``='' symbol is a special register that allows you to paste the result of the expression typed in.
When you type \verb|<ctrl-r>=|, this will open up a prompt which you can type expression into.
For example, you could type 3576*235 into this, and once you hit the enter button, the result should be inserted into the text (by the way, the answer to that equation is 840360).
