\documentclass[a4paper, 11pt]{article}

\usepackage[utf8]{inputenc}
\begin{document}

\section*{Summary so far}

\subsection*{Creighton data}

Creighton et al. found obesity-associated genetic signatures that are specific to obese samples, but not in the lean or overweight samples, in a study using breast tumour samples from patients.
Total of 103 samples were used in the study, majority (about 70 odd samples) were caucasian, and rest of them black or asian.
Expression of genes from the samples were quantified using a microarray chip.
The samples were split into three groups: obese, overweight, and lean samples.

They initially looked at the difference in the gene expression between lean vs. non-lean, but they switched it to obese vs. non-obese group (read paper again).
(I think they did an ANOVA to see the expression difference between all three groups)
From this, they found 799 genetic signatures (600 odd unique genes) that were specific to the BMI status of the samples (i.e. the genetic signatures were able to identify obese  samples from the non-obese samples).
These were identified as obesity specific genes, as they did not correlate (?) with any other clinical variable (judged by the heatmap??).

They identified the genes to be involved in AKT/mTOR pathway, and further validated the results using drugs specific to these on mice with breast cancer (?, maybe it was the Fuentes-Mattei paper).
With these results, we decided to see if these obesity-associated genetic signatures were able to predict the BMI status of the samples/patients in other tumour types from TCGA database.

\subsection{TCGA}





\end{document}
